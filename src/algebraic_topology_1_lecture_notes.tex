\documentclass[wip, topology]{bsteffan-lecturenotes}

\addbibresource{references.bib}

\colorlet{knowngray}{darkgray!70}

\course{Algebraic Topology \uppercase\expandafter{\romannumeral 1\relax}}
\subtitle{The Serre Spectral Sequence, Characteristic Classes, and Bordism}
\lecturer{Prof. Dr. Markus Hausmann}
\assistant{Dr. Elizabeth Tatum}
\author{Ben Steffan}

\begin{document}
\maketitle
\tableofcontents
\listoflectures

\setcounter{section}{-1}
\section*{About These Notes}
This document encompasses lecture notes for the course \makeatletter\@course\makeatother\ taught in the winter term of 2023/24 at the University of Bonn by \makeatletter\@lecturer\makeatother.
The assistant is \makeatletter\@assistant\makeatother.

These notes are for private use. 
They are \strong{not} official lecture notes endorsed by the lecturer.
As such, errors and inaccuracies that persist are generally my own (unless proven otherwise).

This document is not a character-for-character transcript of the lecture.
Changes to form (though generally not to content) have been made to improve readability of these notes as a document.
In particular, I have taken the liberty to make adjustments to notation here and there to more closely align with my personal tastes and opinions.
At points, I have added additional context, explanations, computations, and so on.
These are clearly marked to that effect, although smaller changes and in-text additions (such as citations) are not.

\subsection*{Formatting}
This document has hyperlinks: References, footnote marks, table-of-contents entries and so on are linked and can be clicked to take you to the corresponding item.
Except for footnote marks, which remain black, all such links are highlighted in either \textcolor{linkcol}{orange} or \textcolor{citecol}{violet}. 
\textcolor{highlightcol}{Red} is used to highlight certain items formulas and diagrams.
The colors \textcolor{definitioncol}{green}, \textcolor{exercisecol}{blue}, and \textcolor{theoremcol}{red} are used as border colors to higlight definitions, exercises, and theorems, propositions, lemmas, etc., respectively.
\textcolor{knowngray}{Gray} is used to convey known or secondary information in formulas and diagrams from place to place.

Demarcations for lecture dates are placed in the righthand margin.

\section{Informal Introduction}\lecture{09.10.23}
\begin{note}
	We omit the coefficient group $\Z$ from notation in (co)homology as well as basepoints where they are of no particular relevance in homotopy groups.
\end{note}
One of the big goals of homotopy theory is to compute
\[
	[X, Y]_* = \{\text{basepoint-preserving continuous maps } X \to Y\} / \text{homotopy}
\]
for $X$ and $Y$ pointed CW-complexes.
CW-complexes are built out of spheres, so the building blocks are the sets
\[
	[S^n, S^k]_* = \pi_n(S^k, *)
\]
For $n \geq 1$ these are groups and abelian if $n \geq 2$.
We know that\textellipsis{}
\begin{itemize}
	\item $\pi_n(S^k) = 0$ for $n < k$ by cellular approximation, cf. \cite[Corollary 4.9]{hatcher_algebraic_2002},
	\item $\pi_n(S^n) \isom \Z$ by the Hurewicz theorem (cf. \cite[Theorem 4.32]{hatcher_algebraic_2002}) and $H_n(S^n) \isom \Z$: 
		If $X$ is an $(n - 1)$-connected CW-complex ($n > 1$), then there is an isomorphism $\pi_n(X) \xto{\isom} H_n(X)$, 
	\item $\pi_k(S^1) = 0$ for $k \geq 2$ via covering space theory: The universal cover of $S^1$ is $\R$ which is contractible, 
	\item $\pi_3(S^2) \neq 0$ since the attaching map of the 4-cell for $\CP^2$ is a map $\eta\colon S^3 \to S^2 = \CP^1$; if $\eta$ was nullhomotopic, then $\CP^2$ would be homotopy equivalent to $S^2 \vee S^4$ which contradicts the ring structure on $H^*(\CP^2) \isom \Z[x] / x^3$, and that
	\item The sequence
		\begin{equation*}
			\pi_k(S^n) \to \pi_{k + 1}(S^{n + 1}) \to \pi_{k + 2}(S^{n + 2}) \to \cdots
		\end{equation*}
		always stabilizes by the \emph{Freudenthal suspension theorem} (see \cite[Corollary 4.24]{hatcher_algebraic_2002}).
\end{itemize}
To go beyond this, we will need a new tool, the \emph{Serre spectral sequence}.
To motivate its usefulness for this question, consider the following strategy:
There exists a map $f\colon S^2 \to K(\Z, 2)$ which induces an isomorphism $f_*\colon \pi_2(S^2) \xto{\isom} \pi_2(K(\Z, 2)) \isom \Z$.
We can take its homotopy fibre $H \coloneq \hofib(f_*)$; there is then a fibre sequence $H \to S^2 \xto{f} K(\Z, 2)$ and thus a long exact sequence\footnote{The groups in \textcolor{knowngray}{gray} are assumed known. The groups and properties in \textcolor{col05}{red} follow from those in gray.}
\begin{equation*}
	\begin{tikzpicture}[commutative diagrams/.cd, every diagram, row sep = large]
		\matrix[matrix of math nodes, name = m, commutative diagrams/every cell] {
				& \cdots & \pi_4(K(\Z, 2)) \\
			\pi_3(H) & \pi_3(S^2) & \pi_3(K(\Z, 2)) \\
			\pi_2(H) & \pi_2(S^2) & \pi_2(K(\Z, 2)) \\
			\pi_1(H) & \pi_1(S^2) & \pi_1(K(\Z, 2)) & 0 \\
		};
		\path[commutative diagrams/.cd, every arrow, every label]
			(m-1-2) edge (m-1-3)
			(m-1-3) edge[rounded corners, to path = {
				-- ([xshift = 1em] \tikztostart.east)
				|- ($(m-1-2)!0.5!(m-2-2)$) \tikztonodes
				-| ([xshift = -2ex] \tikztotarget.west)
				-- (\tikztotarget)
			}] (m-2-1)
			(m-2-1) edge["{\color{highlightcol}\isom}"] (m-2-2) 
			(m-2-2) edge (m-2-3)
			(m-2-3) edge[rounded corners, to path = {
				-- ([xshift = 1em] \tikztostart.east)
				|- ($(m-2-1)!0.5!(m-3-1)$) \tikztonodes
				-| ([xshift = -2ex] \tikztotarget.west)
				-- (\tikztotarget)
			}] (m-3-1)
			(m-3-1) edge (m-3-2) 
			(m-3-2) edge["$f_*$", "$\isom$"'] (m-3-3)
			(m-3-3) edge[rounded corners, to path = {
				-- ([xshift = 1em] \tikztostart.east)
				|- ($(m-3-1)!0.5!(m-4-1)$) \tikztonodes
				-| ([xshift = -2ex] \tikztotarget.west)
				-- (\tikztotarget)
			}] (m-4-1)
			(m-4-1) edge (m-4-2) 
			(m-4-2) edge (m-4-3)
			(m-4-3) edge (m-4-4);
		% don't want our labeling nodes to be the usual node grid distance away
		% better to set this via scope than to either pass it to each \node separately or set it for the whole picture
		\begin{scope}[node distance = -5.25pt, knowngray] 
			\node[below = of m-1-3] {$0$};
			\node[below = of m-2-3] {$0$};
			\node[below = of m-3-1, highlightcol] {$0$};
			\node[below = of m-4-1, highlightcol] {$0$};
			\node[below = of m-4-2] {$0$};
			\node[below = of m-4-3] {$0$};
		\end{scope}
	\end{tikzpicture}
\end{equation*}
Hence, $H$ is 2-connected and $\pi_n(H) \to \pi_n(S^2)$ is an isomorphism for all $n \geq 3$.
By the Hurewicz theorem, the following diagram commutes:
\begin{equation*}
	\begin{tikzcd}
		\pi_3(H)
				\ar[r, "\isom"]
				\ar[dr, swap, "\isom"]
			& H_3(H)
		\\
			& \pi_3(S^2)
				\ar[u, swap, "\isom"]
	\end{tikzcd}
\end{equation*}
If we had a way to compute $H_*(H)$ from $H_*(S^2)$ (the computation of which is easy) and $H_*(K(\Z, 2))$ (which is known), we could compute $\pi_3(S^2)$ this way!

\subparagraph{Upshot}
It would be useful to have a tool which relates the homology groups of the three terms in a fibre sequence.
This will also help us to compute $\pi_n(S^k)$ in other ways (for example, we will show that $\pi_n(S^k)$ is finite unless $n = k$ or $n = 2k - 1$ and $k$ is even).
Furthermore, the Serre spectral sequence will allow us to compute the (co)homology of spaces like $\Uni(n)$, $\SU(n)$, $\Omega S^n$, $K(\Zn{2}, n)$ and (re)prove structural theorems like the Hurewicz theorem, the Freudenthal suspension theorem, Thom isomorphisms, and more.

So, given a fibre sequence $F \to Y \to X$, what could the relationship between the homology groups of $F$, $Y$ and $X$ be?
\begin{example}
	Consider the easiest case $F \to X \times F \xto{\pr_X} X$ (a \emph{trivial fibration}).
	Then the Alexander-Whitney map induces an isomorphism $H_n(X \times F; \Z) \xto{\isom} \bigdsum_{p + q = n} H_p(X, H_q(F))$, so it computes the homology of the total space in terms of the homology of $X$ and $F$.
\end{example}
\begin{example}
	Consider the Hopf fibration $S^1 \to S^3 \xto{\eta} S^2$.
	We can compute
	\begin{equation*}
		\renewcommand{\arraystretch}{1.1}
		\begin{tabular}{r|c|c}
			$n$ 		& $H_n(S^3; \Z)$ 	& $\bigdsum_{p + q = n} H_p(S^3; H_q(S^1; \Z))$ \\\hline
			0 			& $\Z$ 				& $\Z$ \\
			1 			& $0$				& $\Z$ \\
			2 			& $0$				& $\Z$ \\
			3 			& $\Z$ 				& $\Z$ \\
			4 			& $0$				& $0$ \\
			$\vdots$ 	& $\vdots$			& $\vdots$
		\end{tabular}
	\end{equation*}
	so $\bigdsum_{p + q = n} H_p(S^3; H_q(S^1; \Z))$ is in some sense \enquote{too big} to describe $H_n(S^3; \Z)$ in degrees $n = 1, 2$.
	Note, however, that we can consider a \enquote{2-step filtration} $S^1 \subseteq S^3$ which satisfies $\tilde{H}_n(S^3 / S^1; \Z) \isom \Z$ if $n = 2, 3$ and $0$ else.
	Then
	\begin{equation*}
		\renewcommand{\arraystretch}{1.1}
		\begin{tabular}{r|c}
			$n$ & $H_n(S^1; \Z) \dsum \tilde{H}_n(S^3 / S^1; \Z)$ \\\hline
			0 	& $\Z$ \\
			1 	& $\Z$ \\
			2 	& $\Z$ \\
			3 	& $\Z$ \\
			4 	& 0 \\
			$\vdots$ 	& $\vdots$ \\
		\end{tabular}
	\end{equation*}
	This does not agree with $H_3(S^3; \Z)$ because in the long exact sequence
	\begin{equation*}
		\begin{tikzcd}[row sep = large]
			\cdots 
					\ar[r]
				& H_3(S^3; \Z)
					\ar[r]
				& \tilde{H}_3(S^3 / S^1; \Z)
					\ar[dll, rounded corners, swap, "\del", to path = {[pos = 1]
						-- ([xshift = 1em] \tikztostart.east)
						|- ($(\tikzcdmatrixname-1-2)!0.5!(\tikzcdmatrixname-2-2)$) \tikztonodes
						-| ([xshift = -2ex] \tikztotarget.west)
						-- (\tikztotarget)
					}]
			\\
			H_2(S^1; \Z)
					\ar[r]
				& H_2(S^3; \Z)
					\ar[r]
				& \tilde{H}_2(S^3 / S^1; \Z)
					\ar[dll, rounded corners, swap, "\del", "{\color{highlightcol}\isom}"', to path = {[pos = 1]
						-- ([xshift = 1em] \tikztostart.east)
						|- ($(\tikzcdmatrixname-2-2)!0.5!(\tikzcdmatrixname-3-2)$) \tikztonodes
						-| ([xshift = -2ex] \tikztotarget.west)
						-- (\tikztotarget)
					}]
			\\
			H_1(S^1; \Z)
					\ar[r]
				& H_1(S^3; \Z)
					\ar[r]
				& \cdots
		\end{tikzcd}
	\end{equation*}
	the boundary map $\tilde{H}_2(S^3 / S^1; \Z) \to H_1(S^1; \Z)$ is an isomorphism.
	Hence, $H_1(S^1; \Z)$ does not contribute to $H_1(S^3; \Z)$ and $\tilde{H}_2(S^3 / S^1; \Z)$ does not contribute to $H_2(S^3; \Z)$.
\end{example}
It turns out that something similar holds for all fibre sequences $F \to Y \to X$: 
There exists a filtration on $C_*(Y; \Z)$
\begin{equation*}
	F_0 \subseteq F_1 \subseteq \cdots \subseteq F_m \subseteq \cdots \subseteq C_*(Y; \Z)
\end{equation*}
of chain complexes such that $H_{p + q}(F_p / F_{p - 1}) \isom C^{\text{cell}}_p(X; H_q(F; \Z))$.
To understand $H_*(Y; \Z)$, one needs to understand the cancellations in the associated long exact sequences.
This is best encoded in a \emph{spectral sequence}.

\section{Spectral Sequences}
\begin{definition}
	A (\strong{homologically}\index{spectral sequence!homologically graded}/\strong{Serre graded}\index{spectral sequence!Serre graded}) \strong{spectral sequence}\index{spectral sequence} is a triple $(E^\bullet, d^\bullet, h^\bullet)$ where
	\begin{itemize}
		\item $(E^r)_{r \geq 2}$ is a sequence of $\Z$-bigraded abelian groups.
			We write $E^r_{p, q}$.
			$E^r$ is called the $r$th \strong{page}\index{page!of a spectral sequence} of the spectral sequence.
		\item $d^r\colon E^r \to E^r$ is a sequence of morphisms (called \strong{differentials}\index{differential!of a spectral sequence}) of bidegree $(-r, r - 1)$ satisfying $d^r \circ d^r = 0$.
		\item $h^r\colon H_*(E^r) \to E^{r + 1}$ is a sequence of bigrading-preserving isomorphisms.
			Here $H_*(E^r)$ denotes the homology of $E^r$ with respect to $d^r$, which inherits a bigrading.
	\end{itemize}
\end{definition}
\lecture{13.10.23}
\begin{figure}[ht]
	\begin{tikzpicture}[every node/.append style = {fill = white, font = \scriptsize}]
		\def\xmin{-3}
		\def\xmax{3}
		\def\ymin{-3}
		\def\ymax{3}
		\begin{scope}
			\coordinate (origin) at (0, 0);
			\draw[very thick, ->] (\xmin, 0) -- (\xmax, 0);
			\draw[very thick, ->] (0, \ymin) -- (0, \ymax);
			%\foreach \x \in {\xmin, ..., \xmax}{
				%\foreach \y \in {\ymin, ..., \ymax} {
					%\node at (\x, \y) {$E^2_{\x, \y}$};
				%}
			%}
		\end{scope}
	\end{tikzpicture}
	\caption{$E^2$ and $E^3$ pages of a homologically graded spectral sequence}
\end{figure}
\begin{definition}
	We say a spectral sequence is \strong{first quadrant}\index{spectral sequence!first quadrant} if all the groups $E^2_{p, q}$ are trivial whenever $p < 0$ or $q < 0$.
\end{definition}
\begin{lemma}
	For a first quadrant spectral sequence $(E^\bullet, d^\bullet, h^\bullet)$ we have $E^r_{p, q} = 0$ if $p < 0$ or $q < 0$ for all $r \geq 2$.
	Moreover, for a given $(p, q) \in \Z \times \Z$ the map $h$ induces an isomorphism for all $r > r_0 \coloneq \max(p, q + 1)$, i.e. the groups $E^r_{p, q}$ stabilize as $r \to \infty$.
\end{lemma}
\begin{proof}
	The first statement follows immediately from the existence of $h^\bullet$ by induction on $r$.
	For the second statement, if $r > r_0$, then the target of the differential $d^r\colon E^r_{p, q} \to E^r_{p - r, q + r - 1}$ is trivial since $p - r < 0$, so every element of $E^r_{p, q}$ is a cycle.
	Moreover, the domain of the incoming differential $E^r_{p + r, q - r + 1} \to E^r_{p, q}$ is trivial since $q - r + 1 < 0$, so $E^r_{p, q} \isom H_*(E^r_{p, q}) \isom E^{r + 1}_{p, q}$.
\end{proof}
\begin{definition}
	For a first quadrant spectral sequence $(E^\bullet, d^\bullet, h^\bullet)$ we define its \strong{$E^\infty$-page}\index{$E^\infty$-page} as the bigraded abelian group
	\begin{equation*}
		E^\infty_{p, q} \coloneq E^{r_0(p, q) + 1}_{p, q}
	\end{equation*}
	with $r_0(p, q) \coloneq \max(p, q + 1)$.
	By the previous lemma, $E^\infty_{p, q} \isom E^r_{p, q}$ whenever $r > r_0(p, q)$.
\end{definition}
By a \strong{filtered object}\index{filtered object} $(H, F)$ in an abelian category $\mathcal{A}$ we mean an object $H \in \mathcal{A}$ together with a sequence of inclusions
\begin{equation*}
	0 = F^{-1} \subseteq F^0 \subseteq F^1 \subseteq \ldots \subseteq F^n \subseteq \ldots \subseteq H
\end{equation*}
We will apply this to the category of graded abelian groups and $H = H_*(E; \Z)$.
Notationally, if $(H, F)$ is a filtered object in abelian groups, we write $F^n_m$ for the $n$th object in the filtration associated to the group $H_m$; in other words, $F^0_m \subseteq F^1_m \subseteq \ldots \subseteq H_m$ is the filtration associated to $H_m$.
\begin{definition}
	A first quadrant spectral sequence is said to \strong{converge}\index{convergence!of spectral sequences} to a filtered object in graded abelian groups $(H, F)$ if there is a chosen isomorphism
	\begin{equation*}
		E^\infty_{p, q} \isom F^p_{p + q} / F^{p - 1}_{p + q}
	\end{equation*}
	for all values of $p$ and $q$ and moreover $F^p_n = H_n$ if $p \geq n$.
	In this case, we write $E^2_{p, q} \Rightarrow H$.
\end{definition}
\begin{remark}
	\leavevmode
	\begin{itemize}
		\item Convergence is really a \emph{datum} of the isomorphism $E^\infty_{p, q} \isom F^p_{p + q} / F^{p - 1}_{p + q}$ and not a property.
		\item Convergent spectral sequences are often simply encoded as $E^2_{p, q} \Rightarrow H$, but this suppresses not only this data but also the higher pages, the differentials, and the filtration on $H$!
	\end{itemize}
\end{remark}

\subsection{Fibre Sequences}
In order to be able to move onto the definition of the Serre spectral sequence for fibre sequences, let us define exactly what we mean by \enquote{fibre sequence}.
\begin{definition}
	Let $f\colon X \to Y$ be a map of spaces and $x \in X$ a point.
	The \strong{homotopy fibre}\index{homotopy fibre} $\hofib_x(f)$ of $f$ at $x$ is the space
	\begin{equation*}
		\hofib_x(f) \coloneq P_x X \times_X Y
	\end{equation*}
	where $P_x X = \{\gamma\colon I \to X \mid \gamma(1) = x\}$ is the \strong{based path space}\index{path space} of $X$.
	It comes with the evaluation at 0 map $\ev_0\colon P_x X \to X,\ \gamma \mapsto \gamma(0)$.
	In fact, it is the pullback
	\begin{equation*}
		\begin{tikzcd}
			\hofib_x(f)
					\ar[r]
					\ar[d]
					\ar[dr, phantom, "\lrcorner" very near start]
				& P_x X
					\ar[d, "\ev_0"]
			\\
			X 
					\ar[r, "f"]
				& Y
		\end{tikzcd}
	\end{equation*}
\end{definition}
In words, $\hofib_x(f)$ is the space of pairs $(\gamma, y)$ where $y \in Y$ is a points and $\gamma$ is a path from $f(y)$ to $x$.
We note that $P_x X$ is contractible via the homotopy
\begin{align*}
	H\colon P_x X \times I &\to P_x X \\
	(\gamma, t) &\mapsto (s \mapsto \gamma((1 - t)s + t))
\end{align*}
\begin{example}
	If $* \xto{f} X$ is the inclusion of any point, then $\hofib_x(f) = \Omega_x X$.
\end{example}
\begin{definition}
	A \strong{fibre sequence of topological spaces} is a sequence $F \xto{i} Y \xto{f} X$, a basepoint $x \in X$, and a homotopy $h\colon F \to X^I$ from the composite $f \circ i$ to the constant map $c_x\colon F \to X$ such that the induced map
	\begin{equation*}
		F \to \hofib_x(f),\ z \mapsto (h(z), i(z))
	\end{equation*}
	is a weak homotopy equivalence.
\end{definition}
\begin{example}\label{expl:fibresequences}
	\leavevmode
	\begin{enumerate}
		\item Let $f\colon Y \to X$ be any continuous map, $x \in X$ a point.
			Then the pair $(\hofib_x(f) \xto{i} Y \xto{f} X, h)$ where $i(\gamma, y) \coloneq y$ is a fibre sequence since by construction the map $\hofib_x(f) \to \hofib_x(f)$ is just the identity.
	\end{enumerate}
\end{example}
Every fibre sequence is equivalent to such an example in the following sense:
Given $(F \xto{i} Y \xto{f} X)$, there is a commutative diagram
\begin{equation*}
	\begin{tikzcd}
		F
				\ar[r, "\htpyeqv_w"]
				\ar[d]
			& \hofib_x(f)
				\ar[d]
		\\
		Y
				\ar[r, equal, "\id_Y"]
				\ar[d, "f"]
			& Y
				\ar[d, "f"]
		\\
		X 
				\ar[r, equal, "\id_X"]
			& X
	\end{tikzcd}
\end{equation*}
and \enquote{equivalence of fibre sequences}.
In particular, $\Omega X \to * \to X$ is a fibre sequence where $h\colon \Omega X \times I \to X$ is the evaluation map.

\strong{Warning:} If one instead chooses $h$ to be the constant homotopy, one does not obtain a fibre sequence (unless $X$ is weakly contractible) just because the induced map $\Omega X \to \hofib_x(f) = \Omega X$ is the constant map which is not a weak homotopy equivalence.
Hence, the choice of $h$ is important!

% TODO starred
\begin{example}[continuation of example \ref{expl:fibresequences}]
	\leavevmode
	\begin{enumerate}[resume]
		\item For every two spaces $F$ and $X$ and all basepoints $x \in X$, the pair $(F \to F \times X \xto{\pr_X} X, \const)$ is a fibre sequence called the \strong{trivial fibre sequence}\index{trivial fibre sequence}.
			To see this, note that
			\begin{equation*}
				\hofib_x(\pr_X) = F \times P_x X
			\end{equation*}
			with induced map $F \to F \times P_x X,\ y \mapsto (y, \const_x)$ which is a homotopy equivalence as $P_x X$ is contractible.
		\item\label{expl:fibrebundlefibresequence} Let $p\colon E \to B$ be a fibre bundle with fibre $F = p^{-1}(b)$ for some $b \in B$.
			Then the sequence $F \to E \xto{p} B$ together with the constant homotopy is a fibre sequence.
			This is a special case of the next example:
		\item Recall that $p\colon E \to B$ is a \strong{Serre fibration}\index{fibration!Serre} if in every commutative diagram of the form
			\begin{equation*}
				\begin{tikzcd}
					D^n \times \{0\}
							\ar[r]
							\ar[d, hook]
						& E
							\ar[d, "p"]
					\\
					D^n \times I
							\ar[r]
							\ar[ur, dashed]
						& B
				\end{tikzcd}
			\end{equation*}
			a lift $D^n \times I \to E$ exists making the whole diagram commute.
			Given a Serre fibration $p\colon E \to B$ and a point $b \in B$, the sequence $F = p^{-1}(b) \incl E \to B$ together with the constant homotopy is a fibre sequence (the proof of this is exercise \ref{ex:serrefib}).
		\item\label{expl:hopfbundle} As a special case of \ref{expl:fibrebundlefibresequence}, the \strong{Hopf fibration}\index{Hopf fibration} is a fibre bundle
			\begin{equation*}
				S^1 \to S^3 \xto{\eta} S^2
			\end{equation*}
			It arises by letting $S^1 = U(1)$ act on $S^3 \subseteq \C^2$ via
			\begin{equation*}
				\lambda \cdot (x_1, x_2) = (\lambda x_1, \lambda x_2)
			\end{equation*}
			The quotient space of this action is $\CP^1 = S^2$.
		\item The previous example generalizes to fibre bundles
			\begin{equation*}
				S^1 \to S^{2n + 1} \to \CP^n
			\end{equation*}
			with limit case
			\begin{equation*}
				\begin{tikzcd}
					S^1
							\ar[r]
							\ar[d, "\htpyeqv"]
						& S^\infty
							\ar[r]
							\ar[d, "\htpyeqv"]
						& \CP^\infty
							\ar[d, equal]
					\\
					\Omega \CP^\infty
							\ar[r]
						& * 
							\ar[r]
						& \CP^\infty
				\end{tikzcd}
			\end{equation*}
	\end{enumerate}
\end{example}

\section{Exercises}
\begin{exercise}\label{ex:serrefib}
	The goal of the first problem is to recall the notion of a Serre fibration and its homotopical properties.
	\begin{itemize}
		\item A map of spaces $p\colon E \to B$ has the \emph{homotopy lifting property with respect to a space $X$} if for every commutative diagram of the form
			\begin{equation*}
				\begin{tikzcd}
					X \times \{0\} 
							\ar[r, "\tilde{f}_0"]
							\ar[d, hook]
						& E
							\ar[d, "p"]
					\\
					X \times I
							\ar[r, "f"]
						& B
				\end{tikzcd}
			\end{equation*}
			there exists a map $\tilde{f}\colon X \times I \to E$ making the diagram commute.
		\item A map $p\colon E \to B$ has the \emph{homotopy lifting property with respect to a pair of spaces $(X, A)$} if for every commutative diagram of the form
			\begin{equation*}
				\begin{tikzcd}
					X \cup_A (A \times I)
							\ar[r]
							\ar[d, hook]
						& E
							\ar[d, "p"]
					\\
					X \times I
							\ar[r, "f"]
						& B
				\end{tikzcd}
			\end{equation*}
			there exists a map $\tilde{f}\colon X \times I \to E$ making the diagram commute. (The space $X \cup_A (A \times I)$ is defined by gluing $A \times I$ to $X$ along the natural map $A \times \{0\} \to X$.)
		\item A map of spaces $p\colon E \to B$ is said to be a \emph{Serre fibration} if it has the homotopy lifting property with respect to all discs $D^n$, $n \geq 0$.
			It can be shown that having the homotopy lifting property with respect to all discs is equivalent to having the homotopy lifting property with respect to all CW-pairs.
	\end{itemize}
	Furthermore, we recall the notion of a \emph{homotopy fibre}.
	For a space $X$ and $x \in X$ we let $P_x X$ denote the space of paths in $X$ to $x$, that is $P_x X \coloneq \{\gamma\colon I \to X \mid \gamma(1) = x\}$, equipped with the compact-open topology.
	Given a map $f\colon Y \to X$, the \emph{homotopy fibre} of $f$ at $x$ is then defined as the space
	\begin{equation*}
		\hofib_x(f) \coloneq P_x X \times_X Y = \{(y, \gamma) \in P_x \times Y \mid \gamma(0) = f(y)\}
	\end{equation*}
	Now let $p\colon E \to B$ be a Serre fibration and $b \in B$ a basepoint.
	We write $F = p^{-1}(b) \subseteq E$ for the fibre and define a map $\varphi\colon F \to \hofib_b(p)$ by the formula  
	\begin{equation*}
		z \mapsto (c(b), i(z))
	\end{equation*}
	Here, $c(b)$ denotes the constant path in $B$ at the basepoint $b$.

	Prove that $\varphi$ is a weak homotopy equivalence, i.e. that it induces an isomorphism on homotopy groups for all basepoints.
	\begin{hint}
		If you have trouble with the proof, first focus on showing that $\varphi$ induces a bijection on path components.
	\end{hint}
\end{exercise}

\begin{exercise}
	Let $C$ be a chain complex filtered by subcomplexes $C_0 \subseteq C_1 \subseteq C$.
	The pairs $(C_1, C_0)$ and $(C, C_1)$ have associated long exact sequences of homology groups
	\begin{equation}\label{ex2:les1}
		\cdots \to H_n(C_0) \to H_n(C_1) \xto{q_*} H_n(C_1 / C_0) \xto{\del^{(C_1, C_0)}} H_{n - 1}(C_0) \to \cdots
	\end{equation}
	and 
	\begin{equation}\label{ex2:les2}
		\cdots \to H_n(C_1) \to H_n(C) \to H_n(C / C_1) \xto{\del^{(C, C_1)}} H_{n - 1}(C_1) \to \cdots
	\end{equation}
	respectively.
	The goal of this exercise is to compute the homology of $C$ in terms of the homology of the complexes $C_0$, $C_1 / C_0$, and $C / C_1$.
	This is the length 2 special case of the spectral sequence associated to a filtered complex which we will later discuss in the lecture.
	\begin{enumerate}
		\item Use the long exact sequences above to define maps $f\colon H_*(C / C_1) \to H_{* - 1}(C_1 / C_0)$ and $g\colon H_*(C_1 / C_0) \to H_{* - 1}(C_0)$.
			Show that $g \circ f = 0$.
			In spectral sequence terminology, the maps $g$ and $f$ are the only potentially non-zero $d^1$-differentials.
		\item Next we will construct the only potentially nonzero $d^2$-differential.
			Use the long exact sequences above once more to construct another map $d\colon \ker(f) \to \coker(g)$ of degree $-1$ so that there are isomorphisms
			\begin{itemize}
				\item $\coker(d) \isom \img(H_*(C_0) \to H_*(C))$
				\item $\ker(g) / \img(f) \isom \img(H_*(C_1) \to H_*(C)) / \img(H_*(C_0) \to H_*(C))$
				\item $\ker(d) \isom H_*(C) / \img(H_*(C_1) \to H_*(C))$
			\end{itemize}
			where $\img({{-}})$ is the image of a map.
			In words, $\coker(d)$, $\ker(g) / \img(f)$, and $\ker(d)$ are isomorphic to the subquotients in the filtration on $H_*(C)$ given by the filtration on $H_*(C)$ given by the images of $H_*(C_0)$ and $H_*(C_1)$.
	\end{enumerate}
\end{exercise}

\printbibliography
\printindex
\end{document}
