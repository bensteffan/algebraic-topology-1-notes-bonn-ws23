\documentclass[wip, topology]{bsteffan-lecturenotes}

\addbibresource{references.bib}

\colorlet{knowngray}{darkgray!70}

\course{Algebraic Topology \uppercase\expandafter{\romannumeral 1\relax}}
\lecturer{Prof. Dr. Markus Hausmann}
\assistant{Dr. Elizabeth Tatum}
\author{Ben Steffan}

\begin{document}
\maketitle
\tableofcontents
\listoflectures

\setcounter{section}{-1}
\section*{About These Notes}
This document encompasses lecture notes for the course \makeatletter\@course\makeatother\ taught in the winter term of 2023/24 at the University of Bonn by \makeatletter\@lecturer\makeatother.
The assistant is \makeatletter\@assistant\makeatother.

These notes are for private use. 
They are \strong{not} official lecture notes endorsed by the lecturer.
As such, errors and inaccuracies that persist are generally my own (unless proven otherwise).

This document is not a character-for-character transcript of the lecture.
Changes to form (though generally not to content) have been made to improve readability of these notes as a document.
In particular, I have taken the liberty to make adjustments to notation here and there to more closely align with my personal tastes and opinions.
At points, I have added additional context, explanations, computations, and so on.
These are clearly marked to that effect, although smaller changes and in-text additions (such as citations) are not.

\subsection*{Formatting}
This document has hyperlinks: References, footnote marks, table-of-contents entries and so on are linked and can be clicked to take you to the corresponding item.
Except for footnote marks, which remain black, all such links are highlighted in either \textcolor{linkcol}{orange} or \textcolor{citecol}{violet}. 
\textcolor{highlightcol}{Red} is used to highlight certain items formulas and diagrams.
The colors \textcolor{definitioncol}{green}, \textcolor{exercisecol}{blue}, and \textcolor{theoremcol}{red} are used as border colors to higlight definitions, exercises, and theorems, propositions, lemmas, etc., respectively.
\textcolor{knowngray}{Gray} is used to convey known or secondary information in formulas and diagrams from place to place.

Demarcations for lecture dates are placed in the righthand margin.

\section{Informal Introduction}\lecture{09.10.23}
\begin{note}
	We omit the coefficient group $\Z$ from notation in (co)homology as well as basepoints where they are of no particular relevance in homotopy groups.
\end{note}
One of the big goals of homotopy theory is to compute
\[
	[X, Y]_* = \{\text{basepoint-preserving continuous maps } X \to Y\} / \text{homotopy}
\]
for $X$ and $Y$ pointed CW-complexes.
CW-complexes are built out of spheres, so the building blocks are the sets
\[
	[S^n, S^k]_* = \pi_n(S^k, *)
\]
For $n \geq 1$ these are groups and abelian if $n \geq 2$.
We know that\textellipsis{}
\begin{itemize}
	\item $\pi_n(S^k) = 0$ for $n < k$ by cellular approximation, cf. \cite[Corollary 4.9]{hatcher_algebraic_2002},
	\item $\pi_n(S^n) \isom \Z$ by the Hurewicz theorem (cf. \cite[Theorem 4.32]{hatcher_algebraic_2002}) and $H_n(S^n) \isom \Z$: 
		If $X$ is an $(n - 1)$-connected CW-complex ($n > 1$), then there is an isomorphism $\pi_n(X) \xto{\isom} H_n(X)$, 
	\item $\pi_k(S^1) = 0$ for $k \geq 2$ via covering space theory: The universal cover of $S^1$ is $\R$ which is contractible, 
	\item $\pi_3(S^2) \neq 0$ since the attaching map of the 4-cell for $\CP^2$ is a map $\eta\colon S^3 \to S^2 = \CP^1$; if $\eta$ was nullhomotopic, then $\CP^2$ would be homotopy equivalent to $S^2 \vee S^4$ which contradicts the ring structure on $H^*(\CP^2) \isom \Z[x] / x^3$, and that
	\item The sequence
		\begin{equation*}
			\pi_k(S^n) \to \pi_{k + 1}(S^{n + 1}) \to \pi_{k + 2}(S^{n + 2}) \to \cdots
		\end{equation*}
		always stabilizes by the \emph{Freudenthal suspension theorem} (see \cite[Corollary 4.24]{hatcher_algebraic_2002}).
\end{itemize}
To go beyond this, we will need a new tool, the \emph{Serre spectral sequence}.
To motivate its usefulness for this question, consider the following strategy:
There exists a map $f\colon S^2 \to K(\Z, 2)$ which induces an isomorphism $f_*\colon \pi_2(S^2) \xto{\isom} \pi_2(K(\Z, 2)) \isom \Z$.
We can take its homotopy fibre $H \coloneq \hofib(f_*)$; there is then a fibre sequence $H \to S^2 \xto{f} K(\Z, 2)$ and thus a long exact sequence\footnote{The groups in \textcolor{knowngray}{gray} are assumed known. The groups and properties in \textcolor{col05}{red} follow from those in gray.}
\begin{equation*}
	\begin{tikzpicture}[commutative diagrams/.cd, every diagram, row sep = large]
		\matrix[matrix of math nodes, name = m, commutative diagrams/every cell] {
				& \cdots & \pi_4(K(\Z, 2)) \\
			\pi_3(H) & \pi_3(S^2) & \pi_3(K(\Z, 2)) \\
			\pi_2(H) & \pi_2(S^2) & \pi_2(K(\Z, 2)) \\
			\pi_1(H) & \pi_1(S^2) & \pi_1(K(\Z, 2)) & 0 \\
		};
		\path[commutative diagrams/.cd, every arrow, every label]
			(m-1-2) edge (m-1-3)
			(m-1-3) edge[rounded corners, to path = {
				-- ([xshift = 1em] \tikztostart.east)
				|- ($(m-1-2)!0.5!(m-2-2)$) \tikztonodes
				-| ([xshift = -2ex] \tikztotarget.west)
				-- (\tikztotarget)
			}] (m-2-1)
			(m-2-1) edge["{\color{highlightcol}\isom}"] (m-2-2) 
			(m-2-2) edge (m-2-3)
			(m-2-3) edge[rounded corners, to path = {
				-- ([xshift = 1em] \tikztostart.east)
				|- ($(m-2-1)!0.5!(m-3-1)$) \tikztonodes
				-| ([xshift = -2ex] \tikztotarget.west)
				-- (\tikztotarget)
			}] (m-3-1)
			(m-3-1) edge (m-3-2) 
			(m-3-2) edge["$f_*$", "$\isom$"'] (m-3-3)
			(m-3-3) edge[rounded corners, to path = {
				-- ([xshift = 1em] \tikztostart.east)
				|- ($(m-3-1)!0.5!(m-4-1)$) \tikztonodes
				-| ([xshift = -2ex] \tikztotarget.west)
				-- (\tikztotarget)
			}] (m-4-1)
			(m-4-1) edge (m-4-2) 
			(m-4-2) edge (m-4-3)
			(m-4-3) edge (m-4-4);
		% don't want our labeling nodes to be the usual node grid distance away
		% better to set this via scope than to either pass it to each \node separately or set it for the whole picture
		\begin{scope}[node distance = -5.25pt, knowngray] 
			\node[below = of m-1-3] {$0$};
			\node[below = of m-2-3] {$0$};
			\node[below = of m-3-1, highlightcol] {$0$};
			\node[below = of m-4-1, highlightcol] {$0$};
			\node[below = of m-4-2] {$0$};
			\node[below = of m-4-3] {$0$};
		\end{scope}
	\end{tikzpicture}
\end{equation*}
Hence, $H$ is 2-connected and $\pi_n(H) \to \pi_n(S^2)$ is an isomorphism for all $n \geq 3$.
By the Hurewicz theorem, the following diagram commutes:
\begin{equation*}
	\begin{tikzcd}
		\pi_3(H)
				\ar[r, "\isom"]
				\ar[dr, swap, "\isom"]
			& H_3(H)
		\\
			& \pi_3(S^2)
				\ar[u, swap, "\isom"]
	\end{tikzcd}
\end{equation*}
If we had a way to compute $H_*(H)$ from $H_*(S^2)$ (the computation of which is easy) and $H_*(K(\Z, 2))$ (which is known), we could compute $\pi_3(S^2)$ this way!

\subparagraph{Upshot}
It would be useful to have a tool which relates the homology groups of the three terms in a fibre sequence.
This will also help us to compute $\pi_n(S^k)$ in other ways (for example, we will show that $\pi_n(S^k)$ is finite unless $n = k$ or $n = 2k - 1$ and $k$ is even).
Furthermore, the Serre spectral sequence will allow us to compute the (co)homology of spaces like $\Uni(n)$, $\SU(n)$, $\Omega S^n$, $K(\Zn{2}, n)$ and (re)prove structural theorems like the Hurewicz theorem, the Freudenthal suspension theorem, Thom isomorphisms, and more.

So, given a fibre sequence $F \to Y \to X$, what could the relationship between the homology groups of $F$, $Y$ and $X$ be?
\begin{example}
	Consider the easiest case $F \to X \times F \xto{\pr_X} X$ (a \emph{trivial fibration}).
	Then the Alexander-Whitney map induces an isomorphism $H_n(X \times F; \Z) \xto{\isom} \bigdsum_{p + q = n} H_p(X, H_q(F))$, so it computes the homology of the total space in terms of the homology of $X$ and $F$.
\end{example}
\begin{example}
	Consider the Hopf fibration $S^1 \to S^3 \xto{\eta} S^2$.
	We can compute
	\begin{equation*}
		\renewcommand{\arraystretch}{1.1}
		\begin{tabular}{r|c|c}
			$n$ 		& $H_n(S^3; \Z)$ 	& $\bigdsum_{p + q = n} H_p(S^3; H_q(S^1; \Z))$ \\\hline
			0 			& $\Z$ 				& $\Z$ \\
			1 			& $0$				& $\Z$ \\
			2 			& $0$				& $\Z$ \\
			3 			& $\Z$ 				& $\Z$ \\
			4 			& $0$				& $0$ \\
			$\vdots$ 	& $\vdots$			& $\vdots$
		\end{tabular}
	\end{equation*}
	so $\bigdsum_{p + q = n} H_p(S^3; H_q(S^1; \Z))$ is in some sense \enquote{too big} to describe $H_n(S^3; \Z)$ in degrees $n = 1, 2$.
	Note, however, that we can consider a \enquote{2-step filtration} $S^1 \subseteq S^3$ which satisfies $\tilde{H}_n(S^3 / S^1; \Z) \isom \Z$ if $n = 2, 3$ and $0$ else.
	Then
	\begin{equation*}
		\renewcommand{\arraystretch}{1.1}
		\begin{tabular}{r|c}
			$n$ & $H_n(S^1; \Z) \dsum \tilde{H}_n(S^3 / S^1; \Z)$ \\\hline
			0 	& $\Z$ \\
			1 	& $\Z$ \\
			2 	& $\Z$ \\
			3 	& $\Z$ \\
			4 	& 0 \\
			$\vdots$ 	& $\vdots$ \\
		\end{tabular}
	\end{equation*}
	This does not agree with $H_3(S^3; \Z)$ because in the long exact sequence
	\begin{equation*}
		\begin{tikzcd}[row sep = large]
			\cdots 
					\ar[r]
				& H_3(S^3; \Z)
					\ar[r]
				& \tilde{H}_3(S^3 / S^1; \Z)
					\ar[dll, rounded corners, swap, "\del", to path = {[pos = 1]
						-- ([xshift = 1em] \tikztostart.east)
						|- ($(\tikzcdmatrixname-1-2)!0.5!(\tikzcdmatrixname-2-2)$) \tikztonodes
						-| ([xshift = -2ex] \tikztotarget.west)
						-- (\tikztotarget)
					}]
			\\
			H_2(S^1; \Z)
					\ar[r]
				& H_2(S^3; \Z)
					\ar[r]
				& \tilde{H}_2(S^3 / S^1; \Z)
					\ar[dll, rounded corners, swap, "\del", "{\color{highlightcol}\isom}"', to path = {[pos = 1]
						-- ([xshift = 1em] \tikztostart.east)
						|- ($(\tikzcdmatrixname-2-2)!0.5!(\tikzcdmatrixname-3-2)$) \tikztonodes
						-| ([xshift = -2ex] \tikztotarget.west)
						-- (\tikztotarget)
					}]
			\\
			H_1(S^1; \Z)
					\ar[r]
				& H_1(S^3; \Z)
					\ar[r]
				& \cdots
		\end{tikzcd}
	\end{equation*}
	the boundary map $\tilde{H}_2(S^3 / S^1; \Z) \to H_1(S^1; \Z)$ is an isomorphism.
	Hence, $H_1(S^1; \Z)$ does not contribute to $H_1(S^3; \Z)$ and $\tilde{H}_2(S^3 / S^1; \Z)$ does not contribute to $H_2(S^3; \Z)$.
\end{example}
It turns out that something similar holds for all fibre sequences $F \to Y \to X$: 
There exists a filtration on $C_*(Y; \Z)$
\begin{equation*}
	F_0 \subseteq F_1 \subseteq \cdots \subseteq F_m \subseteq \cdots \subseteq C_*(Y; \Z)
\end{equation*}
of chain complexes such that $H_{p + q}(F_p / F_{p - 1}) \isom C^{\text{cell}}_p(X; H_q(F; \Z))$.
To understand $H_*(Y; \Z)$, one needs to understand the cancellations in the associated long exact sequences.
This is best encoded in a \emph{spectral sequence}.

\section{Spectral Sequences}
\begin{definition}
	A (\strong{homologically}\index{spectral sequence!homologically graded}/\strong{Serre graded}\index{spectral sequence!Serre graded}) \strong{spectral sequence}\index{spectral sequence} is a triple $(E^\bullet, d^\bullet, h^\bullet)$ where
	\begin{itemize}
		\item $(E^r)_{r \geq 2}$ is a sequence of $\Z$-bigraded abelian groups.
			We write $E^r_{p, q}$.
			$E^r$ is called the $r$th \strong{page}\index{page!of a spectral sequence} of the spectral sequence.
		\item $d^r\colon E^r \to E^r$ is a sequence of morphisms (called \strong{differentials}\index{differential!of a spectral sequence}) of bidegree $(-r, r - 1)$ satisfying $d^r \circ d^r = 0$.
		\item $h^r\colon H_*(E^r) \to E^{r + 1}$ is a sequence of bigrading-preserving isomorphisms.
			Here $H_*(E^r)$ denotes the homology of $E^r$ with respect to $d^r$, which inherits a bigrading.
	\end{itemize}
\end{definition}

\printbibliography
\printindex
\end{document}
