\section{Exercises}
This section contains all of the weekly exercises and solutions for the course.
The solutions presented here are my own, unless explicitly stated otherwise.
Most of them have been checked by a pair of skilled eyes but by no means all, so tread with appropriate care.

At the end of the section (page \pageref{sect:quickquest} and following) you will find both sets of quick questions together with their (justified, again by me) answers.

\begin{exercise}\label{ex:serrefib}
	The goal of the first problem is to recall the notion of a Serre fibration and its homotopical properties.
	\begin{itemize}
		\item A map of spaces $p\colon E \to B$ has the \emph{homotopy lifting property with respect to a space $X$} if for every commutative diagram of the form
			\begin{equation*}
				\begin{tikzcd}
					X \times \{0\} 
							\ar[r, "\tilde{f}_0"]
							\ar[d, hook]
						& E
							\ar[d, "p"]
					\\
					X \times I
							\ar[r, "f"]
						& B
				\end{tikzcd}
			\end{equation*}
			there exists a map $\tilde{f}\colon X \times I \to E$ making the diagram commute.
		\item A map $p\colon E \to B$ has the \emph{homotopy lifting property with respect to a pair of spaces $(X, A)$} if for every commutative diagram of the form
			\begin{equation*}
				\begin{tikzcd}
					X \cup_A (A \times I)
							\ar[r]
							\ar[d, hook]
						& E
							\ar[d, "p"]
					\\
					X \times I
							\ar[r, "f"]
						& B
				\end{tikzcd}
			\end{equation*}
			there exists a map $\tilde{f}\colon X \times I \to E$ making the diagram commute. (The space $X \cup_A (A \times I)$ is defined by gluing $A \times I$ to $X$ along the natural map $A \times \{0\} \to X$.)
		\item A map of spaces $p\colon E \to B$ is said to be a \emph{Serre fibration} if it has the homotopy lifting property with respect to all discs $D^n$, $n \geq 0$.
			It can be shown that having the homotopy lifting property with respect to all discs is equivalent to having the homotopy lifting property with respect to all CW-pairs.
	\end{itemize}
	Furthermore, we recall the notion of a \emph{homotopy fibre}.
	For a space $X$ and $x \in X$ we let $P_x X$ denote the space of paths in $X$ to $x$, that is $P_x X \coloneq \{\gamma\colon I \to X \mid \gamma(1) = x\}$, equipped with the compact-open topology.
	Given a map $f\colon Y \to X$, the \emph{homotopy fibre} of $f$ at $x$ is then defined as the space
	\begin{equation*}
		\hofib_x(f) \coloneq P_x X \times_X Y = \{(y, \gamma) \in P_x \times Y \mid \gamma(0) = f(y)\}
	\end{equation*}
	Now let $p\colon E \to B$ be a Serre fibration and $b \in B$ a basepoint.
	We write $F = p^{-1}(b) \subseteq E$ for the fibre and define a map $\varphi\colon F \to \hofib_b(p)$ by the formula  
	\begin{equation*}
		z \mapsto (c(b), i(z))
	\end{equation*}
	Here, $c(b)$ denotes the constant path in $B$ at the basepoint $b$.

	Prove that $\varphi$ is a weak homotopy equivalence, i.e. that it induces an isomorphism on homotopy groups for all basepoints.
	\begin{hint}
		If you have trouble with the proof, first focus on showing that $\varphi$ induces a bijection on path components.
	\end{hint}
\end{exercise}

\begin{exercise}
	Let $C$ be a chain complex filtered by subcomplexes $C_0 \subseteq C_1 \subseteq C$.
	The pairs $(C_1, C_0)$ and $(C, C_1)$ have associated long exact sequences of homology groups
	\begin{equation}\label{ex2:les1}
		\cdots \to H_n(C_0) \to H_n(C_1) \xto{q_*} H_n(C_1 / C_0) \xto{\del^{(C_1, C_0)}} H_{n - 1}(C_0) \to \cdots
	\end{equation}
	and 
	\begin{equation}\label{ex2:les2}
		\cdots \to H_n(C_1) \to H_n(C) \to H_n(C / C_1) \xto{\del^{(C, C_1)}} H_{n - 1}(C_1) \to \cdots
	\end{equation}
	respectively.
	The goal of this exercise is to compute the homology of $C$ in terms of the homology of the complexes $C_0$, $C_1 / C_0$, and $C / C_1$.
	This is the length 2 special case of the spectral sequence associated to a filtered complex which we will later discuss in the lecture.
	\begin{enumerate}
		\item Use the long exact sequences above to define maps $f\colon H_*(C / C_1) \to H_{* - 1}(C_1 / C_0)$ and $g\colon H_*(C_1 / C_0) \to H_{* - 1}(C_0)$.
			Show that $g \circ f = 0$.
			In spectral sequence terminology, the maps $g$ and $f$ are the only potentially non-zero $d^1$-differentials.
		\item Next we will construct the only potentially nonzero $d^2$-differential.
			Use the long exact sequences above once more to construct another map $d\colon \ker(f) \to \coker(g)$ of degree $-1$ so that there are isomorphisms
			\begin{itemize}
				\item $\coker(d) \isom \img(H_*(C_0) \to H_*(C))$
				\item $\ker(g) / \img(f) \isom \img(H_*(C_1) \to H_*(C)) / \img(H_*(C_0) \to H_*(C))$
				\item $\ker(d) \isom H_*(C) / \img(H_*(C_1) \to H_*(C))$
			\end{itemize}
			where $\img({{-}})$ is the image of a map.
			In words, $\coker(d)$, $\ker(g) / \img(f)$, and $\ker(d)$ are isomorphic to the subquotients in the filtration on $H_*(C)$ given by the filtration on $H_*(C)$ given by the images of $H_*(C_0)$ and $H_*(C_1)$.
	\end{enumerate}
\end{exercise}
\begin{figure}[hb]
	\begin{equation*}
		\begin{tikzcd}[column sep = small]
			\cdots
					\ar[r, gray]
				& H_n(C_0)
					\ar[d, gray, "i^0_*"]
			\\
			\cdots
					\ar[r]
				& H_n(C_1)
					\ar[r, gray, "q^0_*"]
					\ar[d, "i^1_*"]
				& H_n(C_1 / C_0) 
					\ar[r, gray, "\del^0"]
					\ar[r, citecol, shift right = 1.5, swap, "g"]
				& H_{n - 1}(C_0)
					\ar[d, gray, "i^0_*"]
			\\
				& H_n(C)
					\ar[r, "q^1_*"]
				& H_n(C / C_1)
					\ar[r, "\del^1"]
					\ar[r, shift right = 1.5, highlightcol, dash]
				& H_{n - 1}(C_1)
					\ar[r, gray, "q_*^0"]
					\ar[r, shift right = 1.5, highlightcol, swap, "f" near start]
					\ar[d, "i_1^*"]
				& H_{n - 1}(C_1 / C_0)
					\ar[r, gray, "\del^0"]
				& \cdots
			\\
				& & & H_{n - 1}(C)
					\ar[r, "q^1_*"]
				& H_{n - 1}(C / C_1)
					\ar[r, "\del^1"]
				& \cdots
		\end{tikzcd}
	\end{equation*}
	\caption{%
		Sequences \ref{ex2:les1} and \ref{ex2:les2} arranged in a staircase diagram. 
		The morphisms belonging to sequence \ref{ex2:les1} are set in \textcolor{gray}{gray}.
		The two maps $f$ and $g$ constructed in part two of the exercise are highlight in \textcolor{highlightcol}{red} and \textcolor{citecol}{violet}, respectively.
	}
	\label{fig:staircase}
\end{figure}
\begin{solution}
	We will use the morphism names indicated in figure \ref{fig:staircase}.
	That figure will also be helpful for following along.
	\begin{enumerate}
		\item We put $f \coloneq q^0_* \circ \del^1$ and $g \coloneq \del^0$.
			Then $g \circ f = \del^0 \circ q^0_* \circ \del^1 = 0$ since $\del^0 \circ q^0_* = 0$ by exactness of sequence \ref{ex2:les1}.
		\item Note that the condition $g \circ f = 0$ amounts to saying that
			\begin{equation*}
				H_n(C / C_1) \xto{f} H_{n - 1}(C_1 / C_0) \xto{g} H_{n - 2}(C_0)
			\end{equation*}
			is a chain complex for all $n$; denoting its homology as $\bar{H}^0_n = \ker f$, $\bar{H}^1_{n - 1} = \ker g / \img f$, and $\bar{H}^2_{n - 2} = \coker g$, our goal is to find a map $d\colon \bar{H}^0_* \to \bar{H}^2_{* - 1}$ with the required properties.
			To this end, note that there is an isomorphism $\sigma\colon H_n(C_1) \supseteq \img i^0_* \isom \bar{H}^2_n$ since by exactness of sequence \ref{ex2:les1}, $\img g = \img \del^0 = \ker i^0_*$.
			We now put $d \coloneq \sigma \circ \del^1$.
			First of all, note that this composition is well-defined since if $\alpha \in \ker f$, then $q^0_*(\del^1(\alpha)) = 0$ implies that $\del^1(\alpha) \in \ker q^0_* = \img i^0_*$, so $\del^1(\alpha)$ is in the domain of $\sigma$.

			Now for the properties:
			\begin{itemize}
				\item Since $\img g = \ker i^0_*$ and $\img \del^1 = \ker i^1_*$ by exactness, we have $\coker d \isom (H_*(C_0) / \ker i^0_*) / \sigma(\ker i^1_*) \isom \img\bigg(H_*(C_0) \xto{i^1_* \circ i^0_*} H_*(C)\bigg)$ since $\sigma$ is the inverse to the quotient isomorphism $H_*(C_0) / \ker i^0_* \xto{\isom} \img i^0_*$. 
				\item For this item we will do a diagram chase.
					Our goal is to define a surjective map $\varphi\colon \ker g \to \img i^1_* / \img (i^1_* \circ i^0_*)$ whose kernel is $\img f$.
					In fact, let us put $\varphi(x) \coloneq [y]$ where $y \in \img i^1_* \subseteq H_n(C)$ is the image $i^1_*(y')$ of an element $y' \in (q^0_*)^{-1}(x) \subseteq H_n(C_1)$.
					This is well-defined: 
					First of all, we note that $\ker g = \ker \del^0 = \img q^0_*$ so that we can take preimages under $q^0_*$.
					Next, if $y'' \in (q^0_*)^{-1}(x)$ is another lift, we note that $q^0_*(y' - y'') = 0$, thus $y' - y'' \in \img i^0_*$ by exactness. 
					Therefore, $i^1_*(y' - y'') \in \img (i_1^* \circ i_0^*)$ which is to say that $[i_1^*(y')] = [i_1^*(y'')]$ in the quotient, i.e. the choice of lift $y'$ does not matter.

					For surjectivity, we note again that $\ker q^0_* = \img i^0_*$, which implies that $\varphi$ is surjective since the map $H_*(C_1) / \img i^0_* \to \img i^1_* / \img (i^1_* \circ i^0_*)$ induced by $i^1$ is and $\bar{q}^0_*\colon H_*(C_1) / \img i^0_* \to \img q^0_*$ is an isomorphism.

					Finally, we have on one hand that $\img f \subseteq \ker \varphi$ as $\img f = \img q^0_*|_{\img \del^1} = \img q^0_*|_{\ker i^1_*}$, i.e. $(q^0_*)^{-1}(z) \in \ker i^1_*$ for all $z \in \img f$ and therefore $\varphi(z) = 0$, while on the other $\ker \varphi \subseteq \img f$ as well:
					Saying that $\varphi(z) = 0$ for some $z \in \ker g$ is equivalent to saying that $i^1_*(z') \in \img (i^1_* \circ i^0_*)$ for any lift $z' \in (q^0_*)^{-1}(z)$ by definition.
					But $\ker q^0_* = \img i^0_*$, so fixing $0 \in H_*(C_1)$ as the lift of $0 \in \ker g$ without loss of generality, we may assume that $z' \in \ker i^1_* = \img \del^1$, in other words that there exists a preimage $z'' \in (\del^1)^{-1}(z)$ such that $f(z'') = z$, i.e. $z \in \img f$.
				\item Since $\sigma$ is an isomorphism, we see that 
					\begin{equation*}
						\ker d = \ker \del^1 = \img q^1_* \isom H_*(C) / \ker q^1_* = H_*(C) / \img i^1_* 
					\end{equation*}
					using the exactness of sequence \ref{ex2:les2}.
			\end{itemize}
			This concludes the proof.
			\qedhere
	\end{enumerate}
\end{solution}

\begin{exercise}\label{ex:gysinwang}
	\leavevmode
	\begin{enumerate}
		\item Suppose that $S^n \to E \to B$ is a fibre sequence, $n \neq 0$, $B$ simply connected.
			Show that there is a long exact sequence of the form
			\begin{equation*}
				\cdots \to H_{p - n}(B) \to H_p(E) \to H_p(B) \to H_{p - n - 1}(B) \to H_{p - 1}(E) \to \cdots
			\end{equation*}
			This sequence is called the \strong{Gysin sequence}\index{Gysin sequence} of the sphere bundle.
		\item Let $F \to E \to S^n$ be a fibre sequence over a sphere with $n \neq 0, 1$.
			Show that there exists a long exact sequence of the form
			\begin{equation*}
				\cdots \to H_q(F) \to H_q(E) \to H_{q - n}(F) \to H_{q - 1}(F) \to H_{q - 1}(E) \to \cdots
			\end{equation*}
			This sequence is called the \strong{Wang sequence}\index{Wang sequence}.
	\end{enumerate}
\end{exercise}
\begin{solution}
	\leavevmode
	\begin{enumerate}
		\item Consider the homological Serre spectral sequence for $S^n \to E \to B$ (which we can do since $S^n$ is connected and $B$ simply connected).
			On the $E^2$-page, we have that
			\begin{equation}\label{ex:gysinwang:gysingroups}
				E^2_{p, q} = H_p(B; H_q(S^n)) \isom \begin{cases}
					H_p(B) 	& q = 0, n \\
					0 		& \text{else}
				\end{cases}
			\end{equation}
			since $H_q(S^n) \isom \Z$ if $q = 0, n$ and 0 else.
			Therefore, the only possible non-trivial differentials are $d^{n + 1}\colon E^{n + 1}_{p, 0} \to E^{n + 1}_{p - n - 1, n}$ for any $p$.
			Turning to the $E^\infty$-page, we see that every antidiagonal passes through at most two non-trivial groups, namely 
			\begin{equation*}
				E^\infty_{p, n} \isom \coker\big(d^{n + 1}\colon E^{n + 1}_{p + n + 1, 0} \to E^{n + 1}_{p, n}\big)
			\end{equation*}
			and 
			\begin{equation*}
				E^\infty_{p - n, 0} \isom \ker\big(d^{n + 1}\colon E^{n + 1}_{p - n, 0} \to E^{n + 1}_{p - 2n - 1, n}\big)
			\end{equation*}
			Thus, we have short exact sequences
			\begin{equation}\label{ex:gysinwang:inftyseq}
				0 \to E^\infty_{p, n} \xto{\iota} H_p(E) \xto{\pi} E^\infty_{p - n, 0} \to 0
			\end{equation}
			for all $p$.
			Putting all of this together, we obtain a long exact sequence
			\begin{equation*}
				\begin{tikzcd}[column sep = small]
					\cdots 
							\ar[r]
						& E^{n + 1}_{p, 0}
							\ar[r, "d^{n + 1}"]
						&[1em] E^{n + 1}_{p - n - 1, n}
							\ar[r, "\iota \circ \coker(d^{n + 1})"]
						&[4em] H_{p - 1}(E)
							\ar[r, "\kappa \circ \pi"]
						&[1em] E^{n + 1}_{p - 1, 0} 
							\ar[r]
						& \cdots
				\end{tikzcd}
			\end{equation*}
			where $\kappa\colon E^\infty_{p - 1, 0} \incl E^{n + 1}_{p - 1, 0}$ is the kernel inclusion of the outgoing differential (this is exact since $\coker(d^{n + 1})$ is surjective and $\kappa$ is injective, by which exactness at and around $H_{p - 1}(E)$ reduces to exactness of sequence \eqref{ex:gysinwang:inftyseq}).
			Finally substituting in the groups from equation \eqref{ex:gysinwang:gysingroups} for the $E^{n + 1}$-terms yields the claim.
		\item This part is almost entirely analogous to part one so we will allow ourselves to be somewhat brief.

			Note that the $E^2$-page of the associated homological Serre spectral sequence of the given fibre sequence has the form
			\begin{equation}\label{ex:gysinwang:wanggroups}
				E^2_{p, q} = H_p(S^n; H_q(F)) \isom \begin{cases}
					H_q(F)  & p = 0, n \\
					0 		& \text{else}
				\end{cases}
			\end{equation}
			so that all nontrivial differentials are of the form $d^n\colon E^n_{n, q} \to E^n_{0, q + n - 1}$ for $q$ any.
			On the $E^\infty$-page, we again have a maximum of two nonzero entries per antidiagonal leading to short exact sequences
			\begin{equation*}
				0 \to E^\infty_{0, q} \to H_q(E) \to E^\infty_{n, q - n} \to 0
			\end{equation*}
			for any $q$.
			In this situation we can identify $E^\infty_{0, q}$ and $E^\infty_{n, q - n}$ with the cokernel and kernel of the in- and outgoing differentials, respectively, like before so we obtain a long exact sequence
			\begin{equation*}
				\begin{tikzcd}
					\cdots 
							\ar[r]
						& E^n_{0, q}
							\ar[r]
						& H_q(E)
							\ar[r]
						& E^n_{n, q - n}
							\ar[r, "d^n"]
						& E^n_{0, q - 1}
							\ar[r]
						& \cdots
				\end{tikzcd}
			\end{equation*}
			which yields the Wang sequence after plugging in the groups from equation \eqref{ex:gysinwang:wanggroups}.
			\qedhere
	\end{enumerate}
\end{solution}

\begin{exercise}
	Let $X$ be a simply connected space which is not weakly contractible.
	Prove that it is not possible for both $X$ and $\Omega X$ to have the homotopy type of a finite CW-complex.
	\begin{hint}
		\leavevmode
		\begin{enumerate}
			\item Use the Hurewicz theorem to find a prime $p$ such that both $\tilde{H}_*(X; \F_p)$ and $\tilde{H}_*(\Omega X; \F_p)$ are nontrivial.
			\item Consider the fibre sequence $\Omega X \to * \to X$.
		\end{enumerate}
	\end{hint}
\end{exercise}
\begin{solution}
	Let us start out by proving the following algebraic lemma:
	\begin{lemma}
		Let $A \neq 0$ be a $\Z$-module.
		Then there exists some prime $p \in \Z \cup \{\infty\}$ such that $A \tensor_{\Z} \F_p \neq 0$.
		Here we define $\F_\infty \coloneq \Q$.
	\end{lemma}
	\begin{smallproof}
		As $A \tensor_{\Z} \Q \isom A / \tors(A)$ where $\tors(A) \coloneq \{a \in A \mid na = 0 \text{ for some } n \in \Z\}$ we can choose $p = \infty$ whenever $A$ is not entirely torsion.
		Otherwise, pick a prime $p$ such that $A$ has $p$-torsion.
		We then note that the map $A \xto{\cdot p} A$ is not surjective: if it were, we would have $A / \ker p \isom A$, but $A / \ker p$ does not have any $p$-torsion while $A$ does.
		Therefore, $A \tensor_{\Z} \F_p \isom A / p A$ is non-trivial.
	\end{smallproof}
	Let now $n > 1$ be minimal such that $\pi_n X \neq 0$ (such an $n$ exists since $X$ is not weakly contractible and simply connected).
	Since $\pi_k \Omega X \isom \pi_{k + 1} X$ for all $k$, this implies that $\pi_{n - 1} \Omega X \isom \pi_n X$ is the first non-trivial homotopy group of $\Omega X$ as well, so noting that in the case $n = 2$ the group $\pi_1 \Omega X \isom \pi_2 X$ is already abelian\footnote{so that $(\pi_1 \Omega X)^\text{ab} = \pi_1 \Omega X$ for the $n = 1$ special case of the Hurewicz theorem}, the Hurewicz theorem yields isomorphisms $H_n(X) \isom H_{n - 1}(\Omega X) \isom \pi_n X \neq 0$.

	An application of the universal coefficient theorem now tells us that
	\begin{equation*}
		H_n(X; \F_p) \isom (H_n(X; \Z) \tensor \F_p) \dsum \Tor(\underbrace{H_{n - 1}(X; \Z)}_{= 0}, \F_p) \isom H_n(X; \Z) \tensor_{\Z} \F_p
	\end{equation*}
	so by the lemma there is a prime $p$ with $H_n(X; \F_p) \neq 0$ and $H_{n - 1}(\Omega X; \F_p) \neq 0$. Applying the universal coefficient theorem once more, this time over the field $\F_p$, we see that 
	\begin{equation*}
		H_n(X; H_m(\Omega X; \F_p)) \isom H_n(X; \F_p) \tensor_{\F_p} H_m(\Omega X; \F_p)
	\end{equation*}
	The tensor product on the right hand side here is zero if and only if both $H_n(X; \F_p)$ and $H_m(\Omega X; \F_p)$ are since they are vector spaces and therefore free.

	This has the consequence that if $H_*(X; \Z)$ and $H_*(\Omega X; \Z)$ are both concentrated in degrees $0, \ldots, k$ and $0, \ldots, l$ respectively for some $k, l \in \N$, then so are the groups $H_*(X; H_*(\Omega X; \F_p))$ (with the appropriate bigrading), and if we take $k$ and $l$ to be tight then $H_k(X; H_l(\Omega X; \F_p)) \neq 0$.
	Looking now at the homological Serre spectral sequence for the fibre sequence $\Omega X \to * \to X$, this translates to $E^2_{k, l}$ being non-zero.
	But since $E^2_{n, m} = 0$ whenever $n > k$ or $m > l$, no incoming our outgoing differential at this group is ever non-trivial and it survives onto the $E^\infty$-page.
	But the total space is a point, so the $E^\infty$-page is a barren wasteland save for $E^\infty_{0, 0} \isom \F_p$, which is a contradiction!
	Thus the $\F_p$-homology of at least one of $X$ or $\Omega X$ must be infinite, whence we conclude that space cannot be (weakly) homotopy equivalent to a finite CW-complex since the homology of a finite CW-complex is again finite (via cellular homology).
\end{solution}

\begin{exercise}
	The $E_\infty$-term of the Serre spectral sequence will not determine the cohomology of the total space uniquely in general because of extension problems.
	Give an example of two fibre sequences $F \to Y \to X$ with $F = \RP^\infty$ and $X = \CP^\infty$ such that the $E_r$-pages of both Serre spectral sequences are isomorphic for all $r$, but $H^\bullet(E; \Z) \neq H^\bullet(E'; \Z)$.
	\begin{hint}
		Show that there are exactly two homotopy classes of maps $\CP^\infty \to K(\Zn{2}, 2)$ and consider their homotopy fibres.
	\end{hint}
\end{exercise}
\begin{solution}
	By the representability of ordinary cohomology, there is a natural bijection $[\CP^\infty, K(\Zn{2}, 2)]_* \isom H^2(\CP^\infty; \Zn{2}) \isom \Zn{2}$.
	Thus, exactly two homotopy classes of maps $\CP^\infty \to K(\Zn{2}, 2)$ exist; let $f$ represent the trivial and $g$ the non-trivial class and let $F_f$ and $F_g$ be their respective homotopy fibres.
	We then obtain long exact sequences
	\begin{equation*}
		\begin{tikzcd}
			0
					\ar[r]
				& \pi_2 F_f
					\ar[r]
				& \underbrace{\pi_2 \CP^\infty}_{\isom \Z}
					\ar[r, "f_*", "= 0"']
				& \underbrace{\pi_2 K(\Zn{2}, 2)}_{\isom \Zn{2}}
					\ar[r]
				& \pi_1 F_f
					\ar[r]
				& 0
		\end{tikzcd}
	\end{equation*}
	and
	\begin{equation*}
		\begin{tikzcd}
			0
					\ar[r]
				& \pi_2 F_g
					\ar[r]
				& \underbrace{\pi_2 \CP^\infty}_{\isom \Z}
					\ar[r, two heads, "g_*"]
				& \underbrace{\pi_2 K(\Zn{2}, 2)}_{\isom \Zn{2}}
					\ar[r]
				& \pi_1 F_g
					\ar[r]
				& 0
		\end{tikzcd}
	\end{equation*}
	using that $\CP^\infty$ is a $K(\Z, 2)$ to see that all remaining terms in the sequences are zero.
	By choice, we have that $f_* = 0$ and that $g_*$ realizes the quotient projection $\Z \twoheadrightarrow \Zn{2}$, so we can read off that
	\begin{equation*}
		\pi_k F_f \isom \begin{cases}
			\Z 		& k = 2 \\
			\Zn{2} 	& k = 1 \\
			0 		& \text{else}
		\end{cases}
	\end{equation*}
	and
	\begin{equation*}
		\pi_k F_g \isom \begin{cases}
			\Z 		& k = 2 \\
			0 		& \text{else}
		\end{cases}
	\end{equation*}
	In particular, this implies that $H_1(F_f) \isom \Zn{2}$ and therefore that $H^2(F_f)$ has 2-torsion whereas $H_1(F_g) = 0$ and $H_2(F_g) \isom \Z$ implies that $H^2(F_g) \isom \Z$ as well (all via Hurewicz and the universal coefficient theorem), so $H^*(F_f) \mathrel{\not\isom} H^*(F_g)$.
	However, taking another round of homotopy fibres we obtain fibre sequences
	\begin{gather*}
		\RP^\infty \to F_f \to \CP^\infty \\
		\RP^\infty \to F_g \to \CP^\infty 
	\end{gather*}
	using that $\Omega K(\Zn{2}, 2)$ is a $K(\Zn{2}, 1)$ and that all (CW-models of) $K(\Zn{2}, 1)$ are homotopy equivalent.
	Considering the (cohomological) Serre spectral sequence(s) associated to these two fibre sequences, we start out with $E_2^{p, q} = H^p(\CP^\infty; H^q(\RP^\infty)) \isom H^p(\CP^\infty) \tensor_{\Z} H^q(\RP^\infty)$ as $H^*(\CP^\infty)$ is free and of finite type, so $E_2^{p, q} \neq 0$ if and only if $p$ and $q$ are both even (as $H^*(\CP^\infty) \isom \Z[x]$ and $H^*(\RP^\infty) \isom \Z[y] / (2y)$ with $|x| = |y| = 2$).
	But any differential $d^r$ is a map of bidegree $(r, 1 - r)$, and as only one of $r$ and $1 - r$ is ever even at the same time, no nontrivial differentials exist and the $E_2$-page is in fact the $E_\infty$-page (in particular, no information about $F_f$ or $F_g$ influences these data via the differentials).
\end{solution}

\begin{exercise}
	Use the Serre spectral sequence to compute $H^*(F; \Z)$ for $F$ the homotopy fibre of a map $S^k \to S^k$ of degree $n$ for $k, n > 1$ and show that the cup product structure on $H^*(F; \Z)$ is trivial.
\end{exercise}
\begin{solution}
	Let $f\colon S^k \to S^k$ be a map of degree $n$ (i.e. $f_*\colon H_k(S_k) \to H_k(S_k)$ is multiplication by $n$).
	We will start out by gathering some information.
	First off, note that the induced map $f_*\colon \pi_k S^k \to \pi_k S^k$ is also multiplication by $n$:
	This follows from the naturality of the Hurewicz isomorphism and the fact that $S^k$ is $(k - 1)$-connected.
	We thus get a long exact sequence 
	\begin{equation*}
		\begin{tikzcd}
			\cdots 
					\ar[r]
				&\pi_{k + 1} S^k
					\ar[r]
				&\pi_k F 
					\ar[r]
				&\underbrace{\pi_k S^k}_{\isom \Z}
					\ar[r, "f_*", "= (\cdot n)"']
				&\underbrace{\pi_k S^k}_{\isom \Z}
					\ar[r]
				&\pi_{k - 1} F 
					\ar[r]
				& 0
		\end{tikzcd}
	\end{equation*}
	with the zero at the right coming from the group $\pi_{k - 1} S^k = 0$.
	As such, we have an isomorphism $\pi_{k - 1} F \isom \coker f_* \isom \coker(\Z \xto{\cdot n} \Z) = \Zn{n}$.
	As below degree $k - 1$ the long exact sequence collapses, $\pi_{k - 1} F$ is in fact the lowest non-trivial homotopy group of $F$; accordingly, Hurewicz tells us that $H_{n - 1}(F) \isom \pi_{k - 1} F \isom \Zn{n}$ is the lowest non-trivial homology group of $F$ (noting for the case $k = 2$ that $\Zn{n}$ is abelian).

	We will now move on to considering the Serre spectral sequence(s) for the fibre sequence $F \to S^k \xto{f} S^k$.
	Although our goal is to also determine the cohomology \emph{ring}, we will start out by calculating $H_*(F)$ using the homological version of the Serre spectral sequence because it avoids an extension problem on the $k$th antidiagonal arising in cohomology.
	Since the homology of $S^k$ is free and finite, we have isomorphisms
	\begin{equation*}
		E^2_{p, q} = H_p(S^k; H_q(F)) \isom H_p(S^k) \tensor H_q(F)
	\end{equation*}
	so $E^2_{p, q} = 0$ if $p \neq 0, k$ or $0 \neq q < k - 1$.
	In particular, the only possibly non-trivial differentials are $d^k\colon E^k_{k, q} \to E^k_{0, q + k - 1}$ for any $q$, so in particular the differential
	\begin{equation*}
		d^k\colon \underbrace{E^k_{k, 0}}_{\isom \Z} \to \underbrace{E^k_{0, k - 1}}_{\isom \Zn{n}}
	\end{equation*}
	must be surjective as the $(k - 1)$st antidiagonal on the $E_\infty$-page must be trivial since $H_{k - 1}(S^k) = 0$.
	Moving upwards, all antidiagonals become zero, and therefore all differentials isomorphisms, so using that $E^k_{k, q} \isom E^k_{0, q}$, we find by induction that 
	\begin{equation*}
		H_m(F) \isom \begin{cases}
			\Z 		& m = 0 \\
			\Zn{n} 	& (k - 1) \mid m \text{ and } m > 0 \\
			0 		& \text{else}
		\end{cases}
	\end{equation*}
	With help of the universal coefficient theorem, we deduce that
	\begin{equation*}
		H^m(F) \isom \begin{cases}
			\Z 		& m = 0 \\
			\Zn{n} 	& (k - 1) \mid (m - 1) \text{ and } m > 1 \\
			0 		& \text{else}
		\end{cases}
	\end{equation*}
	The distribution of nontrivial degrees in $H^*(F)$ is almost enough to conclude triviality of the cup product structure:
	If $k > 2$, then given $\alpha, \beta \in H^*(F)$ of degree $|\alpha| = a (k - 1) + 1$ and $|\beta| = b (k - 1) + 1$, respectively, we have $|\alpha \beta| = |\alpha| + |\beta| = (a + b)(k - 1) + 2$ which is not of the form $c(k - 1) + 1$ for any $c \in \N$ and therefore the product $\alpha \beta$ \enquote{falls through the grates.} 
	For $n = 2$, however, there are no grates to fall through as $H^m(F) \isom \Zn{n}$ for all $m \geq 2$, so we resort to studying the \emph{cohomological} Serre spectral sequence:
	Similar to before, we have an isomorphism
	\begin{equation*}
		E_2^{\bullet, \bullet} \isom H^\bullet(S^2) \tensor_{\Z} H^\bullet(F) \isom \Lambda(e) \tensor_{\Z} H^\bullet(F)
	\end{equation*}
	of bigraded rings for $e \in H^2(S^2)$ a generator since $H^*(S^2)$ is nice.
	The differential
	\begin{equation*}
		d_2\colon \underbrace{E_2^{0, 2}}_{\isom \Zn{n}} \to \underbrace{E_2^{2, 1}}_{= 0}
	\end{equation*}
	starting at the first interesting group in the $p = 0$ column is necessarily trivial, its codomain being 0.
	Above that, all differentials
	\begin{equation*}
		d_2\colon \underbrace{E_2^{0, q}}_{\isom \Zn{n}} \to \underbrace{E_2^{2, q - 1}}_{\isom \Zn{n}}
	\end{equation*}
	are isomorphisms for convergence reasons.
	Let $x_i \in H^i(F)$ be a generator for all $i \geq 2$, and let $x_1 \in H^1(F) = 0$ denote a \enquote{ghost class}\footnote{a fancy name for 0 :)}.
	Then $E^2_{2, q - 1}$ is generated by $e x_{q - i}$ for all $q \geq 2$, and we may without loss of generality arrange that $d_2(x_i) = e x_{i - 1}$ whenever $i \geq 2$ (noting the special case $d_2(x_2) = e x_1 = 0$).
	Let now $i$ be minimal with the property that $l x_i = x_{i - a} x_a$ for some $l \in \Zn{n}$, $2 \leq a \leq i - 2$.
	Using that differentials are graded derivations, we compute
	\begin{align*}
		l e x_{i - 1} = d_2(l x_i) &= d_2(x_{i - a} x_a) \\
								   &= d_2(x_{i - a}) \cdot x_a + (-1)^{i - a} x_{i - a} \cdot d_2(x_a) \\
								   &= e x_{i - a - 1} \cdot x_a + (-1)^{i - a} x_{i - a} \cdot e x_{a - 1} \\
								   &= e (x_{i - a - 1} \cdot x_a + (-1)^{i - a} x_{i - a} \cdot x_{a - 1})
	\end{align*}
	(noting that this makes sense with our convention for $x_1$).
	But $l e x_{i - 1} \neq 0$, so at least one of $x_{i - a - 1} x_a$ or $x_{i - a} x_{a - 1}$ is nonzero as well and therefore a multiple of the generator $x_{i - 1} \in H^{i - 1}(F)$, contradicting minimality of $i$.
	We thus conclude that $x_i x_j = 0$ for all $i, j$ and therefore that the ring structure on $H^*(F)$ is trivial.
\end{solution}

\begin{exercise}
	Use the Serre spectral sequence to compute $\pi_5(S^3, *)$.
	\begin{hint}
		\leavevmode
		\begin{enumerate}
			\item Use the Whitehead tower for $S^3$.
			\item There are different ways to go about this, but you might need to know $H_n(K(\Zn{2}, 3))$ for $n \leq 5$.
				For this, recall that
				\begin{equation*}
					H_n(K(\Zn{2}, 1)) \isom \begin{cases}
						\Z 		& n = 0 \\
						\Zn{2} 	& n \text{ odd} \\
						0 		& n \text{ even and } n > 0 \\
					\end{cases}
				\end{equation*}
				Start with the path-loop fibration of $K(\Zn{2}, 2)$ to compute $H_n(K(\Zn{2}, 2))$ in low degrees, then do the same for $K(\Zn{2}, 3)$.
		\end{enumerate}
	\end{hint}
\end{exercise}
\begin{solution}
	\emph{For following along, it might be helpful to consider the $E_2$-pages of the spectral sequences used in this argument printed on page \pageref{fig:firstspecseq}, although one should be careful that these include all the computed results already.}

	Let 
	\begin{equation*}
		\begin{tikzcd}
			\cdots
				\ar[d]
			\\
			\tau_{\geq 5} S^3
				\ar[d, "\phi_5"]
			\\
			\tau_{\geq 4} S^3
				\ar[d, "\phi_4"]
			\\
			S^3
		\end{tikzcd}
	\end{equation*}
	be a Whitehead tower for $S^3$\footnote{We adopt the convention that $\tau_{\geq n} X$ is $(n - 1)$-connected with $\pi_k(\tau_{\geq n} X, *) \isom \pi_k(X, *)$ for all $k \geq n$.}.
	Exploiting the usual Hurewicz trick, it is enough to compute $H_5(\tau_{\geq 5} S^3) \isom \pi_5(\tau_{\geq 5} S^3, *) \isom \pi_5(S^3, *)$.
	Noting that we already know that $\pi_4(\tau_{\geq 4} S^3, *) \isom \pi_4(S^3, *) \isom \Zn{2}$ from the lecture, we have two Serre fibrations
	\begin{gather*}
		K(\Zn{2}, 3) \to \tau_{\geq 5} S^3 \xto{\phi_5} \tau_{\geq 4} S^3 \mathrlap{\quad\text{and}}\\
		K(\Z, 2) \to \tau_{\geq 4} S^3 \xto{\phi_4} S^3
	\end{gather*}
	The second fibration should look familiar, and recalling the uniqueness theorem for Whitehead towers (or one particular method of their construction with which this coincides) we see that we have already calculated $H^*(\tau_{\geq 4} S^3)$ in example 2.31 in the lecture\footnote{Actually, the indices in the in-lecture calculation are incorrect (if I have noted them down correctly), but this is easily remedied by having a good look at it.} to be  
	\begin{equation*}
		H_n(\tau_{\geq 4} S^3) \isom \begin{cases}
			\Z 		& n = 0 \\
			\Zn{k} 	& n = 2k \text{ and } k > 1 \\
			0 		& \text{else}
		\end{cases}
	\end{equation*}
	With this input, we can attempt a calculation:
	Consider the Serre spectral sequence for $K(\Zn{2}, 3) \to \tau_{\geq 5} S^3 \to \tau_{\geq 4} S^3$ with $E_2$-page
	\begin{equation*}
		E^2_{p, q} = H_p(\tau_{\geq 4} S^3; H^q(K(\Zn{2}, 3)))
	\end{equation*}
	The lowest index at which a non-zero term appears is $E^2_{0, 3} \isom \Zn{2}$, with only interesting differential $d^4\colon \Zn{2} \isom E^4_{4, 0} \to E^4_{0, 3} \isom \Zn{2}$, which must be an isomorphism as $\tau_{\geq 5} S^3$ is 4-connected (via Hurewicz).
	Next, we note that $E^2_{0, 4} \isom H_4(K(\Zn{2}, 3)) = 0$ since it has no incoming interesting differentials and converges to 0.
	Moreover, we note that the only unknown group on the 5th antidiagonal of the convergence is $E^\infty_{0, 5}$, and that there is only one incoming interesting differential at that index, namely $d^6\colon \Zn{3} \isom E^6_{6, 0} \to E^6_{0, 5}$.
	This differential is trivial (either by noting that $H_*(K(\Zn{2}, 3))$ is 2-power-torsion via Serre class theory or by the direct calculation of $H_5(K(\Zn{2}, 3))$ which we are about to undertake), so we obtain that 
	\begin{equation*}
		\pi_5(\tau_{\geq 5} S^3, *) \isom H_5(\tau_{\geq 5} S^3) \isom E^\infty_{0, 5} \isom E^2_{0, 5} \isom H_5(K(\Zn{2}, 3))
	\end{equation*}
	Let us thus calculate this latter group.
	Our present spectral sequence will take us no further in this effort, so we make use of the usual trick of iteratively computing (co)homology of $K(G, n)$s via path-loop fibrations.
	We start out with $(\Omega K(\Zn{2}, 2) \to P K(\Zn{2}, 2) \to K(\Zn{2}, 2)) \htpyeqv (\RP^\infty \to * \to K(\Zn{2}, 2))$ and consider the Serre spectral sequence with $E_2$-page
	\begin{equation*}
		E_2^{*, *} = H^*(K(\Zn{2}, 2); H^*(\RP^\infty))
	\end{equation*}
	The first interesting trivial differential (between the first nonzero groups) is $d_3\colon \Zn{2} \isom E_3^{0, 2} \to E_3^{3, 0} \isom \Zn{2}$ which we see must be an isomorphism, being the only differential affecting either group and seeing as the total space is contractible.
	In multiplicative terms, if we let $x \in H^2(\RP^\infty)$ be the generator and $e \in H^3(K(\Zn{2}, 3))$ the unique nontrivial class, this is equivalent to $d_3(x) = e$.
	Staying in the same column(s), the next interesting differential is $d_3\colon \Zn{2} \isom E_3^{0, 4} \to E_3^{3, 2} \isom \Zn{2}$.
	Multiplicatively, this is given by the formula $d_3(x^2) = d_3(x) x + x d_3(x) = 2 x d_3(x) = 0$ and therefore trivial.
	We now look further to the right.
	The group $E_2^{4, 0}$ is trivial as it has no potentially nontrivial incoming differentials, all nontrivial groups left of it sitting in even rows.
	Looking now at the group $E_2^{5, 0}$, we find exactly two potential incoming differentials: One from $E_5^{0, 4}$ (which we have previously found to survive its outoing $d_3$ and which is unaffected by any $d_4$s for degree reasons) and one from $E_3^{2, 2} \isom E_2^{2, 2}$.
	One may at first glance be surprised that this latter group is non-zero, but via the universal coefficient theorem we compute $E_2^{2, 2} \isom H^2(K(\Zn{2}, 2); \Zn{2}) \isom \Hom(H_2(K(\Zn{2}, 2)), \Zn{2}) \dsum \Ext(H_1(K(\Zn{2}, 2)), \Zn{2}) \isom \Zn{2}$ as $H_2(K(\Zn{2}, 2)) \isom \Zn{2}$ and $H_1(K(\Zn{2}, 2)) = 0$.
	Accordingly, we find that $E_2^{5, 0} \eqcolon A$ is some mystery group of order 4, being given by the extension problem
	\begin{equation*}
		\begin{tikzcd}
			0 
					\ar[r]
				& \underbrace{\Zn{2}}_{\isom E_3^{2, 2}}
					\ar[r, hook]
				& A
					\ar[r, two heads]
				& \underbrace{\Zn{2}}_{\isom E_5^{0, 4}}
					\ar[r]
				& 0
		\end{tikzcd}	
	\end{equation*}
	as these differentials present the last chances for the involved groups to die.
	We will content ourselves with not determining $A$ further and move on.

	Next, consider the Serre spectral sequence of the fibre sequence $K(\Zn{2}, 2) \to * \to K(\Zn{2}, 3)$ with $E_2$-page
	\begin{equation*}
		E^2_{*, *} = H_*(K(\Zn{2}, 3); H_*(\Zn{2}, 2))
	\end{equation*}
	Noting that we have computed (via a standard application of the universal coefficient theorem) that
	\begin{equation*}
		H_k(K(\Zn{2}, 2)) \isom \begin{cases}
			\Z & k = 0 \\
			\Zn{2} & k = 2 \\
			A & k = 4 \\
			0 & k < 5 \text{ odd}
		\end{cases}
	\end{equation*}
	and that we know
	\begin{equation*}
		H_k(K(\Zn{2}, 3)) \isom \begin{cases}
			\Z & k = 0 \\
			\Zn{2} & k = 3 \\
			0 & k \neq 0, 3 \text{ and } k \leq 4
		\end{cases}
	\end{equation*}
	we quickly see that the of all the outgoing differentials of $E^2_{5, 0}$, the only interesting one is $d^5\colon E^5_{5, 0} \to E^5_{0, 4} \isom A$ as all other differentials have trivial codomain.
	The only other incoming differential to $E^5_{0, 4}$ is $d_3^{3, 2}\colon \Zn{2} \isom E_3^{3, 2} \to E^3_{0, 4}$ and these together must kill it, so in particular we see that $E^5_{0, 4} \neq 0$ as $\coker d_3^{3, 2}$ is a group of order at least two. Unfortunately, distinguishing between the cases $E^5_{5, 0} \isom A$ and $E^5_{5, 0} \isom \Zn{2}$ seems intractable here, so we resort to yet another spectral sequence.

	Observe that from any short exact sequence $0 \to G \to F \to H \to 0$ of abelian groups, we can obtain fibre sequences $K(G, n) \to K(F, n) \to K(H, n)$ for all $n \geq 1$ by realizing the second map as a map $\pi_n(K(F, n)) \to \pi_n(K(H, n))$\footnote{This can be done (in the case $n > 1$ which is the only case interesting us here) e.g. via direct construction by building up $K(F, n)$ and $K(H, n)$ starting with a wedge sum of copies of $S^n$ for each generator of the respective group, realizing the map on these generators, attaching $(n + 1)$-cells to realize all necessary relations between these generators, and finally killing all higher homotopy groups. It is part of the proof that this construction works that the map so constructed can be extended over all higher cells.}, taking the homotopy fibre and noting that the long exact sequence of homotopy groups implies that it is a $K(G, n)$.
	We apply this to 
	\begin{equation*}
		0 \to \Z \to \Z \to \Zn{2} \to 0
	\end{equation*}
	with $n = 3$ to obtain a fibre sequence $K(\Z, 3) \to K(\Z, 3) \to K(\Zn{2}, 3)$.
	Unfortunately, we do not have a good understanding of the (co)homology of $K(\Z, 3)$ right now, so we will have to make yet another detour through the path-loop fibration and study the fibre sequence $\CP^\infty \htpyeqv K(\Z, 2) \to * \to K(\Z, 3)$.
	Consider therefore the Serre spectral sequence with $E_2$-page
	\begin{equation*}
		E_2^{*, *} = H^*(K(\Z, 3); H^*(\CP^\infty)) \isom H^*(K(\Z, 3)) \tensor H^*(\CP^\infty) \isom H^*(K(\Z, 3)) \tensor \Z[x]
	\end{equation*}
	using that $H^*(\CP^\infty)$ is free and of finite type for the first and that $H^*(\CP^\infty) \isom \Z[x]$ with $x \in H^2(\CP^\infty)$ a generator for the second isomorphism.
	By Hurewicz and universal coefficients, the first nonzero reduced cohomology group of $K(\Z, 3)$ is $H^3(K(\Z, 3)) \isom \Z$.
	In particular, it seems wise to start the calculation by considering all differentials of the form $d_3\colon \Z \isom E_3^{0, 2q} \to E_3^{3, 2(q - 1)} \isom \Z$.
	The first such differential, $d_3^{0, 2}$ must be an isomorphism, presenting the only chance both for domain and codomain to die.
	In other words, $d_3(x) = e$ where $e \in E^3_{3, 0} \isom H^3(K(\Z, 3))$ is a generator.
	Two places up, we then have $d_3(x^2) = d_3(x) x + x d_3(x) = 2 e x$, i.e. $d_3^{0, 4}$ is multiplication by 2.
	Let us move to the right now.
	Clearly $E_2^{4, 0} = E_2^{5, 0} = 0$ as all differentials ending in these groups begin in a trivial group, so we turn to $E_2^{6, 0}$.
	Here we only have a single potential ingoing differential, namely $d_3^{3, 2}$.
	At $E_3^{3, 2}$ we therefore get both an incoming $d_3$, which is multiplication by 2, and an outgoing $d_3$, whose kernel must therefore contain $2\Z$ and which must kill both domain and codomain.
	The only way to make this work is if $E_3^{6, 0} \isom E_2^{6, 0} \isom \Zn{2}$.

	We now return to our penultimate sequence, for which we have now gathered enough information to proceed.
	We consider the Serre spectral sequence with $E_2$-page
	\begin{equation*}
		E_2^{*, *} \isom H^*(K(\Zn{2}, 3); H^*(K(\Z, 3)))
	\end{equation*}
	Keeping in mind that our goal is to calculate $H^6(K(\Zn{2}, 3)) \isom H_5(K(\Zn{2}, 3))$, we find ourselves lucky:
	There is \emph{no} incoming differential at $E_2^{6, 0}$, all other relevant nontrivial groups living at $(0, 0)$, $(0, 3)$, $(4, 0)$, $(4, 3)$, and $(0, 6)$.
	In other words, $E_2^{6, 0} \isom E_\infty^{6, 0}$ and therefore $E_2^{6, 0}$ contributes to the convergence to $H^6(K(\Z, 3)) \isom \Zn{2}$, so since $\Zn{2}$ does not admit any interesting filtrations we directly conclude that $E_2^{6, 0} = 0$ or $E_2^{6, 0} \isom \Zn{2}$.
	But the first case we have excluded earlier, so we finally conclude that $\Zn{2} \isom H^6(K(\Zn{2}, 3)) \isom H_5(K(\Zn{2}, 3)) \isom \pi_5(S^3, *)$.
\end{solution}

\begin{exercise}
	Compute the cohomology of the space $\map(S^1, S^3)$ of continuous (not necessarily basepoint-preserving) maps $f\colon S^1 \to S^3$.
\end{exercise}
\begin{solution}
	There is a Serre fibration $\Omega S^3 \xto{\iota} \map(S^1, S^3) \xto{\ev_1} S^3$ as $S^1$ is locally compact and well-pointed.
	Consider first the long exact sequence of homotopy groups associated to this fibration, which ends in
	\begin{equation*}
		\begin{tikzcd}[column sep = 2.6em]
			\cdots 
					\ar[r]
				& \pi_3(\map(S^1, S^3), *)
					\ar[r, "(\ev_1)_*"]
				& \pi_3(S^3, *)
					\ar[dll, rounded corners, swap, to path = {[pos = 1]
						-- ([xshift = 1em] \tikztostart.east)
						|- ($(\tikzcdmatrixname-1-2)!0.5!(\tikzcdmatrixname-2-2)$) \tikztonodes
						-| ([xshift = -2ex] \tikztotarget.west)
						-- (\tikztotarget)
					}]
			\\ 
			\pi_2(\Omega S^3, *)
					\ar[r, "\iota_*"]
				& \pi_2(\map(S^1, S^3), *)
					\ar[r]
				& 0
		\end{tikzcd}
	\end{equation*}
	as $S^3$ is 2-connected and $\Omega S^3$ simply connected.
	Note that $\ev_1\colon \map(S^1, S^3) \to S^3$ admits a section, namely $\sigma\colon S^3 \to \map(S^1, S^3)$ being given by $\sigma(x) = \const_x$, the constant loop at $x$.
	This implies that $(\ev_1)_*\colon \pi_*(\map(S^1, S^3), *) \to \pi_*(S^3, *)$ is surjective, so we can deduce that $\iota_*\colon \pi_2(\Omega S^3, *) \to \pi_2(\map(S^1, S^3), *)$ is an isomorphism, i.e. $\pi_2(\map(S^1, S^3), *) \to \pi_2(\Omega S^3, *) \isom \pi_3(S^3, *) \isom \Z$.
	By Hurewicz, this further implies that $H_2(\map(S^1, S^3)) \isom \Z$ and therefore $H^2(\map(S^1, S^3)) \isom \Z$ by the universal coefficient theorem.

	We now consider the Serre spectral sequence with $E_2$-page
	\begin{equation*}
		E_2^{*, *} = H^*(S^3; H^*(\Omega S^3)) \isom H^*(S^3) \tensor H^*(\Omega S^3) \isom \Lambda(e) \tensor \Gamma(x)
	\end{equation*}
	where $e \in H^3(S^3)$ and $x \in H^2(\Omega S^3)$ generators, with the first isomorphism stemming from the fact that $H^*(S^3)$ is free and finite and the second from the description of these rings calculated in the lecture.
	For the differentials, the only possibly nonzero candidates are all of the form $d_3\colon E_3^{0, q} \to E_3^{3, q - 2}$ with $q$ even and $\geq 2$.
	But note that the first such differential, $d_3^{0, 2}\colon \Z \isom E_3^{0, 2} \to E_3^{3, 0} \isom \Z$ must be trivial: Certainly $\ker d_3^{0, 2} \isom \Z$ as it is the only group contributing to the convergence to $H^2(\map(S^1, S^3)) \isom \Z$, so $d_3^{0, 2}$ must be multiplication by some $k > 1$; but then its cokernel is torsion and isomorphic to $H^3(\map(S^1, S^3))$ (as it is lonely remaining on the third antidiagonal) which is torsion-free via the universal coefficient theorem as by our calculation of $H_2(\map(S^1, S^3)) \isom \Z$ above.

	Long story short, $d_3^{0, 2}$ is trivial, and reexpressed in multiplicative terms we see that $d_3(x) = 0$.
	But every other $E_3^{0, 2q}$-group is generated by divided powers of $x$, so all other differentials must be trivial as well (as $0 = d^3(x^k) = d^3\big(k! \cdot \frac{x^k}{k!}\big)$ implies that $d^3\big(\frac{x^k}{k!}\big) = 0$ by freeness).
	In other words, $E_\infty^{*, *} \isom E_2^{*, *}$ as bigraded rings, so since no antigiagonal contains more than one nontrivial term, we obtain that $H^*(\map(S^1, S^3)) \isom \Lambda(e) \tensor \Gamma(x)$ with $e \in H^3(\map(S^1, S^3))$ and $x \in H^2(\map(S^1, S^3))$ generators.
\end{solution}
\begin{figure}[p!]
	\centering
	\tikzsetnextfilename{excs_postnikov5_specseq}
	\begin{tikzpicture}[
		pagemember/.style = {
			fill = white, 
			font = \scriptsize, 
			inner sep = 0.5pt
		},
		differential/.style = {
			commutative diagrams/every arrow, 
			thick, 
			preaction = {
				draw = white, 
				arrows = -, 
				line width = 0.5ex
			}
		}]
		\matrix[
			name = m, 
			nodes in empty cells, 
			matrix of math nodes, 
			nodes = {outer sep = 0ex, inner sep = 2pt},
			column sep = {4.5ex, between origins},
			%row sep = {4ex, between origins},
			row sep = 0.5ex,
			column 1/.style = {anchor = base east, font = \scriptsize}, 
			row 8/.style = {font = \scriptsize}] {
				5 &[-2ex] \Zn{2} & 0 & 0 & 0 	  & \Zn{2} & \Zn{2}	& ? 	 \\
				4 & 0  			 & 0 & 0 & 0 	  & 0 	   & 0 		& 0  	 \\
				3 & \Zn{2}		 & 0 & 0 & 0      & \Zn{2} & \Zn{2}	& ? 	 \\
				2 & 0  			 & 0 & 0 & 0 	  & 0  	   & 0 		& 0 	 \\
				1 & 0  			 & 0 & 0 & 0 	  & 0  	   & 0 		& 0 	 \\
				0 & \Z 			 & 0 & 0 & 0 	  & \Zn{2} & 0 		& \Zn{3} \\
			 	  & 0  			 & 1 & 2 & 3 	  & 4      & 5  	& 6 	 \\
		};

		% bounding box coordinates for the "graph" drawing area
		\coordinate (top left) at ($(m-1-2.north west) + (0, 0.3)$);
		\coordinate (bottom left) at (m-6-2.south west -| top left);
		\coordinate (bottom right) at ($(bottom left -| m-6-8.south east) + (0.3, 0)$);
		\coordinate (top right) at (bottom right |- top left);

		\draw[spectral sequence/axis, line cap = round] (bottom left) -- (bottom right) node[below] {$p$};
		\draw[spectral sequence/axis, line cap = round] (bottom left) -- (top left) node[left] {$q$};

		\node[font = \scriptsize, draw, anchor = west] at (top right) {$E^2_{p, q} = H_p(\tau_{\geq 4} S^3; H_q(K(\Zn{2}, 3)))$};
\end{tikzpicture}
	\caption{$E^2$-page of the Serre spectral sequence for $K(\Zn{2}, 3) \to \tau_{\geq 5} S^3 \to \tau_{\geq 4} S^3$}
	\label{fig:firstspecseq}
\end{figure}
\begin{figure}[p!]
	\centering
	\tikzsetnextfilename{excs_rpinfty_specseq}
	\begin{tikzpicture}[
		pagemember/.style = {
			fill = white, 
			font = \scriptsize, 
			inner sep = 0.5pt
		},
		differential/.style = {
			commutative diagrams/every arrow, 
			thick, 
			preaction = {
				draw = white, 
				arrows = -, 
				line width = 0.5ex
			}
		}]
		\matrix[
			name = m, 
			nodes in empty cells, 
			matrix of math nodes, 
			nodes = {outer sep = 0ex, inner sep = 2pt},
			column sep = {4.5ex, between origins},
			%row sep = {4ex, between origins},
			row sep = 0.5ex,
			column 1/.style = {anchor = base east, font = \scriptsize}, 
			row 6/.style = {font = \scriptsize}] {
				4 &[-2ex] \Zn{2} & 0 & \Zn{2} & \Zn{2} 	  & ? 	   & ? 		  	 \\
				3 & 0			 & 0 & 0 & 0 & 0 & 	0 			 \\
				2 & \Zn{2}		 & 0 & \Zn{2} & \Zn{2} 	  & ?  	   & ? 		 	 \\
				1 & 0  			 & 0 & 0 & 0 	  & 0  	   & 0 		 	 \\
				0 & \Z 			 & 0 & 0 & \Zn{2} & 0 	   & A 		  \\
			 	  & 0  			 & 1 & 2 & 3 	  & 4      & 5  	 	 \\
		};

		% bounding box coordinates for the "graph" drawing area
		\coordinate (top left) at ($(m-1-2.north west) + (0, 0.3)$);
		\coordinate (bottom left) at (m-5-2.south west -| top left);
		\coordinate (bottom right) at ($(bottom left -| m-5-7.south east) + (0.3, 0)$);
		\coordinate (top right) at (bottom right |- top left);

		\draw[spectral sequence/axis, line cap = round] (bottom left) -- (bottom right) node[below] {$p$};
		\draw[spectral sequence/axis, line cap = round] (bottom left) -- (top left) node[left] {$q$};

		\node[font = \scriptsize, draw, anchor = west] at (top right) {$E_2^{p, q} = H^p(K(\Zn{2}, 2); H^q(\RP^\infty))$};
\end{tikzpicture}
	\caption{$E^2$-page of the Serre spectral sequence for $\RP^\infty \to * \to K(\Zn{2}, 2)$}
\end{figure}
\begin{figure}[p!]
	\centering
	\tikzsetnextfilename{excs_K_Zn2_2_K_Zn2_3_specseq}
	\begin{tikzpicture}[
		pagemember/.style = {
			fill = white, 
			font = \scriptsize, 
			inner sep = 0.5pt
		},
		differential/.style = {
			commutative diagrams/every arrow, 
			thick, 
			preaction = {
				draw = white, 
				arrows = -, 
				line width = 0.5ex
			}
		}]
		\matrix[
			name = m, 
			nodes in empty cells, 
			matrix of math nodes, 
			nodes = {outer sep = 0ex, inner sep = 2pt},
			column sep = {4.5ex, between origins},
			row sep = 0.5ex,
			column 1/.style = {anchor = base east, font = \scriptsize}, 
			row 6/.style = {font = \scriptsize}] {
				4 &[-2ex] A 	 & 0 & 0 & ? 	  & ? 	   & ? 		  	 \\
				3 & 0			 & 0 & 0 & 0 & 0 & 	0 			 \\
				2 & \Zn{2}		 & 0 & 0 & \Zn{2} 	  & \Zn{2}  	   & \Zn{2} 		 	 \\
				1 & 0  			 & 0 & 0 & 0 	  & 0  	   & 0 		 	 \\
				0 & \Z 			 & 0 & 0 & \Zn{2} & 0 	   & \Zn{2} 		  \\
			 	  & 0  			 & 1 & 2 & 3 	  & 4      & 5  	 	 \\
		};

		% bounding box coordinates for the "graph" drawing area
		\coordinate (top left) at ($(m-1-2.north west -| m-3-2.west) + (0, 0.3)$);
		\coordinate (bottom left) at (m-5-2.south west -| top left);
		\coordinate (bottom right) at ($(bottom left -| m-5-7.south east) + (0.3, 0)$);
		\coordinate (top right) at (bottom right |- top left);

		\draw[spectral sequence/axis, line cap = round] (bottom left) -- (bottom right) node[below] {$p$};
		\draw[spectral sequence/axis, line cap = round] (bottom left) -- (top left) node[left] {$q$};

		\node[font = \scriptsize, draw, anchor = west] at (top right) {$E^2_{p, q} = H_p(K(\Zn{2}, 3); H_q(K(\Zn{2}, 2)))$};
\end{tikzpicture}
	\caption{$E^2$-page of the Serre spectral sequence for $K(\Zn{2}, 2) \to * \to K(\Zn{2}, 3)$}
\end{figure}
\begin{figure}[p!]
	\centering
	\tikzsetnextfilename{excs_cpinfty_specseq}
	\begin{tikzpicture}[
		pagemember/.style = {
			fill = white, 
			font = \scriptsize, 
			inner sep = 0.5pt
		},
		differential/.style = {
			commutative diagrams/every arrow, 
			thick, 
			preaction = {
				draw = white, 
				arrows = -, 
				line width = 0.5ex
			}
		}]
		\matrix[
			name = m, 
			nodes in empty cells, 
			matrix of math nodes, 
			nodes = {outer sep = 0ex, inner sep = 2pt},
			column sep = {4ex, between origins},
			%row sep = {4ex, between origins},
			row sep = 0.5ex,
			column 1/.style = {anchor = base east, font = \scriptsize}, 
			row 6/.style = {font = \scriptsize}] {
				4 &[-2ex] \Z & 0 & 0 & \Z & 0 & 0 & \Zn{2} \\
				3 & 0  		 & 0 & 0 & 0  & 0 & 0 & 0 	   \\
				2 & \Z 		 & 0 & 0 & \Z & 0 & 0 & \Zn{2} \\
				1 & 0  		 & 0 & 0 & 0  & 0 & 0 & 0 	   \\
				0 & \Z 		 & 0 & 0 & \Z & 0 & 0 & \Zn{2} \\
			 	  & 0 & 1 & 2 & 3 & 4 & 5 & 6 \\
		};

		% bounding box coordinates for the "graph" drawing area
		\coordinate (bottom left) at (m-5-2.south west);
		\coordinate (bottom right) at ($(bottom left -| m-5-8.south east) + (0.3, 0)$);
		\coordinate (top left) at ($(m-1-2.north west) + (0, 0.3)$);
		\coordinate (top right) at (bottom right |- top left);

		\draw[spectral sequence/axis, line cap = round] (bottom left) -- (bottom right) node[below] {$p$};
		\draw[spectral sequence/axis, line cap = round] (bottom left) -- (top left) node[left] {$q$};

		\node[font = \scriptsize, draw, anchor = west] at (top right) {$E_2^{p, q} = H^p(K(\Z, 3)) \tensor H^q(\CP^\infty)$};
\end{tikzpicture}
	\caption{$E_2$-page of the Serre spectral sequence for $\CP^\infty \to * \to K(\Z, 3)$}
\end{figure}
\begin{figure}[p!]
	\centering
	\tikzsetnextfilename{excs_K_Z_3_K_Zn2_3_specseq}
	\begin{tikzpicture}[
		pagemember/.style = {
			fill = white, 
			font = \scriptsize, 
			inner sep = 0.5pt
		},
		differential/.style = {
			commutative diagrams/every arrow, 
			thick, 
			preaction = {
				draw = white, 
				arrows = -, 
				line width = 0.5ex
			}
		}]
		\matrix[
			name = m, 
			nodes in empty cells, 
			matrix of math nodes, 
			nodes = {outer sep = 0ex, inner sep = 2pt},
			column sep = {4.5ex, between origins},
			%row sep = {4ex, between origins},
			row sep = 0.5ex,
			column 1/.style = {anchor = base east, font = \scriptsize}, 
			row 8/.style = {font = \scriptsize}] {
				6 &[-1.5ex] \Zn{2} & 0 & 0 & \Zn{2} & \Zn{2} & \Zn{2} & \Zn{2} \\
				5 & 0  			 & 0 & 0 & 0 	  & 0 	   & 0 	 	& 0 	 \\
				4 & 0  			 & 0 & 0 & 0 	  & 0 	   & 0 		& 0  	 \\
				3 & \Z 			 & 0 & 0 & 0 	  & \Zn{2} & 0 		& \Zn{2} \\
				2 & 0  			 & 0 & 0 & 0 	  & 0  	   & 0 		& 0 	 \\
				1 & 0  			 & 0 & 0 & 0 	  & 0  	   & 0 		& 0 	 \\
				0 & \Z 			 & 0 & 0 & 0 	  & \Zn{2} & 0 		& \Zn{2} \\
			 	  & 0  			 & 1 & 2 & 3 	  & 4      & 5  	& 6 	 \\
		};

		% bounding box coordinates for the "graph" drawing area
		\coordinate (top left) at ($(m-1-2.north west) + (0, 0.3)$);
		\coordinate (bottom left) at (m-7-2.south west -| top left);
		\coordinate (bottom right) at ($(bottom left -| m-7-8.south east) + (0.3, 0)$);
		\coordinate (top right) at (bottom right |- top left);

		\draw[spectral sequence/axis, line cap = round] (bottom left) -- (bottom right) node[below] {$p$};
		\draw[spectral sequence/axis, line cap = round] (bottom left) -- (top left) node[left] {$q$};

		\node[font = \scriptsize, draw, anchor = west] at (top right) {$E_2^{p, q} = H^p(K(\Zn{2}, 3); H^q(K(\Z, 3)))$};
\end{tikzpicture}
	\caption{$E_2$-page of the Serre spectral sequence for $K(\Z, 3) \to K(\Z, 3) \to K(\Zn{2}, 3)$}
\end{figure}

\begin{exercise}
	\leavevmode
	\begin{enumerate}
		\item Consider the fibre sequence
			\begin{equation*}
				S^0 \to S^1 \xto{p} S^1
			\end{equation*}
			where $p\colon S^1 \to S^1$ denotes the 2-sheeted cover $p(z) = z^2$.
			Let $H_0(F_{({{-}})}, \Z)$ denote the local coefficient system on $S^1$ induced by the fibration.

			Show that local coefficient homology $H_*(S^1, H_0(F_{({{-}})}; \Z))$ is not isomorphic to the homology $H_*(S^1, H_0(S^1; \Z))$ with the constant coefficient system $H_0(S; \Z)$.
		\item Let $X$ be a connected CW-complex with basepoint $x \in X$ and universal cover $q\colon \tilde{X} \to X$.
			We suppose that $\pi_1(X, x)$ acts on another CW-complex $Y$.
			This also induces an action on homology,
			\begin{equation*}
				\alpha\colon \pi_1(X, x) \times H_*(Y; \Z) \to H_*(Y; \Z)
			\end{equation*}
			Now consider the fibre bundle
			\begin{equation*}
				Y \to \tilde{X} \times_{\pi_1(X, x)} Y \xto{q \times x} X
			\end{equation*}
			with $\pi_1(X, x)$ acting on $\tilde{X}$ through deck transformations and $\tilde{X} \times_{\pi_1(X, x)} Y$ is defined as the quotient of $\tilde{X} \times Y$ by the diagonal $\pi_1(X, x)$-action. 
			(You do not have to show that this is a fibre bundle.)

			As described in the lecture, the homologies of the fibres form a functor on the fundamental groupoid of $X$.
			In particular, the fundamental group $\pi_1(X, x)$ acts on the homology of the fibre at the point $x$, which canonically identifies with $Y$.
			Let $\beta\colon \pi_1(X, x) \times H_*(Y; \Z) \to H_*(Y; \Z)$ denote this action.

			Show that the two actions $\alpha$ and $\beta$ on homology are the same.
	\end{enumerate}
\end{exercise}

\begin{exercise}
	Prove the following, making use of the Serre spectral sequence:
	\begin{theorem}[Leray-Hirsch]
		Let $F \xto{i} E \xto{p} B$ be a fibration, with $B$ path-connected.
		Assume that there exists a set of classes $\{c_j\} \in H^*(E; \Z)$, of which only finitely many lie in a given degree, such that their restrictions $\{i^*(c_j)\}$ form a $\Z$-basis for the cohomology $H^*(F; \Z)$ of the fibre $F$.
		Then the set $\{c_j\}$ is a basis of $H^*(E; \Z)$ as a module over the ring $H^*(B; \Z)$.
	\end{theorem}
\end{exercise}
\begin{solution}
	We will start out by proving a lemma:
	\begin{lemma}
		Let $F \xto{i} E \xto{p} B$ be fibration with $B$ path-connected such that $i^*\colon H^*(E) \to H^*(F)$ is surjective.
		Then:
		\begin{enumerate}
			\item The action of $\pi_1(B, *)$ on $H^*(F)$ is trivial.
			\item All differentials of the form $d_n^{0, q}\colon E_n^{0, q} \to E_n^{n, q - n + 1}$ for $n, q \geq 0$ in the associated cohomological Serre spectral sequence are trivial.
		\end{enumerate}
	\end{lemma}
	\begin{smallproof}
		Let us consider the Serre spectral sequence for the given fibration with $E_2$-term
		\begin{equation*}
			E_2^{p, q} = H^p(B; H^q(F_{-}; \Z))
		\end{equation*}
		We then have edge homomorphisms $e(p)\colon H^q(E) \to E_\infty^{0, q} \incl E_2^{0, q} \isom H^0(B; H^q(F_{-})) \to H^q(F_*)$ for $* \in E$ any basepoint which agrees with the surjective map $i^*\colon H^q(E) \to H^*(F)$.
		In particular, this means that $H^0(B; H^q(F_{-})) \to H^q(F_*)$ must be surjective, so the action of $\pi_1(B, *)$ on $H^*(F)$ must be trivial\footnote{Life is hard (and short) enough so we will not prove this here, but it is not too difficult to see that $H^0(B; H^q(F_{-})) \isom H^q(F_*)$ is the \enquote{maximal} case that occurs when the $\pi_1(B, *)$-action is trivial, any element acting nontrivially killing some part of the left hand side by a calculation similar to what we did in exercise 1.1. (Apparently, if one is so inclined, one can show that $H^0(B; H^q(F_{-}))) \isom H^q(F)^{\pi_1(B, *)}$, the fixed points of $H^q(F)$ under the $\pi_1(B, *)$-action.)}.
		
		For the second statement, note that our spectral sequence now satisfies
		\begin{equation*}
			E_2^{p, q} \isom H^p(B; H^q(F; \Z))
		\end{equation*}
		as we have just shown the local coefficient system to be constant.
		In the leftmost column (of the first quadrant) we then recover $H^q(F) \isom E_2^{0, q}$.
		Accordingly, if any outgoing differential there is non-trivial, it kills some element of its domain (by not including it in its kernel), so we obtain that $E_\infty^{0, q} \subsetneq E_2^{0, q}$.
		But then $E_2^{0, q} \setminus E_\infty^{0, q}$ is not in the image of the edge homomorphism $e(p)\colon H^q(E) \to H^q(F)$ discussed above, contradicting its surjectivity; thus no such differential exists.
	\end{smallproof}

	Back in the problem setting, $i^*$ is surjective by assumption, so we are freed from working with local coefficients.
	Additionally, the assumption that $H^*(F)$ is free and of finite type gives rise to Serre spectral sequence with $E_2$-term
	\begin{equation*}
		E_2^{*, *} = H^*(B; H^*(F)) \isom H^*(B) \tensor H^*(F)
	\end{equation*}
	The second part of our lemma now tells us that all differentials originating on the $q$-axis are trivial, and this implies already that \emph{all} differentials are trivial:
	All groups on the $E_2$-page are generated by elements of the form $i^*(c_j) b_k$ with $b_k \in E_2^{p, 0} \isom H^p(B)$ for some $p, j, k \geq 0$, but
	\begin{equation*}
		d_2(i^*(c_j) b_k) = \underbrace{d_2(i^*(c_j))}_{= 0} b_k \pm i^*(c_j) \underbrace{d_2(b_k)}_{= 0} = 0
	\end{equation*}
	as the $i^*(c_j)$ live on the $q$- and the $b_k$ on the $p$-axis (where all outgoing differentials run off the page), so $E_3^{*, *} \isom E_2^{*, *}$ and inductively  by the same argument $E_\infty^{*, *} \isom E_2^{*, *}$.
	To finish the proof, we make use of lemma 2.28 from the lecture\footnote{\enquote{Given filtered graded abelian groups $(A, F), (B, G)$ such that the filtrations are in every degree eventually 0, any graded filtration-preserving morphism $f\colon A \to B$ that induces an isomorphism on all associated graded pieces is an isomorphism.}}.
	Consider the map $f\colon H^*(B) \tensor H^*(F) \to H^*(E)$ given by (linearly extending) $f(b \tensor i^*(c_j)) \coloneq p^*(b) \smile c_j$.
	This map preserves the natural grading on the domain and is compatible with the filtrations induced on domain and codomain from the $E_\infty$-page (in particular, filtering the domain by degree of the $H^*(F)$-factor).
	By our description of the $E_\infty$-page (in particular noting that it agrees with its multiplicative structure), it is an isomorphism on graded pieces, so we conclude that it is an isomorphism of graded abelian groups, which was to be shown.
\end{solution}

\begin{exercise}\label{ex:postnikovaction}
	Let $X$ be a connected CW-complex with basepoint $x \in X$.
	Recall that for each $n \geq 1$, $\pi_1(X, x)$ acts on $\pi_n(X, x)$.
	In the homotopy category, this induces a natural action of $\pi_1 X$ on $K(\pi_n(X, x), n)$ which further induces an action on homology
	\begin{equation*}
		\alpha\colon \pi_1(X, x) \times H_* K(\pi_n(X, x), n) \to H_* K(\pi_n(X, x), n)
	\end{equation*}
	Recall that for each $n \geq 2$, the Postnikov tower gives a fibre sequence
	\begin{equation*}
		K(\pi_n X, n) \to \tau_{\leq n} X \to \tau_{\leq n - 1} X
	\end{equation*}
	Let $f_{n - 1}\colon X \to \tau_{\leq n - 1} X$ be the canonical map and let $y = f_{n - 1}(x)$.
	Then $\pi_1(\tau_{\leq n - 1} X, y)$ acts on the homology of the homotopy fibre at the point $y$, i.e. $K(\pi_n(X, x), n)$.
	Let
	\begin{equation*}
		\beta\colon \pi_1(\tau_{\leq n - 1} X, y) \times H_* K(\pi_n(X, x), n) \to H_* K(\pi_n(X, x), n)
	\end{equation*}
	denote this action.
	Show that the two actions $\alpha$ and $\beta$ on homology are the same, under the identification
	\begin{equation*}
		(f_{n - 1})_*\colon \pi_1(X, x) \xto{\isom} \pi_1(\tau_{\leq n - 1} X, y)
	\end{equation*}
\end{exercise}

\begin{exercise}
	As discussed in the lecture, the first $p$-torsion class in $\pi_* S^3$ is found in degree $2p$.
	Recall that for all $n \geq 3$, the Hopf map $\eta\colon S^3 \to S^2$ induces an isomorphism $\eta_*\colon \pi_n S^3 \isom \pi_n S^2$.
	We let $x \in \pi_{2p} S^2$ be a $p$-torsion class.
	Consider the suspension $\Sigma x \in \pi_{2p + 1} S^3$.

	Show that if $p$ is odd, then $\Sigma x = 0$.

	Bonus: Show that when $p = 2$, $\Sigma x$ suspends to a generator of $\pi_5 S^3$.
\end{exercise}
\begin{solution}
	Consider the map $S^3 \to K(\Z, 3)$ inducing the identity on $\pi_3({{-}})$ and take its homotopy fibre to obtain a fibre sequence $F \to S^3 \to K(\Z, 3)$.
	We have seen the calculation of 
	\begin{equation*}
		H_n(F) \isom \begin{cases}
			\Zn{k} 	& n = 2k\; (k > 1) \\
			0 		& \text{else}
		\end{cases}
	\end{equation*}
	in the lecture.
	Using henceforth the letter $r$ in place of $p$ so as to avoid clashing with the usual notation for spectral sequences, we now look at the map $\rho\colon F \to K(\Zn{r}, 2r)$ inducing an isomorphism on the $r$-torsion part of $\pi_{2r}(F)$ (we have seen in the lecture that the $r$-torsion part of $\pi_{2r}(F) \isom \pi_{2r}(S^3)$ is isomorphic to $\Zn{r}$) and take another round of homotopy fibres to obtain a fibre sequence $F' \to F \to K(\Zn{r}, 2r)$.
	The only \enquote{interesting}\footnote{since in all lower and higher degrees $\pi_*(K(\Zn{r}, 2r))$ becomes 0 so that $\pi_k(F') \isom \pi_k(F)$} excerpt of the associated long exact sequence of homotopy groups reads
	\begin{equation*}
		\begin{tikzcd}
			0
					\ar[r]
				& \pi_{2r}(F')
					\ar[r]
				& \pi_{2r}(F)
					\ar[r, two heads]
				& \pi_{2r}(K(\Zn{r}, 2r))
				\ar[dll, rounded corners, to path = {[pos = 1]
					-- ([xshift = 1em] \tikztostart.east)
					|- ($(\tikzcdmatrixname-1-3)!0.5!(\tikzcdmatrixname-2-3)$) \tikztonodes
					-| ([xshift = -1em] \tikztotarget.west)
					-- (\tikztotarget)
				}, swap, "0"]
			\\
				& \pi_{2r - 1}(F')
					\ar[r]
				& \pi_{2r - 1}(F)
					\ar[r]
				& 0
		\end{tikzcd}
	\end{equation*}
	with the connecting map trivial since the preceding map is surjective by construction.
	In other words, we have
	\begin{equation*}
		\pi_k(F') \isom \begin{cases}
			\pi_k(F) / \ker \rho & k = 2r \\
			\pi_k(F) 			 & \text{else}
		\end{cases}
	\end{equation*}
	so in particular $\pi_k(F') \in \catfont{C}$ for all $k < 2r + 1$ where $\catfont{C}$ is the Serre class of finite abelian groups of order coprime to $r$.
	If we can show that $\pi_{2r + 1}(F') \isom \pi_{2r + 1}(F) \isom \pi_{2r + 1}(S^3) \in \catfont{C}$ we are obviously done, and by the modulo $\catfont{C}$ Hurewicz theorem this reduces to showing that $H_{2r + 1}(F') \in \catfont{C}$.
	We will now in fact show that $H_{2r + 1}(F') = 0 \in \catfont{C}$.

	Consider the Serre spectral sequence with $E^2$-page
	\begin{equation*}
		E^2_{p, q} = H_p(K(\Zn{r}, 2r); H_q(F'))
	\end{equation*}
	(see figure \ref{fig:specseq1}).
	\begin{figure}[ht]
		\centering
		\tikzsetnextfilename{excs_htpyfibre_specseq}
		\begin{tikzpicture}
			\matrix[
				name = m, 
				nodes in empty cells, 
				matrix of math nodes, 
				nodes = {outer sep = 0ex, inner sep = 2pt},
				column sep = {3.1em, between origins},
				row sep = 0.5ex,
				column 1/.style = {anchor = base east, font = \scriptsize}, 
				row 10/.style = {font = \scriptsize}] {
					2r + 1 			&[0.2em] ?																											\\
					2r 				& 0 																												\\
					2r - 1 			& H_{2r - 1}(F) = 0  																								\\
					\vdotswithin{2} & \vdotswithin{H_3(F)} 	& 	& 			& 			& \vdotswithin{0}												\\
					4 				& H_4(F) \isom \Zn{2} 	& 	& 			& 			& \Zn{2} \tensor \Zn{p} = 0 									\\
					3 				& H_3(F) = 0 			& 	& 			& 			& 0 															\\
					2 				& H_2(F) = 0 			& 	& 			& 			& 0 															\\
					1 				& H_1(F) = 0 			& 	& 			& 			& 0 															\\
					0 				& \Z 					& 0 & \cdots 	& 0 		& \Zn{r} 						& 0 		& H_{2r + 2}(K) 	\\
									& 0  					& 1 & \cdots    & 2r - 1 	& 2r 							& 2r + 1 	& 2r + 2 			\\
			};

			% bounding box coordinates for the "graph" drawing area
			\coordinate (top left) at ($(m-1-2.north west -| m-3-2.west) + (0, 0.3)$);
			\coordinate (bottom left) at (m-9-8.south -| top left);
			\coordinate (bottom right) at ($(bottom left -| m-9-8.south east) + (0.3, 0)$);
			\coordinate (top right) at (bottom right |- top left);

			\draw[spectral sequence/axis, line cap = round] (bottom left) -- (bottom right) node[below] {$p$};
			\draw[spectral sequence/axis, line cap = round] (bottom left) -- (top left) node[left] {$q$};

			\node[font = \scriptsize, draw, anchor = west] at (m-9-6 |- m-2-2) {$E^2_{p, q} = H_p(K(\Zn{r}, 2r); H_q(F'))$};
			\path[commutative diagrams/.cd, every arrow, every label, thick] 
				(m-9-8) edge[commutative diagrams/two heads, 
								swap, 
								"$d^{2r + 2}$" {pos = .7}, 
								preaction = {
									draw = white, 
									arrows = -, 
									line width = 0.7ex
							}] (m-1-2); % differential
		\end{tikzpicture}
		\caption{$E^2$-page of the Serre spectral sequence for $F' \to F \to K(\Zn{r}, 2r)$. \\ Only groups (tangentially) relevant to the argument presented above are shown and most zeroes are omitted. ($K = K(\Zn{r}, 2r)$)}
		\label{fig:specseq1}
	\end{figure}
	This has $E^2_{p, 0} = 0$ for all $p < 2r$ as well as $E^2_{2r, 0} \isom \Zn{r}$ and $E^2_{2r + 1, 0} = 0$\footnote{This latter equality is a consequence of the often omitted additional statement in the Hurewicz theorem that if $X$ is $(n - 1)$-connected ($n \geq 2$), then the Hurewicz map $\pi_{n + 1}(X) \to H_{n + 1}(X)$ is surjective. This is easily observed from the proof of said theorem that we saw a few weeks ago. Alternatively, it is also not hard to argue the special case that $H_{n + 1}(K(G, n)) = 0$ for all $G, n$ directly via the Serre spectral sequence.}, while in the leftmost column we find that $E^2_{0, q} \isom H_q(F)$ in the range $0 \leq q < 2r$ and $E^2_{0, 2r} = 0$ as that index has no nontrivial incoming differentials and $E^2_{2r, 0} \isom \Zn{r} \isom H^{2r}(F)$ already survives to the $E^\infty$-page. 
	Consequently, the first differential of note is $d^{2r + 2}\colon H_{2r + 2}(K(\Zn{r}, 2r)) \isom E^{2r + 2}_{2r + 2, 0} \to E^{2r + 2}_{0, 2r + 1} \isom H_{2r + 1}(F')$.
	Observe that this differential must be surjective since no other differential affects its target which must die before the $E^\infty$-page, sitting in the convergence antidiagonal for $H_{2r + 1}(F) = 0$.

	If we are now lucky\footnote{and why wouldn't we be}, we can show that $H_{2r + 2}(K(\Zn{r}, 2r)) = 0$ and be done.
	How could one go about this?
	Observe that generally if $H_{k + 2}(K(\Zn{r}, k)) = 0$ for some $k \geq 2$, we obtain $H_{k + 3}(K(\Zn{r}, k + 1)) = 0$ by considering the homological Serre spectral sequence for the fibre sequence $K(\Zn{r}, k) \to * \to K(\Zn{r}, k + 1)$ and applying a common sparsity argument (knowing that the homology on the axes is concentrated in degrees $0, k$  on the vertical and $0, k + 1$ on the horizontal axis in the ranges $0 \leq p, q \leq k + 2$, respectively, it is easy to see that $E^2_{k + 3, 0}$ survives unscathed to the $E^\infty$-page and must therefore be 0).
	This immediately implies the result if we can put the induction on solid footing, so consider the fibre sequence $K(\Z, 3) \to K(\Z, 3) \to K(\Zn{r}, 3)$ associated to the short exact sequence
	\begin{equation*}
		\begin{tikzcd}
			0 
					\ar[r]
				& \Z
					\ar[r, "\cdot r"]
				& \Z
					\ar[r]
				& \Zn{r}
					\ar[r]
				& 0
		\end{tikzcd}
	\end{equation*}
	of abelian groups.
	Noting that we have computed\footnote{in our solution to exercise 4.1} the homology of $K(\Z, 3)$ in low degrees to be
	\begin{equation*}
		H_k(K(\Z, 3)) \isom \begin{cases}
			\Z 		& k = 0, 3 \\
			\Zn{2} 	& k = 5 \\
			0 		& \text{otherwise (up to } k \leq 5\text{)}
		\end{cases}
	\end{equation*}
	we get a Serre spectral sequence with $E^2$-page
	\begin{equation*}
		E^2_{p, q} = H_p(K(\Zn{r}, 3); H_q(K(\Z, 3)))
	\end{equation*}
	(cf. figure \ref{fig:specseq2}).
	\begin{figure}[ht]
		\centering
		\tikzsetnextfilename{excs_K_Z_3_K_Znp_3_specseq}
		\begin{tikzpicture}
			\matrix[
				name = m, 
				nodes in empty cells, 
				matrix of math nodes, 
				nodes = {outer sep = 0ex, inner sep = 2pt},
				column sep = {4.5ex, between origins},
				row sep = 0.5ex,
				column 1/.style = {anchor = base east, font = \scriptsize}, 
				row 7/.style = {font = \scriptsize}] {
					6 &[-1.5ex] \Zn{2} 	& 0 & 0 & 0 	 & 0  	& 0 \\
					4 & 0  			  	& 0 & 0 & 0 	 & 0 	& 0	\\
					3 & \Z 			 	& 0 & 0 & \Zn{r} & 0 	& ? \\
					2 & 0  			 	& 0 & 0 & 0  	 & 0 	& 0 \\
					1 & 0  			 	& 0 & 0 & 0  	 & 0 	& 0 \\
					0 & \Z 			 	& 0 & 0 & \Zn{r} & 0 	& ? \\
					& 0  			 	& 1 & 2 & 3      & 4  	& 5 \\
			};

			% bounding box coordinates for the "graph" drawing area
			\coordinate (top left) at ($(m-1-2.north west) + (0, 0.3)$);
			\coordinate (bottom left) at (m-6-2.south west -| top left);
			\coordinate (bottom right) at ($(bottom left -| m-6-7.south east) + (0.3, 0)$);
			\coordinate (top right) at (bottom right |- top left);

			\draw[spectral sequence/axis, line cap = round] (bottom left) -- (bottom right) node[below] {$p$};
			\draw[spectral sequence/axis, line cap = round] (bottom left) -- (top left) node[left] {$q$};

			\node[font = \scriptsize, draw, anchor = west] at (top right) {$E^2_{p, q} = H_p(K(\Zn{r}, 3); H_q(K(\Z, 3)))$};
		\end{tikzpicture}
		\caption{$E^2$-page of the Serre spectral sequence for $K(\Z, 3) \to K(\Z, 3) \to K(\Zn{r}, 3)$.}
		\label{fig:specseq2}
	\end{figure}
	At the index $(5, 0)$ which we are trying to compute, we see that no outgoing differential hits anything interesting, so the group survives to the $E^\infty$-page as-is.
	But on the 5th antidiagonal the spectral sequence converges to $H_5(K(\Z, 3)) \isom \Zn{2}$, so if $r$ is an odd prime\footnote{For $r = 2$ this fails and in fact $H_5(K(\Zn{2}, 3)) \isom \Zn{2}$ as we showed for exercise 4.1.} we conclude that $H_5(K(\Zn{r}, 3)) = 0$ (since $\Zn{2}$ has no nontrivial subgroups of odd order), which by our previous arguments implies that $H_{2r + 2}(K(\Zn{r}, 2r)) = 0$ so that ultimately $\pi_{2r + 1}(S^3)$ has no $r$-torsion, from which finally we conclude that $\Sigma x = 0$ for any $p$-torsion class $x \in \pi_{2r}(S^3)$.
\end{solution}
\begin{proof}[Proof (sketch) of the bonus question]
	We have a commutative diagram
	\begin{equation*}
		\begin{tikzcd}
			\pi_4(S^3)
					\ar[r, "\Sigma", "\isom"']
					\ar[d, swap, "\eta_*", "\isom"']
				& \pi_5(S^4)
					\ar[d, "(\Sigma \eta)_*"]
			\\
			\pi_4(S^2)
					\ar[r, "\Sigma"]
				& \pi_5(S^3)
		\end{tikzcd}
	\end{equation*}
	where the top map is an isomorphism since $\pi_4(S^3)$ is already in the stable range (in the sense of the Freudenthal suspension theorem).
	The fact that $\eta_*$ is an isomorphism is nothing new (this is immediate from the long exact sequence of homotopy groups of the first Hopf fibration), and we know all groups involved to be isomorphic to $\Zn{2}$, so the top-then-right composite takes $[\Sigma \eta] \in \pi_4(S^3)$ to $[\Sigma^2 \eta \circ \Sigma \eta] \in \pi_5(S^3)$, both of which are nontrivial since suspensions and (suspensions of) squares of Hopf invariant 1 maps are non-trivial (which Prof. Hausmann has promised us to show in the lecture, wherefore we omit the proofs here), showing the claim.
\end{proof}

\begin{exercise}\label{ex:hopfinvariant}
	Let $n \geq 2$ and choose generators $\tilde{x} \in H^n(S^n; \Z)$ and $\tilde{y} \in H^{2n}(S^{2n}; \Z)$.
	Now let $f\colon S^{2n - 1} \to S^n$ be a continuous map with mapping cone $C(f)$.
	We obtain generators $x \in H^n(C(f); \Z)$ and $y \in H^{2n}(C(f); \Z)$ from the isomorphisms $H(C(f); \Z) \xto{\isom} H^n(S^n; \Z)$ and $H^{2n}(S^{2n}; \Z) \xto{\isom} H^{2n}(C(f); \Z)$ induced by the inclusion of the $n$-skeleton $S^n \incl C(f)$ and the projection to the $2n$-cell $C(f) \to S^{2n}$, respectively.
	We then define the Hopf invariant $h(f)$ to be the unique integer such that $x^2 = h(f) y$.
	\begin{enumerate}
		\item Show that $h(f) = 0$ if $n$ is odd.
		\item Show that the Hopf invariant gives rise to a group homomorphism $h\colon \pi_{2n - 1} S^n \to \Z$.
		\item Show that if $g\colon S^n \to S^n$ has degree $d$, then $h(g \circ f) = d^2 h(f)$.
		\item Consider the composite
			\begin{equation*}
				\alpha\colon S^{2n - 1} \to S^n \vee S^n \to S^n
			\end{equation*}
			where the first map is the attaching map of the $2n$-cell of $S^n \times S^n$.
			Show that $h(\alpha)$ is $\pm 2$.
		\item The space $\Omega S^{n + 1}$ is homotopy equivalent to a CW-complex with one cell in every dimension a multiple of $n$ (you do not have to prove this).
			What is the Hopf invariant of the attaching map of the $2n$-cell, up to a sign?
	\end{enumerate}
\end{exercise}
\begin{solution}
	\leavevmode
	\begin{enumerate}
		\item If $n$ is odd, then $x \smile x = -(x \smile x)$ by graded commutativity of the cup product, which is to say that $2 x^2 = 0$, i.e. that $x^2$ is 2-torsion. 
			But $H^{2n}(C(f)) \isom \Z$ does not have any nontrivial 2-torsion, so $x^2 = 0$ and therefore $h(f) = 0$.
		\item Let $[f], [g] \in \pi_{2n - 1}(S^n)$ be two classes represented by maps $f, g\colon S^{2n - 1} \to S^n$, respectively.
			We will compare the cofibres $C(f + g)$ and $C(f \vee g)$ where $f + g\colon S^{2n - 1} \xto{c} S^{2n - 1} \vee S^{2n - 1} \xto{f \vee g} S^n$ is the composite of the equator pinching map $c$ with the pointed sum $f \vee g\colon S^{2n - 1} \vee S^{2n - 1} \to S^n$. 
			Note that $C(f \vee g)$ has a cell structure with one $n$-cell and two $2n$-cells attached via $f$ and $g$, respectively, all meeting at the 0-cell.
			We obtain a map $c'\colon C(f + g) \to C(f \vee g)$ by extending $c$ over the $2n$-cell of $C(f + g)$, collapsing an equator.
			On the level of homology, the induced map $\Z \isom H^{2n}(C(f + g)) \xto{c'_*} H^{2n}(C(f \vee g)) \isom \Z^2$ is given by $y \mapsto (y', y')$\footnote{We write $y$ both for the class in cohomology and its dual. In $C(f \vee g)$, we denote the respective classes by $x'$ and $y'$, where $y'$ in particular is obtained by pulling back $\tilde{y}$ along the summand inclusions $S_i^{2n} \incl S_1^{2n} \vee S_2^{2n}$, $i = 1, 2$ after collapsing the $n$-skeleton. We also allow ourselves to specify maps on generators only.} (this can be seen either by collapsing the $n$-skeleta of both spaces (the act of which does not affect $H^{2n}({{-}})$), the residual map of which operation then simply being the equatorial collapse map $c\colon S^{2n} \to S^{2n} \vee S^{2n}$,\footnote{which one could further study by collapsing either summand in the codomain and deducing the sum by a Mayer-Vietoris argument if one has never done this before} or by considering the cellular chain complex, or\textellipsis{}), so dualizing yields that the map $H^{2n}(C(f \vee g)) \xto{c'^*} H^{2n}(C(f + g))$ on cohomology is given by $(y', 0) \mapsto y$ and $(0, y') \mapsto y$.

			The map $c'$ induces the isomorphism $H^n(C(f \vee g)) \xto{\isom} H^n(C(f + g))$, $x \mapsto x$ since it extends the identity on $S^n$ and the first cells of dimension $> n$ have dimension $2n$ in both spaces.
			For the ring structure, we have that $x'^2 = h(f) (y', 0) + h(g) (0, y')$ in $C(f \vee g)$ (which can be seen either via $C(f \vee g) \isom C(f) \sqcup C(g) / {{\sim}}$ where $\sim$ identifies the $n$-skeleta and noting that this corresponds to \enquote{identifying} the two classes $x$ in degree $n$, or the other way around by considering the Mayer-Vietoris sequence for $C(f \vee g) = C(f) \cup C(g)$ and noting that all boundary maps must be zero for degree reasons), so altogether we have that $x^2 = c'^*(x'^2) = c'^*((h(f)(y', 0) + h(g)(0, y')) = (h(f) + h(g)) y$ in $H^*(C(f + g))$ which is the claim.
		\item Let $\tel\big(S^{2n - 1} \xto{f} S^n \xto{g} S^n\big)$ be the unreduced mapping telescope of $g \circ f$ and let $X \coloneq \tel\big(S^{2n - 1} \xto{f} S^n \xto{g} S^n\big) / (S^{2n - 1} \times \{0\})$ be its \enquote{cone / cofibre}.
			Note that $X$ is homotopy equivalent to $C(g \circ f)$ since $M_g \incl X$ is a closed cofibration (where $M_g \isom \tel\big(S^n \xto{g} S^n\big)$ is the mapping cylinder) and $M_g$ deformation retracts onto $S^n \times \{2\} \subset X$.\footnote{We treat $M_g$ as a subspace of the telescope here, so that its time coordinate runs from $1$ to $2$, not the customary $0$ to $1$.}
			Looking at the long exact sequence in cohomology for the pair $(M_g, S^n \times \{1\})$, we get an excerpt
			\begin{equation*}
				\begin{tikzcd}
					0
							\ar[r]
						& \underbrace{H^n(M_g)}_{\isom \Z}
							\ar[r]
						& \underbrace{H^n(S^n \times \{1\})}_{\isom \Z}
							\ar[r]
						& \underbrace{\tilde{H}^{n + 1}(C(g))}_{\isom \Zn{d}}
							\ar[r]
						& 0
				\end{tikzcd}
			\end{equation*}
			using that $\tilde{H}^*(C(g))$ is a copy of $\Zn{d}$ concentrated in degree $n + 1$ as $g$ is a map of degree $d$ (via cellular cohomology), so we conclude that $S^n \times \{1\} \incl M_g$ induces multiplication by $\pm d$ in cohomology (alternatively, one could see this directly from the cellular (co)chain complex).
			This directly implies that $x^2 = (\pm d x')^2 = d^2 h(f) y$ for $x \in H^n(X)$ representing the \enquote{bottom / righthand}\footnote{Strangely one thinks of the time coordinate in a mapping cylinder as running downwards whereas in a telescope we envision it running from left to right.} copy of $S^n$, $x'$ representing the copy $S^n \times \{1\}$, and $y \in H^{2n}(C(f \circ g))$ the distinguished generator, i.e. that $h(f \circ g) = d^2 h(f)$.
		\item By the Künneth theorem, we have an isomorphism
			\begin{equation*}
				H^*(S^n \times S^n) \isom H^*(S^n) \tensor H^*(S^n)
			\end{equation*}
			of graded rings.
			In particular, $H^{2n}(S^n \times S^n) \isom \Z\{\tilde{x} \tensor \tilde{x}\}$ where $\tilde{x}\in H^n(S^n)$ is the distinguished generator, so $(\tilde{x} \tensor 1) \smile (1 \tensor \tilde{x}) = \tilde{x} \tensor \tilde{x}$ is a multiplicative expression for a generator of $H^{2n}(S^n \times S^n)$ in terms of generators of $H^n(S^n \times S^n)$.
			We now proceed in a similar fashion as we did for the last part:
			Consider the space $X = \tel\big(S^{2n - 1} \xto{f} S^n \vee S^n \xto{\nabla} S^n\big) / (S^{2n - 1} \times \{0\})$ where $f$ is the given attaching map and $\nabla$ the fold map.
			As before, $X$ is homotopy equivalent to $C(\alpha)$ so it is sufficient to treat the multiplicative structure there.
			Restricting our attention to $M_\nabla \subset X$, we find that $(S^n \vee S^n) \times \{1\} \incl M_\nabla$ induces the map $x \mapsto (x, x)$ on $H^n({{-}})$ by a similar argument as before and our understanding of the map $\nabla^*\colon H^*(S^n) \to H^*(S^n \vee S^n)$ (we assume that the fact that this map is the diagonal is known, but it is also not hard to show by any number of arguments).

			We now obtain a map $g\colon S^n \times S^n \to X$ as follows:
			We put $g_n\colon \sk_n(S^n \times S^n) \isom S^n \vee S^n \incl X$ where the last map is the inclusion at time 1 and extend this map over the $2n$-skeleton to form $g$ by noting that the subspace $X_{[0, 1]} \coloneq \{(x, t) \in X \mid t \in [0, 1]\} \subset X$ is a $2n$-cell attached via $f$ to $(S^n \vee S^n) \times \{1\} \subset X$.
			We then have a commutative diagram
			\begin{equation*}
				\begin{tikzcd}[column sep = huge]
					H^{2n}(X)
							\ar[r, "\phi", "\isom"']
						& H^{2n}(S^n \times S^n)
					\\
					H^n(X)
						\ar[r, swap, "{x \mapsto \tilde{x} \tensor 1 + 1 \tensor \tilde{x}}"]
							\ar[u, "\smile"]
						& H^n(S^n \times S^n)
							\ar[u, "\smile"]
				\end{tikzcd}
			\end{equation*}
			since $g$ restricted to the $n$-skeleton is the inclusion and the map is a homeomorphism modulo $(n + 1)$-skeleta\footnote{the $+ 1$ resulting from the fact that $Y \times I$ is a CW-complex of dimension $\dim(Y) + 1$ whenever $Y$ is a CW-complex since $\dim(I) = 1$}.
			Starting with $x \in H^n(X)$ at the bottom left and following the arrows right-up-left, we obtain
			\begin{equation*}
				x^2 = \phi^{-1}((x \tensor 1 + 1 \tensor x)^2) = \phi^{-1}((x \tensor 1)^2 + \pm 2 x \tensor x + (1 \tensor x)^2) = \pm 2 y
			\end{equation*}
			since $\phi$ must take $y \in H^{2n}(X)$ to $\tilde{x} \tensor \tilde{x}$.\footnote{It took me a long time to see where the sign comes in in this argument, but I believe it is the following: One need not worry about, say, the extension of $g_n$ to $g$; this can be made canonical e.g. by observing that we could have constructed $X$ by gluing the top end of $M_\nabla$ to the $n$-skeleton of $S^n \times S^n$. But note that this critically involves a choice: namely the identification of summands of $S^n \vee S^n$ of one space with the other. Swapping the identification will certainly change the sign (at least when $n$ is odd): In fact, $S^n \times S^n$ admits an automorphism extending the swap map $S^n \vee S^n \xto{(a, b) \mapsto (b, a)} S^n \vee S^n$ on the $n$-skeleton which, if $n$ is odd, induces multiplication by $-1$ on $H^{2n}({{-}})$. Distinguishing all the $\tilde{x}$'s, $x$'s, and $y$'s floating around more clearly would probably have helped but oh well.}
			This shows that $h(\alpha) = \pm 2$.
		\item By point 1 the Hopf invariant in question must necessarily be 0 if $n$ is odd, so we can assume that $n$ is even for the following:
			Consider the Serre spectral sequence for the fibre sequence $\Omega S^{n + 1} \to * \to S^{n + 1}$ with $E_2$-page
			\begin{equation*}
				E_2^{*, *} = H^*(S^{n + 1}; H^*(\Omega S^{n + 1})) \isom H^*(S^{n + 1}) \tensor H^*(\Omega S^{n + 1})
			\end{equation*}
			where we note that the cohomology of $S^{n + 1}$ is free and finite for the right hand isomorphism.
			Since $S^{n + 1}$ is $n$-connected, the first group to appear on the $p$-axis is $E_2^{n + 1, 0} \isom \Z$ (via Hurewicz) and similarly we obtain the first group on the $q$-axis, $E_2^{0, n} \isom \Z$.
			Choosing generators $e \in H^{n + 1}(S^{n + 1})$ and $x_n \in H^n(\Omega S^{n + 1})$, we must then have $d_{n + 1}(x_n) = e$ up to sign (which we without loss of generality fix to be positive) since no other possibly nontrivial differential affects either domain or codomain.
			Letting $x_{2n} \in H^{2n}(\Omega S^{n + 1}) \isom \Z$ be a generator, we again must have $d_{n + 1}(x_{2n}) = e x_n$ (up to sign) since no other differential can harm these groups, but $d_{n + 1}(x_n^2) = d_{n + 1}(x_n) x_n + x_n d_{n + 1}(x_n) = 2 e x_n$, so we conclude that $x_n^2 = 2 x_{2n}$.
			But this just says that the Hopf invariant of the attaching map in question is $\pm 2$.
			\qedhere
	\end{enumerate}
\end{solution}

\begin{exercise}
	Let $\pis_* X$ denote the stable homotopy groups of a space $X$.
	Construct a natural long exact sequence of the form 
	\begin{equation*}
		\begin{tikzcd}
			\cdots
					\ar[r]
				& \pis_n A
					\ar[r]
				& \pis_n X
					\ar[r]
				& \pis_n X / A
					\ar[r]
				& \pis_{n - 1} A
					\ar[r]
				& \cdots
		\end{tikzcd}
	\end{equation*}
	for pointed CW-pairs $(X, A)$.
	\begin{hint}
		Let $i\colon A \incl X$ denote the inclusion.
		Construct a natural map $S^1 \wedge \hofib_x(i) \to C(i)$ and show that it is highly connected if $A$ and $X$ are.
	\end{hint}
\end{exercise}
\begin{solution}
	We have a long exact sequence
	\begin{equation*}
		\begin{tikzcd}
			\cdots
					\ar[r]
				& \pi_n(A, x)
					\ar[r]
				& \pi_n(X, x)
					\ar[r]
				& \pi_n(X, A, x)
					\ar[r]
				& \pi_{n - 1}(A, x)
					\ar[r]
				& \cdots
		\end{tikzcd}
	\end{equation*}
	The result now follows after noting that
	\begin{enumerate}
		\item by homotopy excision, $\pi_n(X, A, x) \isom \pi_n(C(i), x) \isom \pi_n(X / A, x)$ if $X$ and $A$ are highly connected, and
		\item after suspending (at least) two times, the sequence ends in $\cdots \to \pi_2(\Sigma^2 A, x) \to \pi_2(\Sigma^2 X) \to \pi_2(\Sigma^2 X, \Sigma^2 A, x) \to 0$.
			\qedhere
	\end{enumerate}
\end{solution}

\begin{exercise}
	Show the following for every $n \geq 0$:
	If there is a weak homotopy equivalence
	\begin{equation*}
		\Sigma^n \RP^\infty \htpyeqv A \vee B
	\end{equation*}
	then $A$ or $B$ is weakly contractible.

	(If you are interested, you can also contemplate the following question for various values of $k$ and $n$: 
	Is $\Sigma^n \RP^k$ weakly equivalent to a wedge sum of two spaces?
	But don't expect to obtain a full answer!)
\end{exercise}
\begin{solution}
	We will use the following theorem below without proof\footnote{Prof. Hausmann also used this in his lecture (without proof), so we feel emboldened. A standard reference for this is \href{https://en.wikipedia.org/wiki/Lucas\%27s_theorem}{Wikipedia}.}:
	\begin{theorem}[Lucas' Theorem]
		Let $p$ be a prime and $m, n \in \N$.
		Then
		\begin{equation*}
			\binom{m}{n} = \prod_{i = 0}^k \binom{m_i}{n_i} \pmod{p}
		\end{equation*}
		where
		\begin{equation*}
			m = m_k p^k + m_{k - 1} p^{k - 1} + \cdots + m_1 p + m_0
		\end{equation*}
		and
		\begin{equation*}
			n = n_k p^k + n_{k - 1} p^{k - 1} + \cdots + n_1 p + n_0
		\end{equation*}
		are the base-$p$ expansions of $m$ and $n$, respectively.
	\end{theorem}
	As a particular consequence, note that $\binom{m}{n} \equiv 1 \pmod{2}$ iff the digits in the binary expansion of $n$ form a subset of those of $m$.

	Let now $\Sq = \sum_{i = 0}^\infty \Sq^i$ be the total square and $\iota \in H^1(\RP^\infty; \F_2)$ the fundamental class.
	We then have
	\begin{equation*}
		\Sq(\iota) = \Sq^0(\iota) + \Sq^1(\iota) + \sum_{i = 2}^\infty \Sq^i(\iota) = \iota + \iota^2 = \iota (1 + \iota)
	\end{equation*}
	directly from the axioms, so multiplicativity of $\Sq$ implies that
	\begin{equation*}
		\Sq(\iota^k) = \Sq(\iota)^k = \iota^k (1 + \iota)^k = \iota^k \sum_{i = 0}^k \binom{k}{i} \iota^i = \sum_{i = 0}^k \binom{k}{i} \iota^{k + i}
	\end{equation*}
	from which we can simply read off that $\Sq^i(\iota^k) = \binom{k}{i} \iota^{k + i}$.
	With the help of Lucas's theorem, we conclude that for every $k, l > 0$ there are tuples of natural numbers $I, J$ such that $\Sq^I(\iota^k) = \Sq^J(\iota^l) \neq 0$:
	If $k = k_r 2^r + k_{r - 1} 2^{r - 1} + \cdots + k_0$ is the binary expansion of $k$, we can apply $\Sq^{2^{k_{i_\text{min}}}}$ where $i_\text{min}$ the least index such that $k_i \neq 0$ and iterate until the total degree is a power of 2 and then apply $\Sq^{2^l}$'s to equalize degrees as needed, with the theorem guaranteeing that no step of this operation is trivial\footnote{Why does algorithm this work? If the binary expansion of $k$ ends in $\ldots 01 \ldots 10 \ldots 0$ where the length of the stretch of 1s is $m$, then adding $2^{k_{i_\text{min}}}$ yields $\ldots 1 0 \ldots 0$ with the 1 in place of the 0 left of the leftmost 1, so the operation decreases the number of set bits by $m - 1$. In particular, if $m = 1$ this increases the least set index by 1, so after finitely many steps we reach a point at which only one bit is set and the result is a power of 2.}.

	Let now $\Sigma^n \RP^\infty \htpyeqv A \vee B$ be a splitting.
	If $n = 0$, then after replacing $A$ and $B$ by CW-approximations we have $\Zn{2} \isom \pi_1(\RP^\infty, *) \isom \pi_1(A \vee B, *) \isom \pi_1(A, *) \ast \pi_1(B, *)$ (using that CW-complexes are locally contractible for applying van Kampen and deducing the right hand isomorphism), so without loss of generality we have $\pi_1(A, *) = 0$ (as $\Zn{2}$ is abelian (and also too small) and therefore cannot be written as a non-trivial free product).
	But $\RP^\infty$ is a $K(\Zn{2}, 1)$, so $\pi_k(\RP^\infty, *) = \pi_k(A \vee B, *) = 0$ for all $k > 1$ and therefore $\pi_*(A, *) = 0$ (as the inclusion $A \incl A \vee B$ admits a retraction and therefore $\pi_k(A, *) \to \pi_k(A \vee B, *)$ is injective).
	
	If $n > 0$, then note first that $A$ is already weakly contractible if $H^*(A; \F_2) = 0$, for by Hurewicz (which applies since $n > 0$ implies that $\Sigma^n \RP^\infty$ and hence also $A$ and $B$ are simply connected) it is enough that $H_*(A; \Z) = 0$, so since $\tilde{H}_k(A; \Z) \dsum H_k(B; \Z) \isom \tilde{H}_k(A \vee B; \Z) \isom \tilde{H}_k(\Sigma^n \RP^\infty; \Z)$ and this latter group is either $\Zn{2}$ or 0, $\tilde{H}_k(A; \Z)$, too, must be $\Zn{2}$ or 0 in each degree (as well as at most one of $H_k(A; \Z)$, $H_k(B; \Z)$ being nontrivial for any $k$); but in the former case the universal coefficient theorem implies that $H^k(A; \F_2) \isom \F_2$, so $H^*(A; \F_2)$ already detects nontriviality of $H_*(A; \Z)$.
	
	If now both $\tilde{H}^*(A; \F_2)$ and $\tilde{H}^*(B; \F_2)$ are nontrivial, pick any $0 \neq \alpha \in \tilde{H}^*(A; \F_2)$ and $0 \neq \beta \in \tilde{H}^*(B; \F_2)$.
	Both classes are the image of $\iota^{|\alpha| - n}$ and $\iota^{|\beta| - n}$ under the suspension isomorphism, respectively, but our previous work tells us that there are sequences $I, J$ with $\Sq^I \iota^{|\alpha| - n} = \Sq^J \iota^{|\beta| - n} \neq 0$, so stability of the squares implies that $\Sq^I \alpha = \Sq^J \beta \neq 0$.
	But $\Sq^I \alpha$ lies in \emph{either} $H^*(A; \F_2)$ or $H^*(B; \F_2)$ which is absurd since no $\Sq^i$ can connect classes from $H^*(A; \F_2)$ with classes from $H^*(B; \F_2)$ or vice-versa, contradiction!
	Thus, either $A$ or $B$ must be weakly contractible.
\end{solution}

\begin{exercise}
	Reprove the Freudenthal suspension theorem stated below by induction on $n$, making use of transgressions in Serre spectral sequences.
	\begin{hint}
		Rephrase the problem in terms of connectivity of the map $X \to \Omega \Sigma X$ and apply the Whitehead / Hurewicz theorem.
	\end{hint}
	\begin{theorem}[Freudenthal]\label{thm:freudenthal}
		Suppose that $X$ is an $(n - 1)$-connected space for some $n \geq 2$.
		Then the suspension homomorphism $\Sigma_*\colon \pi_k(X, *) \to \pi_{k + 1}(\Sigma X, *)$ is an isomorphism if $k < 2n - 1$ and an epimorphism if $k = 2n - 1$.
	\end{theorem}
\end{exercise}
\begin{solution}
	Let $\eta\colon \Id_{\Top_*} \Rightarrow \Omega \Sigma$ be the unit natural transformation of the suspension-loop space adjunction which at each based space $X \in \Top_*$ is given by applying the map $X \to \Omega \Sigma X$ adjoint to $\Sigma X \xto{\Sigma \id_X} \Sigma X$.
	Concretely, this means $\eta$ is given by $\eta(x) = (t \mapsto (t, x))$.
	Passing to based homotopy classes we obtain a commutative diagram
	\begin{equation*}
		\begin{tikzcd}
			{[Y, X]_*}
					\ar[r, "\Sigma_*"]
					\ar[dr, swap, "\eta_*"]
				& {[\Sigma Y, \Sigma X]_*}
					\ar[d, "\isom"]
			\\
				& {[Y, \Omega \Sigma X]_*}
		\end{tikzcd}
	\end{equation*}
	which after specializing to $Y = S^n$ yields the commutative triangle
	\begin{equation*}
		\begin{tikzcd}
			\pi_n(X, *) 
					\ar[r, "\Sigma_*"]
					\ar[dr, swap, "\eta_*"]
				& \pi_{n + 1}(\Sigma X, *)
					\ar[d, "\isom"]
			\\
				& \pi_n(\Omega \Sigma X, *)
		\end{tikzcd}
	\end{equation*}
	so $\eta_*$ \enquote{embodies}\footnote{i.e. it is a continuous map inducing an algebraic map that does not a priori arise this way} $\Sigma_*$.
	We also note that $\eta$ is an embedding: 
	Clearly it is injective and the map $\rho = \pr_X \circ \ev_{1 / 2}$ is a continuous inverse on $\img(\eta)$.

	Note now that the map $p = \ev_1\colon P \Sigma X \to \Sigma X$, $\gamma \mapsto \gamma(1)$ admits a section, namely 
	\begin{align*}
		\bar{s}\colon \Sigma X &\to P \Sigma X \\
		(t, x) &\mapsto (t' \mapsto (t t', x))
	\end{align*}
	Moreover, this section arises by factoring the map
	\begin{align*}
		s\colon C X &\to P \Sigma X \\
		(t, x) &\mapsto (t' \mapsto (t t', x))
	\end{align*}
	over the quotient $\Sigma X \isom C X / X$.
	Now $s|_X \colon X \to P \Sigma X$ is the map $s|_X(x) = (t' \mapsto (t', x))$\footnote{For this to work out nicely we deviate from the norm and assume that $C X = X \times I / (X \times \{0\}) \cup (\{x\} \times I)$ instead of collapsing the copy of $X$ at height 1 as usual.}, i.e. $s|_X = \eta$, so the map of pairs $s\colon (C X, X) \to (P \Sigma X, \Omega \Sigma X)$ induces a map of long exact sequences
	\begin{equation}\label{diag:pairsequence}
		\begin{tikzcd}[column sep = small]
			\cdots
					\ar[r]
				& H_k(CX)
					\ar[r]
					\ar[d, "s_*"]
				& H_k(CX, X)
					\ar[r, "\del", "\isom"']
					\ar[d, "s_*"]
				& H_{k - 1}(X)
					\ar[r]
					\ar[d, "\eta_*"]
				& H_{k - 1}(CX)
					\ar[r]
					\ar[d, "s_*"]
				& \cdots
			\\
			\cdots
					\ar[r]
				& H_k(P \Sigma X)
					\ar[r]
				& H_k(P \Sigma X, \Omega \Sigma X)
					\ar[r, "\del", "\isom"']
				& H_{k - 1}(\Omega \Sigma X)
					\ar[r]
				& H_{k - 1}(P \Sigma X)
					\ar[r]
				& \cdots
		\end{tikzcd}
	\end{equation}
	where the terms at the left and right are trivial since $C X$ and $P \Sigma X$ are contractible.
	Moreover, we obtain a commutative triangle
	\begin{equation*}
		\begin{tikzcd}[column sep = small]
			H_*(C X, X)
					\ar[rr, "s_*"]
					\ar[dr, swap, "q_*"]
				& & H_*(P \Sigma X, \Omega \Sigma X)
					\ar[dl, "p_*"]
			\\
				& H_*(\Sigma X, *)
		\end{tikzcd}
	\end{equation*}
	where $q\colon C X \to \Sigma X$ is the quotient map since $q = p \circ s$ holds outright on the level of spaces, so since $\del\colon H_*(CX, X) \xto{\isom} H_{* - 1}(X)$ is the composite
	\begin{equation*}
		\begin{tikzcd}
			\del\colon H_*(CX, X)
					\ar[r, "q_*"]
				& H_*(\Sigma X, *)
					\ar[r, "\sigma^{-1}", "\isom"']
				& H_{* - 1}(X)
		\end{tikzcd}
	\end{equation*}
	where $\sigma\colon H_*(X) \to H_{* + 1}(\Sigma X)$ is the suspension isomorphism, we conclude that if $p_*$ is an isomorphism then so is $s_*$, and therefore so is $\eta_*$ (by virtue of the middle square in diagram \eqref{diag:pairsequence}).

	What's all of this good for?
	Recall that in the lecture\footnote{This is a blatant lie because we have only treated transgressions in cohomology. However, it is reasonable to expect that the \enquote{dual} description holds in homology, and the treatment on page 540 and following of Hatcher's Spectral Sequences confirms this.} we have proved that the transgressions in the Serre spectral sequence are (very) roughly given by diagrams of the form
	\begin{equation}\label{diag:transdef}
		\begin{tikzcd}
				& H_n(P \Sigma X, \Omega \Sigma X) 
					\ar[r, "\del", "\isom"']
					\ar[d, "p_*"]
				& H_{n - 1}(\Omega \Sigma X)
			\\
			H_n(\Sigma X) 
					\ar[r, "\isom"]
					\ar[urr, bend right, swap, "{\text{\textquotedblleft}d^n_{0, n}\text{\textquotedblright}}"]
				& H_n(\Sigma X, *)
		\end{tikzcd}
	\end{equation}
	up to restricting to a subgroup of the domain, projecting to a quotient of the codomain, and accounting (in a similar fashion) for the map $p_*$ running in the wrong direction, none of which will be relevant to our use case here.

	Assume then that $X$ is $(n - 1)$-connected and consider the homological Serre spectral sequence for the fibre sequence $\Omega \Sigma X \to * \to \Sigma X$.
	By the Hurewicz theorem, $\Sigma X$ is $n$-connected and $\Omega \Sigma X$ is $(n - 1)$-connected, so the first possibly nontrivial groups on the axes in positive degree are $E^2_{n + 1, 0}$ and $E^2_{0, n}$.
	We note that since this implies that the first (in the sense of least total degree) possibly non-trivial group off the axes is $E^2_{n + 1, n}$, the differentials $d^{n + 1 + k}\colon H_{n + 1 + k}(\Sigma X) \isom E^{n + 1 + k}_{n + 1 + k, 0} \to E^{n + 1 + k}_{0, n + k} \isom H_{n + k}(\Omega \Sigma X)$ are the only possibly nontrivial ones affecting their domains and codomains for $k = 0, \ldots n - 1$, which by contractibility of the total space implies that they must all be isomorphisms.
	Back in diagram \eqref{diag:transdef}, this implies that the big arrow is defined outright and an isomorphism, so $p_*$ must be an isomorphism as well.
	By our previous discussion, this entails that $\eta_*$ is an isomorphism in all degrees up to $2n - 1$.
	As a statement about the pair\footnote{Here we use that $\eta$ is an embedding. Strictly speaking one does not need this since one could work just as well with a mapping cylinder, but it's nicer this way :)} $(\Omega \Sigma X, X)$ this is to say that $H_k(\Omega \Sigma X, X) = 0$ for all $k < 2n$, so by the relative Hurewicz theorem we have that $\pi_k(\Omega \Sigma X, X) = 0$ for all $k < 2n$ which entails that $\eta_*\colon \pi_k(X) \to \pi_k(\Omega \Sigma X)$ is an isomorphism up to degree $k = 2n - 2$ and surjective for $k = 2n - 1$ as desired (via the long exact sequence of homotopy groups for the given pair).
\end{solution}

\begin{exercise}\label{ex:euclideanmetric}
	Using a partition of unity, show that any vector bundle over a paracompact base space can be given a Euclidean metric.
\end{exercise}
\begin{solution}
	Let $p\colon E \to B$ be an $n$-dimensional $\R$-vector bundle with paracompact base $B$ and cover $B$ by trivialization neighborhoods (i.e. open neighborhoods of points in $B$ over which $p$ is trivializable) $U_i$, $i \in I$ with $I$ some index set.
	Pick a locally finite refinement $V_j$, $j \in J$ ($J$ some index set) of $\{U_i\}_{i \in I}$ (i.e. $\{V_j\}_{j \in J}$ is a cover of $B$ by open neighborhoods such that each $V_j$ is contained in some $U_i$ and each point of $B$ is contained in but finitely many $V_j$).
	Finally, pick a partition of unity $\{\varphi_j\colon B \to \R\}_{j \in J}$ subordinate to $\{V_j\}_{j \in J}$ (i.e. the $\varphi_j$ are continuous maps with image $[0, 1]$ such that $\supp \varphi_j \subseteq V_j$ and $\sum_{j \in J} \varphi_j(x) = 1$ for all $x \in B$, with the local finiteness condition of $\{V_j\}_{j \in J}$ ensuring that this sum is finite and therefore defined).

	Let $\mu(x) = x_1^2 + \ldots + x_n^2$ be the standard quadratic form on $\R^n$.
	For each $j \in J$, let $h_j\colon p^{-1}(V_j) \xto{\isom} V_j \times \R^n$ be a trivialization (since $V_j \subseteq U_i$ for some $i \in I$, such a trivialization exists) and denote by $\mu_j$ the (fibrewise) pullback of $\mu$ along $h_j$ (i.e. $\mu_j(e) = \mu(\pr_{\R^n}(h_j(e)))$ for all $e \in E$).
	We now piece this together using the $\varphi_j$ to get a metric on all of $E$: namely, define $\mu_E\colon E \to \R$ via $\mu(e) = \sum_{j \in J} \varphi_j(p(e)) \mu_j(e)$.
	This is well-defined by the definitional properties of partitions of unity, noting in particular that $\supp \varphi_j \subseteq V_j$ (which entails that $\varphi_j|_{\compl{V_j}} = 0$) implies that we can make sense of the expression under the sum even if $\mu_j$ is not defined outside of $V_j$ by simply declaring $\mu_j|_{\compl{V_j}} \coloneq 0$ as well.
	Being a finite sum of continuous functions at each point, $\mu_E$ is continuous, so the only thing left to show is that it is positive-definite (fibrewise).
	But this is immediate: each $\mu_j$ is positive-definite (fibrewise), and this certainly does not change after taking a non-negatively weighted sum of them.
\end{solution}

\begin{exercise}
	\leavevmode
	\begin{enumerate}
		\item As defined in the lecture, let $\eta_\R^{1, n + 1}$ denote the tautological bundle over $\RP^n$.
			Prove that the Thom space $\Th(\eta_\R^{1, n + 1})$ is homeomorphic to $\RP^{n + 1}$.
			Show this also in the limit case $n = \infty$, i.e. show that the Thom space of the universal line bundle $\eta_\R^1$ is again homeomorphic to $\RP^\infty$.
		\item Use the Thom isomorphism and the previous part to give an alternative proof that the ring $H^*(\RP^\infty; \F_2)$ is polynomial on a class in degree 1.
	\end{enumerate}
\end{exercise}
\begin{solution}
	\leavevmode
	\begin{enumerate}
		\item Let $D^{n + 1}_0 \coloneq D^{n + 1} \setminus \{0\}$ be the punctured disk and $\varphi\colon D^{n + 1}_0 \to S^n \times [0, 1)$ be the map $\varphi(x) \coloneq (x / |x|, 1 - |x|)$ with $|x|$ the standard euclidean norm.
			Clearly $\varphi$ is continuous and has a continuous inverse $S^n \times [0, 1) \to D^{n + 1}_0$ given by $(y, t) \mapsto (1 - t) y$, so it is a homeomorphism.
			Next, note that $E\big(\eta_\R^{1, n + 1}\big) \isom S^n \times_{C_2} \R \isom S^n \times_{C_2} (-1, 1)$ where $C_2$ acts on $S^n$ antipodally and on $\R$ and $(-1, 1)$ by sign, which follows from the construction of the map $\{\text{2-sheeted coverings } p\colon X \to \RP^n\} \to \Vect_\R^1(\RP^n)$ in exercise 9.1 together with the observation that $\eta_\R^{1, n + 1}$ corresponds to the unique connected 2-sheeted covering $S^n \to \RP^n$ since it is non-trivial and metrizable.
			We therefore obtain a map 
			\begin{equation*}
				\tilde{\rho}\colon D^{n + 1}_0 \xto[\isom]{\varphi} S^n \times [0, 1) \incl S^n \times (-1, 1) \surj S^n \times_{C_2} (-1, 1)
			\end{equation*}
			which is proper since all its constituent maps are proper (noting in particular that the quotient projection $S^n \times (-1, 1) \to S^n \times_{C_2} (-1, 1)$ is proper since it is closed and has compact fibres, and $S^n \times_{C_2} (-1, 1)$ is locally compact and Hausdorff), so we further obtain a map 
			\begin{equation*}
				\rho\colon D^{n + 1} \isom (D^{n + 1}_0)^+ \xto{\tilde{\rho}^+} (S^n \times_{C_2} (-1, 1))^+ \isom E\big(\eta_\R^{1, n + 1}\big)^+ \isom \Th\big(\eta_\R^{1, n + 1}\big)
			\end{equation*}
			using exercise 2 below.

			This allows us to form a pushout diagram
			\begin{equation*}
				\begin{tikzcd}
					S^n
							\ar[r, "p"]
							\ar[d]
						& \RP^n
							\ar[d]
							\ar[ddr, bend left = 20, "\bar{s}_0"]
					\\
					D^{n + 1}
							\ar[r]
							\ar[drr, bend right = 20, swap, "\rho"]
						& \RP^{n + 1}
							\ar[ul, phantom, "\ulcorner" {pos = -.1}]
							\ar[dr, dashed, swap, "\kappa", "\exists"']
					\\[-1em]
						& &[-1.5em] \Th\big(\eta_\R^{1, n + 1}\big)
				\end{tikzcd}
			\end{equation*}
			where $p$ is the covering map and $\bar{s}_0$ is the zero section $s_0$ of $\eta_\R^{1, n + 1}$ composed with the inclusion into $\mathring{D}\big(E\big(\eta_\R^{1, n + 1}\big)\big) \subset \Th\big(\eta_\R^{1, n + 1}\big)$.
			Since $\rho|_{S^n}$ is the inclusion of $S^n / C_2 \isom \RP^n$ into the Thom space via $\bar{s}_0$, the diagram commutes and the indicated map $\kappa$ exists.
			Moreover, it is bijective: 
			By commutativity of the diagram, it agrees with $\bar{s}_0$ on $\RP^n \subset \RP^{n + 1}$ which is injective, and any point in $(S^n \times_{C_2} (-1, 1))^+ \isom \Th\big(\eta_\R^{1, n + 1}\big)$ not in the image of $\bar{s}_0$ has a unique preimage under $\rho$ by construction\footnote{Festive spirits prevent me from delving into details here :)} of $\tilde{\rho}$.
			But since all spaces involved are compact, $\kappa$ must be a homeomorphism, concluding the finite case of the exercise.

			To pass to the infinite case, note that there are inclusions $E\big(\eta_\R^{1, n + 1}\big) \incl E\big(\eta_\R^{1, n + 2}\big)$ induced by the inclusion $\RP^n \incl \RP^{n + 1}$ (under the observation that the lines constituting $\RP^n \subset \RP^{n + 1}$ all lie in an $(n + 1)$-dimensional hyperplane of $\R^{n + 2}$ under the standard embedding).
			Taking colimits, we see that $E\big(\eta_\R^1\big) \isom \colim_{k > 0} E\big(\eta_\R^{1, n + 1}\big)$, so since $\RP^\infty \isom \colim_{k > 0} \RP^k$, the result follows by noting naturality of the Thom space (e.g. via the naturality of the one-point compactification with regards to proper maps).
		\item The Thom isomorphism in this case says (after squinting slightly) that the map $H^*(\RP^n; \F_2) \xto{u \smile {{-}}} \tilde{H}^{* + 1}(\RP^{n + 1}; \F_2)$ is an isomorphism where $u \in H^1(\RP^{n + 1}; \F_2)$ is the unique nonzero class, using that $\RP^{n + 1} \isom E\big(\eta_\R^{1, n + 1}\big)$ by the previous part.
			Assuming by induction that $H^*(\RP^n; \F_2) \isom \F_2[u] / (u^{n + 1})$ where we identify the classes $u \in H^1(\RP^k; \F_2)$ by the Thom isomorphism $u \smile 1 = u$, and noting that the case $n = 1$ is clear since $\RP^1 \isom S^1$, we find that $H^*(\RP^{n + 1}; \F_2) \isom \F_2[u] / (u^{n + 2})$ since the Thom isomorphism preserves the multiplicative relations between the generators in different degrees\footnote{With the identifications it / we make, it is basically impossible to write this down in a meaningful formula because this statement is essentially just saying that $u \smile u^{k - 1} = u^k$ for all $k$ with all the heavy lifting behind the scenes.}, raising the total degree by 1, and $H^0(\RP^{n + 1}; \F_2) \isom \F_2$ since $\RP^{n + 1}$ is path-connected.

			Passing to the limit, we obtain that $H^*(\RP^\infty; \F_2) \isom \F_2[u]$ as claimed.
			\qedhere
	\end{enumerate}
\end{solution}

\begin{exercise}
	Let $\xi$ be a vector bundle over a compact base space $B$ (recall that for us \enquote{compact} in particular means that $B$ is a Hausdorff space).
	\begin{enumerate}
		\item Show that the total space $E = E(\xi)$ is locally compact and Hausdorff and hence its one-point compactification $E^+$ is defined.
		\item Prove that the Thom space $\Th(\xi)$ is homeomorphic to $E^+$.
	\end{enumerate}
\end{exercise}
\begin{solution}
	\leavevmode
	\begin{enumerate}
		\item Both $\R$ and $B$ are locally compact and Hausdorff, and since these properties are both inherited by open subspaces and local in nature, we conclude that $U \times \R^n$ is so as well for all $U \subseteq B$ open, and therefore that $E$ has these properties since it is locally of this form.
			Any non-compact locally compact Hausdorff space has a one-point compactification, so we see that $E$ has one after making the remaining observation that it is noncompact:
			
			Find a trivialization covering $\{U_i \subseteq B\}_{i \in I}$ such that there exists a point $b$ contained in only a single $U_i$ (e.g. by choosing one $U_i$ containing the point and replacing all other $U_j$ by $U_j \cap (B \setminus \{b\})$, for any trivialization covering).
			Now cover $E$ via pulling back $\{U_j\} \times \symcal{U}_{\R^n}$ along some trivialization over $U_j$ for all $j$ where $\symcal{U}_{\R^n}$ is any open cover of $\R^n$.
			If the cover so defined has a finite subcover, then in particular it contains only a finite subset of $\{U_i\} \times \symcal{U}_{\R^n}$ (in a trivialization over $U_i$), and by construction restricting to the fibre over $b$ yields a finite subcover of $\symcal{U}_{\R^n}$ of $\R^n$.
			But as the choice of $\symcal{U}_{\R^n}$ was arbitrary, this implies that $\R^n$ is compact which is absurd.
		\item $B$ is in particular paracompact, so let $\lVert\plhold\rVert$ be a choice of euclidean metric for $\xi$ (with respect to which we implicitly work).
			We abbreviate $D \coloneq D(E)$, $S \coloneq S(E)$ such that $\Th(\xi) = D / S$ and define $\mathring{D} \coloneq D \setminus S$.

			Before we begin, let us prove the following short-but-useful lemma:
			\begin{lemma}
				Let $X$ be a compact space and $\symcal{C} \coloneq \{C_i \subseteq X\}_{i \in I}$ be a descending family of closed sets where $I$ is totally ordered.
				If $U \subseteq X$ is an open neighborhood of $\bigcap_{i \in I} C_i$, then there exists $i_0 \in I$ such that $C_i \subseteq U$ for all $i > i_0$.
			\end{lemma}
			\begin{smallproof}
				The family $\{X \setminus C_i \mid i \in I\} \cup \{U\}$ is an open cover of $X$, so we find a finite subcover $\{X \setminus C_{i_1}, \ldots, X \setminus C_{i_n}, U\}$, and since $\symcal{C}$ is descending, this implies that $C_i \subseteq U$ for all $i > i_0 \coloneq \max\{i_1, \ldots, i_n\}$.
			\end{smallproof}
			This implies that $D$ and $S$ are compact:
			The collection $\symcal{C} = \{C_i \subseteq B\}_{i \in I}$ of closed neighborhoods contained in a trivialization neighborhood $U$ has the property that the interiors $\mathring{C}_i$ cover $B$ since every point is the intersection of all closed neighborhoods containing it, which after finding a descending subsequence together with the lemma implies that each point has a closed neighborhood contained in a trivialization neighborhood.
			By compactness, this implies that there are finitely many $C_{i_1}, \ldots, C_{i_n}$ which together cover $B$, and since each of them is contained in a trivialization neighborhood we find that $D = \bigcup_{j = 1}^n C_{i_j} \times D(\R^n)$ up to composing with trivialization homeomorphisms in each summand, so $D$ is a finite union of compact subsets and therefore compact, and similarly for $S$.

			Let now $\mathring{D}^+ = \mathring{D} \cup \{\infty\}$ be the one-point compactification and define a bijection $\varphi\colon \Th(\xi) \to \mathring{D}^+$ by $\varphi(d) \coloneq d$ for all $d \in D$ and $\varphi(\zeta) \coloneq \infty$ where $\zeta \in D / S$ is the image of $S$ under the quotient map.
			The restriction $\varphi|_{\mathring{D}}\colon \mathring{D} \xto{\id} \mathring{D}$ is continuous since the inclusion $\mathring{D} \incl \mathring{D}^+$ is an embedding, and by definition of $\mathring{D}^+$ all open neighborhoods of $\infty$ are of the form $(\mathring{D} \setminus C) \cup \{\infty\}$ where $C \subseteq \mathring{D}$ is compact.
			But this implies that $\varphi$ is continuous at $\zeta$ as well since $\varphi^{-1}((\mathring{D} \setminus C) \cup \{\infty\}) = \mathring{D} \setminus C \cup \{\zeta\}$ which is open in $\Th(\xi)$ as $q^{-1}((\mathring{D} \setminus C) \cup \{\zeta\}) = D \setminus C$ is open where $q\colon D \surj \Th(\xi)$ is the quotient projection.

			In other words, $\varphi$ is a continuous bijection, so by the compact-Hausdorff lemma it is a homeomorphism.

			Finally, we note that there is a homeomorphism $\psi\colon E \to \mathring{D}$ given by $\psi(e) \coloneq \frac{e}{1 + \lVert e\rVert}$ with inverse $\psi^{-1}(d) = \frac{d}{1 - \lVert d\rVert}$, both of which are obviously continuous, so that we obtain a homeomorphism $\Th(\xi) \xto{\varphi} \mathring{D}^+ \xto{\psi^+} E^+$ using functoriality of ${{-}}^+$ in topological spaces with proper continuous maps.
			\qedhere
	\end{enumerate}
\end{solution}

\subsection{Quick Questions}\label{sect:quickquest}
This section contains two sets of 20 questions each designed to be answered from the top of your head, together with example solutions.

\begin{questions*}
	Decide whether the following statements are true or false:
	\begin{enumerate}
		\item The equivalence $K(A, n) \htpyeqv \Omega K(A, n + 1)$ is adjoint to an equivalence $\Sigma K(A, n) \htpyeqv K(A, n + 1)$.
		\item The Steenrod algebra $\symcal{A}$ of stable cohomology operations in $\F_2$-cohomology is a polynomial algebra over $\F_2$.
		\item If $\alpha$ is an element of degree $i$ in the Steenrod algebra $\symcal{A}$ and $x \in H^n(X; \F_2)$ is a class of degree $n$ for some space $X$ with $i > n$, then $\alpha(x) = 0$.
		\item Let $V$ and $W$ be vector bundles over a space $X$.
			Assume that $V$ is isomorphic to the trivial bundle $\epsilon^n$ and $W$ is isomorphic to the trivial bundle $\epsilon^m$, then $V \dsum W$ is isomorphic to the trivial bundle $\epsilon^{n + m}$.
			Here, \enquote{isomorphic} is meant in the sense that the two bundles represent the same element in $\Vect_\R^*(X)$, with $*$ the respective dimension.
		\item A subbundle of a trivial vector bundle is always trivializable.
		\item Let $(F \to Y \to X, h)$ be a fibre sequence with $Y$ contractible.
			Then the associated Serre spectral sequence with integral coefficients collapses on the $E_2$-page, i.e. all differentials with $r \geq 2$ are trivial.
		\item Let $X$ be a simply connected finite CW-complex such that $H_i(X; \Q) = 0$ for all $i > 0$.
			Then $\pi_i(X, *)$ is finite for all $i \in \N$ and every basepoint $*$.
		\item The homotopy groups $\pi_m(S^n, *)$ are finite for all $n$ and $m > n$.
		\item Let $(F \to Y \to X, h)$ be a fibre sequence with $F$ path-connected and assume that $p^*\colon H^i(X; \F_2) \to H^i(Y; \F_2)$ is the zero map in all degrees $i > 0$.
			Then no class $x \in E_2^{i, 0}$ with $i > 0$ on the associated Serre spectral sequence is a permanent cycle, i.e. every such $x$ supports a non-trivial differential on some $E_r$-page.
		\item Let $(E_r^{p, q})_{r \in \N}$ be a multiplicative spectral sequence with product $\cdot$ and let $x, y$ be elements on the $E_2$-page.
			If $d_2(x \cdot y) = 0$ then $d_2(x) = 0$ or $d_2(y) = 0$.
		\item The group $\pi_{2024}(S^{2022}, *)$ is generated by $\Sigma^{2020}(\eta) \circ \Sigma^{2021}(\eta)$.
		\item Let $G$ be a group.
			Then the classifying space $BG = K(G, 1)$ is simple iff $G$ is abelian.
		\item If $X$ is a connected CW-complex which only admits trivial real vector bundles, then $X$ is contractible.
		\item Let $B$ be a CW-complex, $p$ a prime, and $E \to B$ an $n$-dimensional real vector bundle over $B$.
			Then the Thom isomorphism theorem gives an isomorphism
			\begin{equation*}
				H^*(B; \F_p) \xto{\isom} H^{* + n}(E, E_0; \F_p)
			\end{equation*}
		\item Stiefel-Whitney classes are multiplicative in the sense that if $\xi$, $\eta$ are vector bundles over a base space $B$, then for all $i$
			\begin{equation*}
				\omega_i(\xi \dsum \eta) = \omega_i(\xi) \cdot \omega_i(\eta)
			\end{equation*}
		\item If $x \in H^2(K(\F_2, 2); \F_2)$ is the generator, then $H^6(K(\F_2, 2); \F_2)$ is of rank two generated by $x^3$ and $\Sq^2(x^2)$.
		\item Let $\xi\colon E \to B$ be a vector bundle of rank $n > 0$ over a paracompact base $B$.
			If there is a nowhere-vanishing section to $\xi$, then $\omega_n(\xi) = 0$.
		\item The class of all finitely generated abelian groups in which every element is 3-torsion forms a Serre class.
		\item Let $(F \to E \to B, h)$ be a fibre sequence with $F$ and $B$ path-connected.
			If $x \in H^{n - 1}(F; \F_2)$ transgresses to $y \in H^n(B; \F_2)$, then $\Sq^i x$ survives to the $E_{n + i}$-page and $d_{n + i}(\Sq^i x)$ is represented by $\Sq^i y$ under the quotient map $H^{n + i}(B; \F_2) \isom E_2^{n + i, 0} \to E_{n + i}^{n + i, 0}$.
		\item Let $(F \to Y \to X, h)$ be a fibre sequence with associated Serre spectral sequence and $A$ an abelian group.
			Then then $E_2$-page is isomorphic in degree $(p, q)$ to $H^p(X; A) \tensor H^q(F; A)$.
	\end{enumerate}
\end{questions*}
\begin{answers}
	\leavevmode
	\begin{enumerate}
		\item \strong{False}: $\Sigma K(A, n)$ is generally not homotopy equivalent to $K(A, n + 1)$, e.g. $\Sigma K(\Z, 1) \isom S^2$ whereas $K(\Z, 2) \isom \CP^\infty$.
		\item \strong{False}: $\symcal{A}$ is not commutative.
		\item \strong{False}: The element $\Sq^2 \Sq^1 \in \symcal{A}$ has degree 3, but for the generator $\iota \in H^1(\RP^\infty; \F_2)$ we compute
			\begin{equation*}
				\Sq^2 \Sq^1 \iota = \Sq^2 \iota^2 = \iota^4 \neq 0
			\end{equation*}
		\item \strong{True}: $V \dsum W$ is trivializable over neighborhoods $U \subseteq X$ over which both $V$ and $W$ are trivializable.
			One such neighborhood is $X$ itself, so $V \dsum W$ is trivial.
		\item \strong{False}: The bundles $\eta_\R^{1, n + 1}$ over $\RP^n$ are by definition subbundles of trivial bundle $\RP^n \times \R^2 \to \RP^n$ but are certainly non-trivial.
		\item \strong{False}: Take for example the fibration $\Omega S^{87} \to P S^{87} \to S^{87}$ with contractible total space.
			Then the first group to appear on the $p$-axis in the associated homological Serre spectral sequence is $E^2_{87, 0} \isom \Z$ while on the $q$-axis we find $E^2_{0, 86} \isom \Z$ as first interesting entry and conclude that $d^{87}\colon E^{87}_{87, 0} \to E^{87}_{0, 86}$ is an isomorphism.
		\item \strong{True}: The homotopy groups of a finite simple CW-complex are finitely generated (by Serre class theory) and any simply connected space is certainly simple.
			The condition $\tilde{H}_*(X; \Q) = 0$ is then the same as saying that $\tilde{H}_*(X; \Z) \in \symcal{C}^\text{tor}$ where $\symcal{C}^\text{tor}$ is the Serre class of torsion abelian groups, so we conclude again by Serre class theory that $\pi_k(X, *)$ is finitely generated and torsion and therefore finite for all $k \in \N$.
		\item \strong{False}: We have computed that $\pi_{2n - 1}(S^n)$ has a $\Z$-summand when $n$ is even.
		\item \strong{True}: Seeing as there are no nontrivial local systems of the form relevant to the Serre spectral sequence with $\F_2$-coefficients, we obtain edge homomorphisms of the form 
			\begin{equation*}
				H^i(X; \F_2) = E_2^{i, 0} \surj E_\infty^{i, 0} \incl H^i(Y; \F_2)
			\end{equation*}
			which agree with $p^*$ and must therefore all be zero.
			But then the projection $E_2^{i, 0} \surj E_3^{i, 0} \surj \cdots \surj E_\infty^{i, 0}$ defining the first map must be trivial, so $E_\infty^{i, 0} = 0$ and no element of $E_2^{i, 0}$ is a permanent cycle.
		\item \strong{False}: Consider the fibre sequence $\RP^\infty \to * \to K(\Zn{2}, 2)$.
			For the $E_2$-page of the associated cohomological Serre spectral sequence we find that $d_2(x) = \iota$ where $x \in H^*(\RP^\infty; \F_2) \isom \F_2[x]$ is the generator and $\iota \in H^2(K(\F_2, 2); \F_2)$ is the fundamental class.
			But then we have $d_2(x^2) = d_2(x) x + x d_2(x) = 2 \iota x = 0$ as we are working over $\F_2$.
		\item \strong{True}: We have computed that $\pi_5(S^3, *) \isom \Zn{2}$ which by the Freudenthal suspension theorem surjects onto the stable group $\pi^{\mathrm{st}}_2 \symbb{S}$.
			Having also seen that $\Sigma^k(\eta \circ \Sigma \eta)$ is homotopically non-trivial for all $k$, we conclude that the stable group is non-trivial and therefore $\pi^{\mathrm{st}}_2 \symbb{S} \isom \Zn{2}$ generated by the suspensions of $\eta \circ \Sigma \eta$.
		\item \strong{True}: Recall that a path-connected space $X$ is simple if $\pi_1(X, *)$ is abelian and acts trivially on $\pi_n(X, *)$ for all $n > 1$ and all $* \in X$.
			As the homotopy groups of a $K(G, 1)$ are concentrated in degree 1 with $\pi_1(K(G, 1), *) \isom G$, the claim follows.
		\item \strong{False}: We claim that $S^3$ does not admit any non-trivial vector bundles.
			To show this, it suffices by universality to compute $\big[S^3, \Gr_n^\R\big]_* = \pi_3\big(\Gr_n^\R\big)$ for all $n \geq 1$.
			To this end, recall first that there are fibre bundles $\Ort(n) \to V_n(\R^\infty) \to \Gr_n^\R$ with total space the infinite Stiefel manifold $V_n(\R^\infty) \coloneq \bigcup_{k \geq 1} V_n(\R^k)$ giving rise to long exact sequences of the form\footnote{We do not track basepoints through all of this since even though $\Ort(n)$ is not path-connected its two path-components are homeomorphic and likewise for other non-connected spaces that appear here.}
			\begin{equation*}
				\begin{tikzcd}[column sep = small]
					\cdots
							\ar[r]
						& \pi_k(\Ort(n))
							\ar[r]
						& \pi_k(V_n(\R^\infty))
							\ar[r]
						& \pi_k\big(\Gr_n^\R\big)
							\ar[r]
						& \pi_{k - 1}(\Ort(n))
							\ar[r]
						& \cdots
				\end{tikzcd}
			\end{equation*}
			We now note that $V_n(\R^k)$ is $(k - n - 1)$-connected since there are fibre bundles $V_{n - 1}(\R^{k - 1}) \to V_n(\R^k) \to S^{k - 1}$ and $V_1(\R^k)$ is homeomorphic to $S^{k - 1}$ (via induction on the long exact sequence of homotopy groups), so taking the limit along the inclusions we obtain that $V_n(\R^\infty)$ is (weakly) contractible.
			Thus, we have that $\pi_k(\Gr_n^\R) \isom \pi_{k - 1}(\Ort(n))$ for all $n, k$ or in other words that $\Gr_n^\R$ is a classifying space for (the Lie group) $\Ort(n)$.
			We now note that $\pi_2(\Ort(1)) = \pi_2(\Ort(2)) = 0$ since $\Ort(1)$ is two points and $\SO(2) \isom S^1$.
			Making now use of the fibration $\Ort(n) \to \Ort(n + 1) \to \Ort(n + 1) / \Ort(n) \htpyeqv S^n$ and its long exact sequence in homotopy groups, we obtain an excerpt
			\begin{equation*}
				\begin{tikzcd}[column sep = small]
					\underbrace{\pi_2(\Ort(2))}_{= 0}
							\ar[r]
						& \pi_2(\Ort(3))
							\ar[r]
						& \underbrace{\pi_2(S^2)}_{\isom \Z}
							\ar[r, "\del"]
						& \underbrace{\pi_1(\Ort(2))}_{\isom \Z}
							\ar[r]
						& \pi_1(\Ort(3))
							\ar[r]
						& \underbrace{\pi_1(S^2)}_{= 0}
				\end{tikzcd}
			\end{equation*}
			which yields that $\pi_2(\Ort(3)) = 0$ upon recalling that $\SU(2)$ is a 2-fold covering of $\SO(3)$ and homeomorphic to $S^3$, which is to say that it is in fact the universal cover of $\SO(3)$ and therefore $\pi_1(\SO(3)) = \pi_1(\Ort(3)) \isom \Zn{2}$ so that the map $\del$ is multiplication by 2.
			Finally, we note that the map $\pi_2(\Ort(3)) \to \pi_2(\Ort(4))$ is surjective and that $\pi_2(\Ort(n)) \to \pi_2(\Ort(n + 1))$ is an isomorphism for all $n > 3$ by the same sequence just considered since $S^n$ is $(n - 1)$-connected, so $\pi_2(\Ort(n)) \isom \pi_3\big(\Gr_n^\R\big) = 0$ for all $n$ as desired\footnote{Alternatively, one can show more generally that $\pi_2(G) = 0$ for any Lie group $G$ using Morse theory.}.
		\item \strong{False}: This is generally only true if the bundle is $\F_p$-orientable.
			For a counterexample, take the Möbius bundle $\gamma_\R^{1, 2}$ over $\RP^1 \isom S^1$.
			Then 
			\begin{equation*}
				H^k\big(E\big(\gamma_\R^{1, 2}\big); \F_p\big) \isom \begin{cases}
					\F_p 	& k = 0, 1 \\
					0 		& \text{else} 
				\end{cases}
			\end{equation*}
			whereas
			\begin{equation*}
				H^*(E, E_0; \F_p) \isom \tilde{H}^*\big(\Th\big(\gamma_\R^{1, 2}\big); \F_p\big) \isom \tilde{H}^*(\RP^2; \F_p) = 0
			\end{equation*}
			for all odd primes $p$, so no such isomorphism exists.
		\item \strong{False}: If this were true, we would have $\omega_1\big(\gamma_\R^{1, 2} \dsum \epsilon^1\big) = \omega_1\big(\gamma_\R^{1, 2}\big) \cdot \omega_1(\epsilon^1) = 0$ whereas multiplicativity of the total class implies that $\omega\big(\gamma_\R^{1, 2} \dsum \epsilon^1\big) = \omega\big(\gamma_\R^{1, 2}\big) = 1 + u$ where $u \in H^1(\RP^1; \F_2)$ is a generator.
		\item \strong{True}: Knowing that $H^*(K(\F_2, 2); \F_2) \isom \F_2\big[\Sq^{2^k} \cdots \Sq^1 x \mid k \geq -1\big]$ from the lecture, we see that $H^6(K(\F_2, 2); \F_2)$ is generated by $x^3$ and $\big(\Sq^1 x\big)^2$, and an application of the Cartan-formula yields
			\begin{align*}
				\Sq^2 x^2 &= \Sq^2 x \smile \Sq^0 x + \Sq^1 x \smile \Sq^1 x + \Sq^0 x \smile \Sq^2 x \\ 
						  &= \big(\Sq^1 x\big)^2
			\end{align*}
			as desired.
		\item \strong{True}: A nowhere vanishing section provides a trivial one-dimensional subbundle of $\xi$, so there is an isomorphism $\xi \isom \bar{\xi} \dsum \epsilon^1$ where $\bar{\xi}$ is the orthogonal complement of said bundle (in some choice of metric).
			The Whitney product formula then says that
			\begin{align*}
				\omega_n(\xi) &= \omega_n(\bar{\xi} \dsum \epsilon^1) \\ 
							  &= \sum_{i = 0}^n \omega_i(\bar{\xi}) \smile \omega_{n - i}(\epsilon^1) \\
							  &= \omega_n(\bar{\xi}) \smile \omega_0(\epsilon^1) \\
							  &= 0
			\end{align*}
			by the dimensionality axiom, using that $\dim(\bar{\xi}) = n - 1$.
		\item \strong{False}: Consider the following short exact sequence:
			\begin{equation*}
				\begin{tikzcd}[column sep = 3.4em]
					0 
							\ar[r]
						& \Zn{3}
							\ar[r, hook, "\cdot 3"]
						& \Zn{9}
							\ar[r, two heads, "(\text{mod } 3)"]
						& \Zn{3}
							\ar[r]
						& 0
				\end{tikzcd}
			\end{equation*}
		\item \strong{True}: This is the statement of the trangression theorem.
		\item \strong{False}: Consider the fibre sequence $\RP^\infty \to 0 \to K(\F_2, 2)$.
			Then 
			\begin{align*}
				E_2^{2, 2} &= H^2\big(K(\F_2, 2); H^2(\RP^\infty; \Z)\big) \\
						   &\isom H^2(K(\F_2, 2); \F_2) \\ 
						   &\isom \Hom(H_2(K(\F_2, 2); \Z), \F_2) \dsum \Ext(H_1(K(\F_2, 2)), \F_2) \\
						   &\isom \Hom(\F_2, \F_2) \dsum \Ext(0, \F_2) \\
						   &\isom \F_2
			\end{align*}
			while $H^2(K(\F_2, 2); \Z) = 0$ by the universal coefficient theorem. 
			\qedhere
	\end{enumerate}
\end{answers}

\begin{questions*}
	Decide whether the following statements are true or false, or answer the question, respectively:
	\begin{enumerate}
		\item Let $\pi\colon S^\infty \to \RP^\infty$ be the familiar covering map.
			Then the map $E(\pi)\colon \R \times_{C_2} S^\infty \to \RP^\infty$ applying $\pi$ to the $S^\infty$-component of a representative is the universal line bundle over paracompact spaces.
		\item Spot the error in this argument:
			Let $\iota \in H^1(\RP^\infty; \F_2)$, $\iota_n \in H^1(\RP^n; \F_2)$ be the respective nontrivial elements.
			The total Stiefel-Whitney classes are $\omega(\gamma_\R^1) = 1 + \iota$ and $\omega(\gamma_\R^{1, n + 1}) = 1 + \iota_n$, the inverse power series being given by $1 + \iota + \iota^2 + \cdots \in H^\pi(\RP^\infty; \F_2)$ in the case of $\RP^\infty$, and analogously for $\RP^n$.
			From this we conclude that $\gamma_\R^1$ and $\gamma_\R^{1, n + 1}$ are not embeddable into trivial bundles.
		\item Recall the proof that $\RP^{2^j}$ cannot be immersed into $\R^{2^{j + 1} - 2}$. 
			Where do you use the immersion property?
		\item The tangent bundle $\tau_{S^n}$ of $S^n$ is trivial for $n \geq 1$.
		\item Two real vector bundles over a paracompact base are isomorphic iff their Stiefel-Whitney classes agree.
		\item Let $\gamma_\R^{1, n + 1}$ denote the tautological bundle over $\RP^n$.
			Compute $\omega(\gamma_\R^{1, n + 1})$ for $n$ a power of 2.
		\item The Grassmannian $\Gr_n^\R$ can be given the structure of a smooth manifold.
		\item Let $\gamma_\C^n$ denote the tautological $n$-dimensional complex vector bundle over the infinite Grassmannian $\Gr_n^\C$ equipped with a euclidean metric.
			Then the associated sphere bundle $S(\gamma_\C^n) \to \Gr_n^\C$ induces a surjection on cohomology $H^*({{-}}; A)$ for every coefficient group $A$.
		\item Let $M$ be a smooth submanifold of $\R^n$. 
			Then the tangent bundle $\tau_M$ of $M$ is trivializable iff the normal bundle $\nu_{M, \R^n}$ is trivializable.
		\item The smooth manifold $\RP^4$ does not allow an immersion into $\R^6$.
		\item Let $(r_1, \ldots, r_6)$ be a sequence of non-negative numbers such that $r_1 + 2 r_2 + \ldots + 6 r_6 = 6$.
			Then the Stiefel-Whitney number $\omega_1^{r_1} \cdots \omega_6^{r_6}[\RP^6]$ is non-zero.
		\item If all the Stiefel-Whitney numbers of a compact smooth manifold $M$ are zero, then the total Stiefel-Whitney class of the tangent bundle $\tau_M$ is equal to 1.
		\item $\RP^3 \times \RP^3$ is cobordant to $\RP^6$.
		\item Euler characteristic modulo 2 is a bordism invariant.
		\item Let $I = (i_1, i_2, \cdots, i_n)$ be a sequence with associated element $\Sq^I = \Sq^{i_1} \cdots \Sq^{i_n}$ in the Steenrod algebra.
			Then if $I$ is not admissible, we have $\Sq^I = 0$.
		\item The integral homology of the homotopy fibre $F$ of the inclusion $\CP^1 \incl \CP^\infty$ is zero in odd degrees and isomorphic to $\Zn{n}$ in degree $2n$, for all $n \in \N$.
		\item Let $X$ be a simply-connected space.
			If there exists $n > 1$ such that $\pi_n(X, *)$ is infinite, then there also exists an $m > 1$ such that $H_m(X; \Z)$ is infinite.
		\item The tangent and normal bundle of the 2-torus $S^1 \times S^1$ viewed as a submanifold of $\R^2 \times \R^2$ are both trivializable.
		\item Assume that $\xi$ is a real, 3-dimensional vector bundle with paracompact base space $X$ satisfying $\omega_3(\xi) \neq 0$ and $H^2(X; \F_2) = 0$.
			Then $\xi$ does not have a 1-dimensional subbundle.
		\item Let $(F \to Y \to X, h)$ be a fibre sequence such that two of $F$, $Y$ and $X$ are compact.
			Then so is the third.
	\end{enumerate}
\end{questions*}
\begin{answers}
	\leavevmode
	\begin{enumerate}
		\item \strong{True:} Pulling back along the inclusion $S^1 = \RP^1 \incl \RP^\infty$ gives the bundle $E\big(\pi|_{\RP^1}\big)\colon \R \times_{C_2} S^1 \to \RP^1$ which is the Möbius bundle and therefore not trivial.
			Thus $E(\pi)$ cannot be the trivial bundle, and since $H^1(\RP^\infty; \F_2) \isom \F_2$ it must be the unique non-trivial bundle $\gamma_\R^1$ over $\Gr_1^\R = \RP^\infty$.
		\item $H^\pi(\RP^n; \F_2) \isom \F_2[\iota_n] / \iota_n^{n + 1}$ is a polynomial ring, so the inverse of $\omega\big(\gamma_\R^{1, n + 1}\big)$ is $1 + \iota_n^2 + \ldots + \iota^n$ which can well be the total Stiefel-Whitney class of an $n$-dimensional bundle over $\RP^n$ (in fact, since $\RP^n$ is compact, we know that $\gamma_\R^{1, n + 1}$ is invertible).
		\item The proof relies on the fact that $\omega\big(\RP^{2^j}\big) = 1 + u + u^{2^j}$ so that $\bar{\omega}\big(\RP^{2^j}\big) = 1 + u + \cdots + u^{2^j - 1}$.
			By the immersion property we have $\dim\big(\tau_{\RP^{2^j}}\big) + \dim(\nu_i) = n$ where the codomain of $i$ is $\R^n$, so the dimensionality axiom of the Stiefel-Whitney classes gives the required lower bound on $n$.
		\item \strong{False:} Parallelizability of $\tau_{S^n}$ is closely linked to the Hopf invariant 1 problem; one can show that $\tau_{S^n}$ is trivial iff there is an element of Hopf invariant one one dimension higher.

			For a concrete counterexample, it is a well-known fact (usually proved in Topology 1) that $\tau_{S^2}$ does not admit a nowhere non-vanishing vector field and can therefore not be trivial.
		\item \strong{False:} Use the previous question together with our computation of $\omega(\tau_{S^n}) = 1$ for a counterexample.
		\item We have seen that $\omega(\gamma_\R^{1, n + 1}) = 1 + u$ independent of $n$.
		\item \strong{False:} $H^*(\Gr_n^\R; \F_2) \isom \F_2[\omega_1, \ldots, \omega_n]$ is a polynomial ring and therefore nontrivial in infinitely many (in fact, all) degrees, but any manifold $M$ of dimension $n$ has $H^k(M; \F_2) = 0$ for $k > n$.
		\item \strong{True:} We have shown that $S(\gamma_\C^n) \htpyeqv \Gr^\C_{n - 1}$ and that the associated projection $\Gr_{n - 1}^\C \to \Gr_n^\C$ induces the map
			\begin{equation*}
				H^*(\Gr_n^\C; \Z) \isom \Z[c_1, \ldots, c_n] \xto[c_i \mapsto c_i\; (i < n)]{c_n \mapsto 0} \Z[c_1, \ldots, c_{n - 1}] \isom H^*(\Gr_{n - 1}^\C; \Z) 
			\end{equation*}
			which is clearly surjective.
		\item \strong{False:} $S^n \subseteq \R^{n + 1}$ has trivial normal line bundle, but we argued earlier that in almost all cases its tangent bundle is not trivializable.
		\item \strong{True:} If it did, then $\tau_{\RP^4}$ would have an inverse of dimension at most 2, but $\omega(\tau_{\RP^4}) = (1 + u)^5 = 1 + u + u^4$ which is obviously neither annihilated by $1 + u$, $1 + u^2$, nor $1 + u + u^2$.
		\item \strong{True:} Recall that $\omega(\tau_{\RP^6}) = 1 + u + u^2 + \cdots + u^6$, so $\omega_1^{r_1}(\tau_{\RP^6}) \cdots \omega_6^{r_6}(\tau_{\RP^6}) = u^6 \neq 0$ for all such sequences.
		\item \strong{False:} We have seen that all Stiefel-Whitney numbers of $\RP^{2n - 1}$ are zero, but for $n = 3$ we compute $\omega\big(\tau_{\RP^5}\big) = (1 + u)^6 = 1 + u^2 + u^4 \neq 1$.
		\item \strong{False:} $\omega(\RP^3 \times \RP^3) = \omega(\RP^3) \times \omega(\RP^3) = 1 \times 1 = 1$, so all Stiefel-Whitney numbers vanish whereas we have just seen that \emph{no} Stiefel-Whitney numbers of $\RP^6$ vanish.
		\item \strong{True:} It is known that $\chi(\del M)$ is even for $M$ any compact manifold with boundary, so if $M \sqcup M'$ is the boundary of a bordism $N$, then $\chi(M \sqcup M') = \chi(M) + \chi(M') \equiv 0 \pmod 2$ and $\chi(M)$ and $\chi(M')$ must have the same parity.
		\item \strong{False:} We know for instance that $\Sq^3 = \Sq^1 \Sq^2$ which is not admissible, but $\Sq^3 \iota^3 = \iota^6 \neq 0$ for $\iota \in H^1(\RP^\infty; \F_2)$ the fundamental class.
		\item \strong{False:} Let $F$ be the homotopy fibre in question.
			From the long exact sequence of homotopy groups and the fact that $S^2 \isom \CP^1 \incl \CP^\infty$ induces an isomorphism on $\pi_2$ we see that $F$ is 2-connected and satisfies $\pi_3(F) \isom \pi_3(S^2) \isom \Z$ since $\CP^\infty$ is a $K(\Z, 2)$.
			Applying the Hurewicz theorem, this yields that $H_3(F; \Z) \isom \pi_3(F) \isom \Z$, contradicting the statement\footnote{Alternatively, one can also easily derive the same information from the associated Serre spectral sequence, or (even simpler) note that since the the cohomology of the base is concentrated in even degrees and free and of finite type, together with the fact that it does not agree with the convergence in almost all degrees, some (in fact infinitely many) cohomology groups of the fibre have to sit in odd degrees.}.
		\item \strong{True:} By Serre class theory applied to the good Serre class of finite abelian groups, we have that $\pi_n(X, *)$ is finite for all $n$ if and only if $H_n(X; \Z)$ is (using that $X$ is simply connected and therefore simple).
			The claim follows.
		\item \strong{True:} The tangent and normal bundles of a product are the product of the tangent and normal bundles of the factors (relative to the product embedding), so since $\tau_{S^1} \isom \nu_{S^1, \R^2} \isom \epsilon$ are trivial the answer follows.
		\item \strong{True:} If $\xi$ splits as $\xi \isom \bar{\xi} \dsum \gamma$ with $\dim \gamma = 1$, then the Whitney product formula together with the dimensionality axiom implies that 
			\begin{equation*}
				0 \neq \omega_3(\xi) = \underbrace{\omega_2(\bar{\xi})}_{\substack{\in H^2(X; \F_2) \\ = 0}} \smile \omega_1(\gamma) = 0
			\end{equation*}
			which is absurd.
		\item \strong{False:} Given any fibre sequence $(F \to Y \to X, h)$ with $Y$ and $X$ compact, we can replace $F$ up to homotopy with any space we want, say $F \times \R$, by precomposing $h$ with a deformation retraction onto $F$.
			This space is non-compact independent of whether $F$ was to start with.
			\qedhere
	\end{enumerate}
\end{answers}
