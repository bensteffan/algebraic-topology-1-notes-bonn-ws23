\lecture{01.12.23}
\section{Vector bundles and characteristic classes}
For the whole section, let $\F$ denote either $\R$ or $\C$.
\begin{definition}
	An $n$-dimensional \strong{$\F$-vector bundle}\index{vector bundle} $\xi$ over a base space $B$ consists of a map $p\colon E(\xi) = E \to B$ together with an $n$-dimensional $\F$-vector space structure on $E_b = p^{-1}(b)$ for each $b \in B$ satisfying the \emph{local triviality condition}:
	For every $b \in B$ there exists an open neighborhood $U \subseteq B$ of $b$ and a homeomorphism
	\begin{equation*}
		h\colon U \times \F^n \xto{\isom} p^{-1}(U)
	\end{equation*}
	such that for each $b \in U$ the map $x \mapsto h(b, x)$ is a vector space isomorphism from $\F^n \isom \{b\} \times \F^n$ to $E_b$.
	In particular, the triangle
	\begin{equation*}
		\begin{tikzcd}[column sep = small]
			U \times \F^n 
			\ar[rr, "h"]
			\ar[dr, swap, "\pr_U"]
				& & p^{-1}(U)
				\ar[dl, "p"]
				\\
				& U
		\end{tikzcd}
	\end{equation*}
	commutes.
\end{definition}
\begin{example}
	Let $B$ be any space and $n \in \N$ arbitrary.
	Then $B \times \F^n \xto{\pr_B} B$ is an $\F$-vector bundle with vector space structure on $\pr_B^{-1}(b) = \{b\} \times \F^n$ carried over from from the identification $\F^n \isom \{b\} \isom \F^n$, $v \mapsto (b, v)$.
	This bundle is called the \strong{trivial bundle}\index{vector bundle!trivial}\index{trivial vector bundle|see {vector bundle!trivial}}.
\end{example}
\begin{example}
	Let $B = \RP^1$ be 1-dimensional real projective space and $E = \{(L, v) \mid L \in \RP^1, v \in L\} \subseteq \RP^1 \times \R^2$ with projection $E \xto{p} \RP^1$, $(L, v) \mapsto L$ (we understand $\RP^1$ here as the space of lines in $\R^2$).
	Then $E_L = p^{-1}(\{L\}) = \{L\} \times L \isom L$ gives $E_L$ an $\R$-vector space structure.
	To see that this defines a 1-dimensional vector bundle over $\RP^1 \isom S^1$, we need to show that it is locally trivial:

	Let $\pi_1, \pi_2\colon \R^2 \to \R$ be the projections and let $U_i \subseteq \RP^1$ be the set of lines $L$ such that $\pi_i|_L\colon L \to \R$ is an isomorphism.
	Then each $U_i$ is open (in fact its complement consists of a single point), $\RP^1 = U_1 \cup U_2$, and
	% TODO picture
	\begin{align*}
		p^{-1}(U_i) &\xto{h} U_i \times \R \\
		(L, v) &\mapsto (L, \pi_i(v))
	\end{align*}
	defines a homeomorphism.
	Each $\pi_i$ is linear, so $h$ is linear on each fibre as required.

	This bundle known as the \strong{Möbius bundle}\index{Möbius bundle}.
	% TODO picture
\end{example}
This example generalizes:
\begin{example}\label{ex:defgrassmannian}
	Let $N, n \geq 1$.
	The \strong{Grassmannian}\index{Grassmannian} $\Gr_n(\F^N)$ is the set of $n$-dimensional subspaces of $\F^N$.
	Let $\Fr_n(\F^N) \subseteq (\F^N)^n$ be the subspace of linearly independent sequences $(v_1, \ldots, v_n)$.
	We obtain a surjective map 
	\begin{align*}
		q\colon \Fr_n(\F^N) &\to \Gr_n(\F^N) \\
		(v_1, \ldots, v_n) &\mapsto \langle v_1, \ldots, v_n\rangle_{\F}
	\end{align*}
	where $\langle \ldots\rangle_{\F}$ denotes the $\F$-span and give $\Gr_n(\F^N)$ the quotient topology for $q$.

	Let $\gamma_{\F}^{n, N} \coloneq \big\{(V, v) \in \Gr_n(\F^N) \times \F^N \mid v \in V\big\}$ and $p\colon \gamma_{\F}^{n, N} \to \Gr_n(\F^N)$ be the projection $(V, v) \mapsto V$.
	The fibre $p^{-1}(V)$ then identifies with $\{V\} \times V \isom V$ and hence form a vector space.

	This is locally trivial:
	For $V \in \Gr_n(\F^N)$ consider the orthogonal projections $\pi_V\colon \F^N \to V$.
	The set $U \coloneq \big\{W \in \Gr_n(\F^N) \mid \pi_V|_W\colon W \to V \text{ is an isomorphism}\big\}$ is open since 
	\begin{equation*}
		q^{-1}(U) = \big\{(v_1, \ldots, v_n) \in (\F^N)^n \mid \pi_V(v_1), \ldots, \pi_V(v_n) \text{ linearly independent}\big\}
	\end{equation*}
	is open in $(\F^N)^n$.
	Again the map $p^{-1}(U) \to U \times V$, $(W, v) \mapsto (W, \pi_V(v))$ is a homeomorphism, yielding a local trivialization.

	The vector bundle $\gamma_{\F}^{n, N}$ is called the \strong{tautological bundle}\index{vector bundle!tautological}\index{tautological vector bundle|see {vector bundle!tautological}} on $\Gr_n(\F^N)$.
\end{example}
As an example, $\Gr_1(\F^N)$ is by definition just $\FP^n$ and $\gamma_{\R}^{1, 2}$ is the Möbius bundle from the previous example.

\subsection{Operations on vector bundles}
\begin{definition}
	\leavevmode
	\begin{enumerate}
		\item A \strong{subbundle}\index{vector bundle!subbundle}\index{vector subbundle|see {vector bundle!subbundle}} of a vector bundle $p\colon E \to B$ is a subspace $E' \subseteq E$ such that each $E' \cap E_b$ is a vector subspace of $E_b$ and $p|_{E'}\colon E' \to B$ is locally trivial.
			\begin{example}
				By definition, each $\gamma_{\F}^{n, N} \to \Gr_n(\F^N)$ is a subbundle of the trivial bundle $\Gr_n(\F^N) \times \F^N \to \Gr_n(\F^N)$.
			\end{example}
		\item A \strong{morphism}\index{vector bundle!morphism of} of vector bundles $E_1 \xto{p_1} B$, $E_2 \xto{p_2} B$ over the same base space is a commutative diagram
			\begin{equation*}
				\begin{tikzcd}[column sep = small]
					E_1 
					\ar[rr, "\varphi"]
					\ar[dr, swap, "p_1"]
						& & E_2
						\ar[dl, "p_2"]
						\\
						& B
				\end{tikzcd}
			\end{equation*}
			such that for each $b \in B$ the restriction $\varphi|_{(E_1)_b}\colon (E_1)_b \to (E_2)_b$ is $\F$-linear.
			Two vector bundles are \strong{isomorphic}\index{vector bundle!isomorphism of} if there exist mutually inverse morphisms $f\colon E_1 \to E_2$ and $g\colon E_2 \to E_1$ between them.
		\item If $p\colon E \to B$ is a vector bundle and $f\colon X \to B$ a continuous map, we define the \strong{pullback bundle}\index{vector bundle!pullback}\index{pullback vector bundle|see {vector bundle!pullback}} $f^* E \coloneq X \times_B E = \{(x, e) \in X \times E \mid f(x) = p(e)\}$ with projection $f^* p\colon f^* E \to X$.
			In other words, we have a pullback square
			\begin{equation*}
				\begin{tikzcd}
					f^* E
					\ar[r]
					\ar[d, swap, "f^* p"]
					\ar[dr, phantom, "\lrcorner" very near start]
						& E
						\ar[d, "p"]
						\\
						X
						\ar[r, "f"]
						& B
				\end{tikzcd}
			\end{equation*}
			Then $(f^* E)_x \isom E_{f(x)}$ via $(x, e) \mapsto e$ which we use to give $(f^* E)_x$ the structure of an $\F$-vector space.
			If $\varphi_U\colon U \times \F^N \xto{\isom} p^{-1}(U)$ is a local trivialization of $p$, then setting $V \coloneq f^{-1}(U)$ we obtain a local trivialization of $f^* p$ via
			\begin{align*}
				V \times \F^n &\xto{\isom} (f^* p)^{-1}(V) \\
				(x, v) &\mapsto (x, \varphi_U(f(x), e))
			\end{align*}
			so $f^* p$ does indeed define a vector bundle.
	\end{enumerate}
	Roughly speaking, any natural continuous operations on vector spaces can be extended to vector bundles.
	We focus on the following examples:
	\begin{enumerate}[resume]
		\item \strong{Sum of vector bundles}\footnote{This operation is also commonly called the \strong{Whitney sum}\index{Whitney sum|see {vector bundle!sum of}}.}\index{vector bundle!sum of}:
			If $p\colon E \to B$ is an $n$-dimensional and $p'\colon E' \to B$ is n $n'$-dimensional $\F$-vector bundle, then there is an $(n + n')$-dimensional $\F$-vector bundle $p \dsum p'\colon E \dsum E' \to B$ with $E \dsum E' \coloneq E \times_B E' = \{(e, e') \in E \times E' \mid p(e) = p'(e')\}$ the fibrewise direct sum with projection $(p \dsum p')(e, e') \coloneq p(e) = p'(e')$.
			We have $(E \dsum E')_b = E_b \times E'_b = E_b \dsum E'_b$ which inherits an $(n + n')$-dimensional $\F$-vector bundle structure.
			The proof of local triviality is left to the reader.
		\item \strong{Realification}\index{vector bundle!realification of}\index{realification|see {vector bundle!realification of}} and \strong{complexification}\index{vector bundle!complexification of}\index{complexification|see {vector bundle!complexification of}}:
			If $p\colon E \to B$ is an $n$-dimensional complex vector bundle, we can also consider it as a $2n$-dimensional real vector bundle by neglecting structure fibrewise.
			We then write $E_{\R}$ to emphasize the real vector bundle structure.

			Conversely, if $p\colon E \to B$ is an $n$-dimensional real vector bundle, we can use the natural identifications
			\begin{align*}
				\C \tensor_{\R} V &\isom V \tensor V \\
				(1, v) &\mapsto (v, 0) \\
				(i, v) &\mapsto (0, v)
			\end{align*}
			to enhance the $2n$-dimensional $\R$-vector bundle $E \dsum E$ to an $n$-dimensional $\C$-vector bundle which we denote by  $E \tensor_{\R} \C$.
		\item \strong{Euclidean bundles}\index{vector bundle!euclidean}\index{euclidean vector bundle|see {vector bundle!euclidean}}.
			Recall from linear algebra that a \emph{euclidean vector space} is a finite-dimensional real vector space equipped with a positive definite quadratic form $\mu\colon V \to \R$, i.e. $\mu(v) > 0$ for all $v \neq 0$ and $v \cdot w \coloneq 1 / 2 (\mu(v + w) - \mu(v) - \mu(w))$ is bilinear.
			A \strong{euclidean vector bundle}\index{euclidean vector bundle} is then a real vector bundle $p\colon E \to B$ together with a continuous function $\mu\colon E \to \R$ which restricts to a positive definite quadratic form on each fibre.
			Such a $\mu$ is called a \strong{euclidean metric}\index{euclidean metric} on $E$.
			\begin{example}
				The trivial bundle $B \times \R^n \to B$ carries the euclidean metric $\mu(b, x) = x_1^2 + \ldots + x_n^2$.
				One can show that
				\begin{itemize}
					\item if $B$ is \emph{paracompact}\index{paracompactness} (i.e. every open cover admits a locally finite refinement), then every vector bundle over $B$ can be given a euclidean metric (this is exercise \ref{ex:euclideanmetric}); and 
					\item if $\mu$ and $\mu'$ are euclidean metrics on the same bundle $p\colon E \to B$, then there exists a bundle automorphism $\varphi$ of $p$ such that $\mu' = \mu \circ \varphi$ (see \cite[Exercise 2-E]{milnor_characteristic_1974}).
				\end{itemize}
			\end{example}
			Similarly one defines \strong{hermitian metrics}\index{hermitian metric} on $\C$-vector bundles.
		\item \strong{Orthogonal complement bundles}\index{vector bundle!orthogonal complement}\index{orthogonal complement of a vector subbundle|see {vector bundle!orthogonal complement}}:
			Let $p\colon E \to B$ be a real euclidean vector bundle with given subbundle $p|_{\hat{E}}\colon \hat{E} \to B$, $\hat{E} \subseteq E$.
			Then the \emph{orthogonal complement bundle} $\hat{E}^\perp$ is defined to be the subspace $E \supseteq \hat{E}^\perp \coloneq \big\{e \in E \mid e \in (\hat{E}_{p(e)})^\perp\big\}$ using the scalar product induced by the metric.

			This is locally trivial:
			Let $h\colon U \times \R^n \to p^{-1}(U)$ be a local trivialization of $p$. 
			Transporting the metric $\mu$ on $E \supseteq p^{-1}(U)$ over along $h$, we obtain a euclidean metric on $U \times \R^n$.
			By the previous comment, we can assume up to isomorphism that this is the standard metric $\mu(b, x) = x_1^2 + \ldots + x_n^2$.
			Replacing $U$ by a smaller neighborhood if necessary, we can further assume that $p|_{\hat{E}}^{-1}(U) \isom U \times \R^k$ can be trivialized.
			Composing the two equivalences, we obtain an isometric embedding
			\begin{align*}
				U \times \R^k &\incl U \times \R^n \\
				(b, v) &\mapsto (b, \psi(b, v))
			\end{align*}
			By reordering the entries of $\R^n$ if necessary and replacing $U$ by a smaller neighborhood, we can assume that $\psi(b, e_1), \ldots, \psi(b, e_k)$ intersect trivially with $\{b\} \times \R^0 \times \R^{n - k}$ for all $b$ so that the tuple 
			\begin{equation*}
				(\psi(b, e_1), \ldots, \psi(b, e_k), e_{k + 1}, \ldots, e_n)
			\end{equation*}
			forms a basis of $\R^n$ for all $b$.
			Applying the Gram-Schmidt process (which is continuous), we obtain a basis 
			\begin{equation*}
				(\psi(b, e_1), \ldots, \psi(b, e_k), \bar{\psi}(b)_1, \ldots, \bar{\psi}(b)_{n - k})
			\end{equation*}
			Then the function
			\begin{align*}
				U \times \R^{n - k} &\incl U \times \R^n \\
				(b, x) &\mapsto \sum x_i \bar{\psi}(b)_i
			\end{align*}
			defines a homeomorphism to the orthogonal complement of the image of the map $U \times \R^k \incl U \times \R^n$, $(b, v) \mapsto (b, \psi(b, v))$ above.
			Translated back to $h$, this provides a local trivialization of $\hat{E}^\perp$.
	\end{enumerate}
\end{definition}
\lecture{04.12.23}
\begin{example}
	Consider the Möbius bundle $\gamma_{\R}^{1, 2} \to \RP^1 \isom S^1$ as a subbundle of the trivial bundle $\RP^1 \times \R^2 \to \RP^1$.
	The orthogonal complement bundle $\big(\gamma_{\R}^{1, 2}\big)^\perp$ is isomorphic to $\gamma_{\R}^{1, 2}$ itself via $\gamma_{\R}^{1, 2} \to \big(\gamma_{\R}^{1, 2}\big)^\perp$, $(L, v) \mapsto (L, Av)$ where $A\colon \R^2 \to \R^2$ is rotation by $\pi / 2$.

	We will soon see that $\gamma_{\R}^{1, 2}$ is non-trivial; however, it becomes trivial after pulling back along the degree-2 map $f\colon S^1 \to \RP^1 \isom S^1$, $x \mapsto [x]$:
	The total space $f^* \gamma_{\R}^{1, 2}$ is given by the space of pairs $\{(x, v) \mid v \in [x]\} \subseteq S^1 \times \R^2$, so the map $S^1 \times \R \to f^* \gamma_{\R}^{1, 2}$, $(x, \lambda) \mapsto (x, \lambda x)$ is a bundle isomorphism.
\end{example}
\begin{definition}
	Let $X$ be a topological space and $n \in \N$ a natural number.
	We denote by $\Vect_{\F}^n(X)$ the set of isomorphism classes of $n$-dimensional $\F$-vector bundles on $X$.
	Via the pullback of bundles this becomes a contravariant functor in $X$.
\end{definition}
We want to study this functor.
We start with the following:
\begin{proposition}\label{prp:vecbundlehtpyinvariance}
	Let $X$ be paracompact and $\xi$ a vector bundle on $X \times I$.
	Then the pullbacks $\iota_0^* \xi$ and $\iota_1^* \xi$ where $\iota_t\colon X \incl X \times I$ is the inclusion $x \mapsto (x, t)$ for all $t \in I$ are isomorphic as vector bundles over $X$.
\end{proposition}
For this, we will need the following lemma:
\begin{lemma}
	There exists an open covering $\{U_j\}_j$ of $X$ such that $\xi$ is trivializable over each $U_j \times I$.
\end{lemma}
\begin{proof}
	First we note that if $U \subseteq X$ is open and $\xi$ is trivializable over $U \times [a, b]$ and over $U \times [b, c]$ for given $a \leq b \leq c \in I$, then it is trivializable over $U \times [a, c]$.
	To see this, let $h_1\colon p^{-1}(U \times [a, b]) \xto{\isom} U \times [a, b] \times \F^n$ and $h_2\colon p^{-1}(U \times [b, c]) \xto{\isom} U \times [b, c] \times \F^n$ be trivializations.
	Then
	\begin{equation*}
		\begin{tikzcd}[column sep = smallish]
			{h_3\colon p^{-1}(U \times [b, c])}
			\ar[r, "h_2"]
				& {U \times [b, c] \times \F^n}
				\ar[r, "{\id_{[b, c]} \times \varphi}"]
				&[2.1em] {U \times [b, c] \times \F^n}
		\end{tikzcd}
	\end{equation*}
	where
	\begin{equation*}
		\begin{tikzcd}[column sep = smallish]
			\varphi\colon U \times \F^n \isom U \times \{b\} \times \F^n
			\ar[r, "h_2^{-1}"]
				& p^{-1}(U \times \{b\})
				\ar[r, "h_1"]
				& U \times \{b\} \times \F^n \isom U \times \F^n
		\end{tikzcd}
	\end{equation*}
	is another trivialization which agrees with $h_1$ on $U \times \{b\}$.
	Hence the two glue together, giving a trivialization on all of $U \times [a, c]$.

	By compactness of $I$, for every $x \in X$ we can find open neighbourhoods $U_{x, 1}, \ldots, U_{x, k}$ of $x$ and partitions $0 = t_0 < t_1 < \ldots < t_k = 1$ such that $\xi$ is trivializable over each $U_{x, i} \times [t_{i - 1}, t_i]$.
	Thus $\xi$ is trivializable on all of $U_x \times I$, with $U_x = U_{x, 1} \cap \ldots \cap U_{x, k}$ by the above.
\end{proof}
\begin{proof}[Proof of proposition \ref{prp:vecbundlehtpyinvariance}]
	Let $\{U_j\}_j$ be an open cover of $X \times I$ as in the previous lemma.
	Since $X$ is paracompact, there exists a countable cover $\{V_i\}_{i \in \N}$ of $X$ subordinate to $\{U_j\}_j$ (i.e. each $V_i$ is contained in some $U_j$) and a partition of unity $\{\varphi_i\colon X \to [0, 1]\}_{i \in \N}$ with $\supp(\varphi_i) \subseteq V_i$.
	Hence $\xi$ is trivializable over each $V_i \times I$.
	Let $\psi_i \coloneq \varphi_1 + \ldots + \varphi_i$ and let $p_i\colon E_i \to X$ be the pullback of $\xi$ along $X \to X \times I$, $x \mapsto (x, \psi_i(x))$ (example \ref{expl:smallcovervbhtpyinv} below may help with intuition).

	We define a homeomorphism $h_i\colon E_i \xto{\isom} E_{i - 1}$ as follows:
	Outside of $p^{-1}(V_i)$, set $h_i \coloneq \id$. 
	Then choose a trivialization $\tilde{h}_i\colon p^{-1}(V_i \times I) \xto{\isom} V_i \times I \times \F^n$, and set $h_i(x, \psi_i(x), v) \coloneq (x, \psi_{i - 1}(x), v)$ where $(x, \psi_i(x), v) \in V_i \times I \times \F^n$.
	The infinite composite $\cdots \circ h_2 \circ h_1$ (which we note is well-defined since the cover $\{V_i\}_{i \in \N}$ is locally finite) is then an isomorphism from the pullback $\iota_0^* \xi$ to the pullback $\iota_1^* \xi$ as desired.
\end{proof}
\begin{example}\label{expl:smallcovervbhtpyinv}
	Let $X = V_1 \cup V_2$ be an open cover.
	Then a trivialization $\varphi_1$ on $V_1 \times I$ yields an isomorphism $\iota_0^* \xi \isom (\id, \varphi_1)^* \xi$ that is constant outside $V_1$.
\end{example}
\begin{corollary}
	If $X$ is paracompact, $f_1, f_2\colon X \to Y$ are two homotopic maps and $\xi$ is a vector bundle over $Y$, then $f_1^* \xi \isom f_2^* \xi$.
\end{corollary}
\begin{proof}
	Let $H\colon X \times I \to Y$ be a homotopy.
	Then $f_1 = H \circ \iota_0$ and $f_2 = H \circ \iota_1$ and hence
	\begin{equation*}
		f_1^* \xi = \iota_0^* H^* \xi \isom i_1^* H^* \xi = f_2^* \xi
	\end{equation*}
	where $H^* \xi$ is the bundle over $X \times I$.
\end{proof}
Next we aim to show that there exist universal bundles, at least over paracompact spaces.
\begin{definition}
	For $n \in \N$ we define the \strong{infinite Grassmann manifold}\index{infinite Grassmann manifold|see {Grassmannian!infinite}} or \strong{infinite Grassmannian}\index{Grassmannian!infinite} $\Gr_n^{\F} \coloneq \Gr_n(\F^\infty)$, where $\F^\infty$ is the $\F$-vector space of sequences $(x_1, x_2, \ldots)$, $x_i \in \F$, with almost all $x_i = 0$.
	Note that $\Gr_n^{\F}$ is not a manifold.
	It comes equipped with the weak topology with respect to the filtration by the $\Gr_n(\F^N)$.
	Similarly, we define $\gamma_{\F}^n \coloneq \{(V, v) \mid v \in V\} \subseteq \Gr_n^{\F} \times \F^\infty$ with $(\gamma_{\F}^n)_V \isom \{V\} \times V \isom V$ a vector space.
\end{definition}
\begin{lemma}
	The projection $p\colon \gamma^n_{\F} \to \Gr_n^{\F}$, $(V, v) \mapsto V$ is an $n$-dimensional $\F$-vector bundle.
\end{lemma}
\begin{proof}
	Similar to the finite-dimensional case, using that $\Gr_n^{\F} \times \F^\infty$ comes with the weak topology.
\end{proof}
\begin{theorem}\label{thm:vecbunrepresentability}
	If $X$ is paracompact, then the natural map
	\begin{equation*}
		\big[X, \Gr_n^{\F}\big] \to \Vect_{\F}^n(X), \qquad \big[f\colon X \to \Gr^{F}_n\big] \mapsto [f^* \gamma_{\F}^n]
	\end{equation*}
	is a bijection.
	In other words, the functor $\Vect_{\F}^n({{-}})$ is represented by the pair $\big(\Gr_n^{\F}, \gamma_{\F}^n\big)$ in the homotopy category $\hToprcpt_*$ of based paracompact topological spaces.
\end{theorem}
\begin{proof}
	For surjectivity, let $p\colon E \to X$ be an $n$-dimensional $\F$-vector bundle.
	Assume there exists continuous map $\tilde{f}\colon E \to \F^\infty$ which is linear and injective on each fibre.
	Then $f\colon X \to \Gr_n^{\F}$, $x \mapsto \tilde{f}(E_x)$ is continuous (this can be checked after trivialization) and $E \to f^* \gamma_{\F}^n \subseteq X \times \Gr_n^{\F} \times \F^\infty$, $e \mapsto (p(e), f(p(e)), \tilde{f}(e))$ defines a bundle isomorphism from $E$ to $f^* \gamma_{\F}^n$.

	To construct $\tilde{f}$, we choose a numerable trivializing cover $\{U_i\}_i$ as before and trivializations $h_i\colon p^{-1}(U_i) \to U_i \times \F^n$.
	Then $\tilde{f}_i\colon p^{-1}(U_i) \xto{h_i} U_i \times \F^n \xto{\pr_{\F^n}} \F^n$ defines such a map for the restricted bundle over each $U_i$ with codomain $\F^n$ instead of $\F^\infty$.
	We choose a partition of unity $\{\varphi_i\}_i$ subordinate to this cover and define
	\begin{equation*}
		\tilde{f} = \big(\varphi_1 \circ p \circ \tilde{f}_1, \varphi_2 \circ p \circ \tilde{f}_2, \ldots\big) \subseteq (\F^n)^\infty \isom \F^\infty
	\end{equation*}
	which has the desired properties.

	For injectivity, let $f_1, f_2\colon X \to \Gr^{\F}_n$ be such that $f_1^* \gamma_{\F}^n \isom f_2^* \gamma_{\F}^n$ and write $E_i$ for the total space of $f_i^* \gamma_{\F}^n$.
	As in the first part, we obtain maps $\tilde{f}_i\colon E_i \to \F^\infty$.
	We can precompose $\tilde{f}_2$ with the a $E_1 \to E_2$ to get two maps $\tilde{f}_1, \tilde{g}\colon E_1 \to \F^\infty$, each linear and injective on fibres.

	It suffices to show that $\tilde{f}_1$ and $\tilde{g}$ are homotopic through maps that are linear and injective on fibres, since this gives a homotopy $f_1 \Rightarrow f_2$.
	To produce this homotopy, we first note that there is a linear injective homotopy $h_t\colon \F^\infty \to \F^\infty$ from $\id_{\F^\infty}$ to the map $\tau(x_1, x_2, \ldots) = (x_1, 0, x_2, 0, \ldots)$, for example the direct path.
	After postcomposing with this homotopy, we can assume that $\tilde{g}$ is concentrated purely in odd and similarly that $\tilde{f}_1$ is concentrated purely in even degrees.
	Then the direct path homotopy $t \mapsto t \tilde{f}_1 + (1 - t) \tilde{g}$ has the desired properties.
\end{proof}
\begin{corollary}
	If $X$ is compact and $p\colon E \to X$ is an $n$-dimensional $\F$-vector bundle, then there exists $m \in \N$ and $f\colon X \to \Gr_n(\F^m)$ such that $f^* \gamma_{\F}^{n, m} \isom E$. 
\end{corollary}
\begin{proof}
	By compactness, the classifying map $X \to \Gr^{\F}_n$ factors through some finite stage in the filtration definining $\Gr^{\F}_n$, i.e. some $\Gr_n(\F^m)$, since $\Gr^{\F}_n$ carries the weak topology.
\end{proof}
\begin{corollary}\label{prop:swanslemma}
	Let $X$ be compact and $p\colon E \to X$ an $n$-dimensional $\F$-vector bundle.
	The there is an $\F$-vector bundle $p'\colon E' \to X$ such that $p \dsum p'\colon E \dsum E' \to X$ is isomorphic to a trivial bundle.
\end{corollary}
\begin{proof}
	Since pullbacks of bundles preserve direct sums and trivial bundles, it suffices to show the statement for the bundles $\gamma_{\F}^{n , m}$ by the previous corollary.
	But $\gamma_{\F}^{n, m}$ is by definition a subbundle of the trivial bundle $\Gr_n(\F^m) \times \F^m$ with complement given by the orthogonal complement bundle for the standard euclidean metric.
\end{proof}
\lecture{08.12.23}
\begin{example}
	From the above, we obtain isomorphisms
	\begin{equation*}
		\Vect_{\R}^1(X) \isom [X, \RP^\infty] \isom H^1(X; \F_2)
	\end{equation*}
	and
	\begin{equation*}
		\Vect_{\C}^1(X) \isom [X, \CP^\infty] \isom H^2(X; \Z)
	\end{equation*}
	if $X$ is a CW-complex.
	It follows that $\gamma_{\R}^1$ is the unique non-trivial line bundle on $\RP^\infty$ since $H^1(\RP^\infty; \F_2) \isom \F_2$.
	Moreover, since $H^1(\RP^\infty; \F_2) \xto{\isom} H^1(\RP^m; \F_2)$ for all $m \geq 1$, we see that $\gamma_{\R}^{1, m + 1}$ is also non-trivial for all $m \geq 1$; in particular the Möbius bundle is non-trivial.
	Similarly, the line bundles $\gamma_{\C}^{1, m + 1}$ are generators of $\Vect_{\C}^1(\CP^m)$ for all $m \geq 1$.
	In particular, $\Vect_{\R}^1(X)$ and $\Vect_{\C}^1(X)$ have natural abelian group structures.
	One can show that this is given by the tensor product of line bundles.
	In fact, natural operations $\Vect_{\R}^1({{-}}) \times \Vect_{\R}^1({{-}}) \to \Vect_{\R}^1({{-}})$ correspond, by the Yoneda lemma, to elements of
	\begin{equation*}
		H^1(\RP^\infty \times \RP^\infty; \F_2) \isom H^1(\RP^\infty \times \{*\}; \F_2) \dsum H^1(\{*\} \times \RP^\infty; \F_2)
	\end{equation*}
	Only two of these are invariant under the $C_2$-action which is the case for the one classifying the tensor product since it is symmetric.
	Since the tensor product is non-trivial in general, it must correspond to the diagonal element $(1, 1) \in \F_2 \dsum \F_2$.
	Hence, every real line bundle $\xi$ on $X$ has an associated cohomology class $\omega_1(\xi) \in H^1(X; \F_2)$ which is a full invariant of $\xi$ up to isomorphism.
	This is an example of a \emph{characteristic class}, the theory of which we now develop systematically.
\end{example}

\subsection{Characteristic classes}
\begin{definition}
	A \strong{characteristic class}\index{characteristic class} is a natural transformation of the form
	\begin{equation*}
		\Vect_{\F}^n({{-}}) \Rightarrow H^m({{-}}; A)
	\end{equation*}
	for $n, m \in \N$ and $A$ an abelian group.
\end{definition}
\begin{remark}
	At least when we restrict to paracompact spaces, characteristic classes correspond to elements in $H^m\big(\Gr^{\F}_n; A\big)$ (by representability and the Yoneda lemma).
\end{remark}
We will mainly focus on the case $\F = \R$ and $A = \F_2$.
Our main goal is to show the following:
\begin{theorem}\label{thm:existenceofswclasses}
	For every real vector bundle $\xi\colon E \to B$ there exist characteristic classes $\omega_i(\xi) \in H^i(B; \F_2)$, $i \in \N$, called the \strong{Stiefel-Whitney classes}\index{Stiefel-Whitney class} satisfying the following properties:
	\begin{enumerate}
		\item\label{ax:stiefwhittriviality} $\omega_0(\xi) = 1$ and $\omega_i(\xi) = 0$ for $i > \dim \xi$.
		\item \strong{Naturality}:
			If $f\colon B' \to B$ is a continuous map, then
			\begin{equation*}
				\omega_i(f^* \xi) = f^* \omega_i(\xi)
			\end{equation*}
		\item\label{ax:whitneyproduct} \strong{Whitney product formula}\index{Whitney product formula}:
			If $\xi$ and $\eta$ are real vector bundles over the same base $B$, then
			\begin{equation*}
				\omega_k(\xi \dsum \eta) = \sum_{i = 0}^k \omega_i(\xi) \smile \omega_{k - i}(\eta)
			\end{equation*}
		\item\label{ax:stiefwhitnormalization} $\omega_1\big(\gamma_{\R}^{1, 2}\big) \neq 0$ in $H^1(\RP^1; \F_2)$.
	\end{enumerate}
	Moreover, when restricted to paracompact spaces the $\omega_i$ are unique with these properties.
\end{theorem}
We first assume the theorem and record some elementary properties:
\begin{itemize}
	\item If $\epsilon$ is a trivial bundle, then $\omega_i(\epsilon) = 0$ for $i > 0$ since $\epsilon$ can be pulled back from a point and $H^i({{*}}; \F_2) = 0$ for $i > 0$.
	\item We have $\omega_i(\xi \dsum \epsilon) = \omega_i(\xi)$ for all $i$ by the Whitney product formula and the previous item (this is to say that the $\omega_i$ are \emph{stable}).
	\item If we set $H^\pi(B; \F_2) \coloneq \prod_{n \in \N} H^n(B; \F_2)$ with ring structure given by extending the cup product, i.e.
		\begin{equation*}
			(x_i)_{i \in \N} \cdot (y_j)_{j \in \N} \coloneq \bigg(\sum_{i = 0}^k x_i \smile y_{k - i}\bigg)_{k \in \N}
		\end{equation*}
		then the \strong{total Stiefel-Whitney class}\index{Stiefel-Whitney class!total class} $\omega(\xi) \in H^\pi(B; \F_2)$ is defined as $\omega(\xi) \coloneq \omega_0(\xi) + \omega_1(\xi) + \omega_2(\xi) + \cdots$.
		The Whitney product formula implies that  $\omega(\xi \dsum \eta) = \omega(\xi) \omega(\eta)$.
		Hence, if $\xi \dsum \eta = \epsilon^m$ is a trivial bundle, we have $\omega(\xi) \omega(\eta) = \omega(\xi \dsum \eta) = \omega(\epsilon^m) = 1$ and therefore $\omega(\xi) = \omega(\eta)^{-1}$.
	\item We have $\Vect_{\R}^1(\RP^1) \isom H^1(\RP^1; \F_2) \isom \F_2$, hence axiom \ref{ax:stiefwhitnormalization} forces $\omega_1\big(\gamma_{\R}^{1, 2}\big)$ to be the unique non-trivial element of $H^1(\RP^n; \F_2)$.
		Since $\gamma_{\R}^{1, 2}$ is the pullback of $\gamma_{\R}^1$ along $\RP^1 \incl \RP^\infty$ and the induced map $H^1(\RP^\infty; \F_2) \to H^1(\RP^1; \F_2)$ is an isomorphism, $\omega_1(\gamma_{\R}^1) \in H^1(\RP^\infty; \F_2)$ must be the non-trivial element $u$.
		Thus, $\omega(\gamma_{\R}^1) = 1 + u \in H^\pi(\RP^\infty; \F_2)$ and its inverse is given by $1 + u + u^2 + u^3 + \cdots \in H^\pi(\RP^\infty; \F_2)$, but this has infinitely many non-trivial terms and therefore cannot be equal to $\omega(\eta)$ for some vector bundle $\eta$ on $\RP^\infty$.
		Hence, $\gamma_{\R}^1$ does not embed into a trivial bundle, in contrast to bundles over compact spaces (cf. proposition \ref{prop:swanslemma}).

		In contrast, over $\RP^n$ we have that $\omega\big(\gamma_{\R}^{1, n + 1}\big)^{-1} = 1 + u + u^2 + \cdots + u^n = \omega(\xi)$ where $\gamma_{\R}^{1, n + 1} \dsum \xi = \epsilon^{n + 1}$ (i.e. $\xi \isom \big(\gamma_{\R}^{1, n + 1}\big)^\perp$).
\end{itemize}
\subsubsection{Construction of the Stiefel-Whitney classes}
\begin{note}
	Throughout this whole subsection, we implicitly assume $\F_2$-coefficients for (co)homology.
\end{note}
We start our construction of the $\omega_i$ with a brief treatment of \emph{Thom classes} and \emph{Thom isomorphisms}.
Let $p\colon E \to B$ be a real vector bundle and set $E_0 \subset E$ to be all elements which are not the 0-element in their respective fibre.
Then $p|_{E_0}\colon E_0 \to B$ is a fibre bundle with fibres homeomorphic to $\R^n \setminus \{0\}$.
We can apply the relative Serre spectral sequence constructed in proposition \ref{prop:relserrespecseq} of the form $H^p(B; H^q(\R^n, \R^n \setminus \{0\})) \Rightarrow H^{p + q}(E, E_0)$.
Now $H^q(\R^n, \R^n \setminus \{0\}) \isom \F_2$ if $q = n$ and 0 else, and the local coefficient system is necessarily trivial since any automorphism of $\F_2$ is the identity, so we obtain the following picture:
\begin{center}
	\pgfsetlayers{background,main}
	\tikzsetnextfilename{vecbun_thom_iso_spec_seq}
	\begin{tikzpicture}
		\matrix[
			spectral sequence/page,
			name = m, 
			column sep = {3.5em, between origins},
			row sep = 3ex, 
			column 1/.append style = {anchor = base},
			row 4/.style = {font = \scriptsize}] {
				n &[-1.4em] H^0(B) & H^1(B) & \cdots \\
				\vdotswithin{0} & \phantom{0} \\
				0 & \phantom{0} & \phantom{0} & \phantom{0} \\[-2ex]
				& 0 & 1 & \cdots \\
		};

		\coordinate (Origin) at (m-1-2.west |- m-3-2.south);

		\draw[spectral sequence/axis] (Origin) -- (Origin |- m-1-2.north) -- ++(0, 1) node[left] (q) {$q$};
		\draw[spectral sequence/axis] (Origin) -- (m-3-4.south east) -- ++(1.5, 0) node[below] (p) {$p$};

		\coordinate (Top Right) at (p |- q);

		\coordinate (Slightly Left of p) at ($(p) - (0.14, 0)$);
		\coordinate (Slightly Below q) at ($(q) - (0, 0.14)$);

		\begin{pgfonlayer}{background}
			\draw[spectral sequence/zero region] (Origin) rectangle (Slightly Left of p |- Slightly Below q); 
			\fill[white] (m-1-2.south west) rectangle (m-1-3.north -| Slightly Left of p);
		\end{pgfonlayer}

		\node[spectral sequence/page label] at ($(Top Right) - (0.3, 0.3)$) {$E_2^{p, q}$};
	\end{tikzpicture}
\end{center}
\begin{theorem}
	There is a natural isomorphism 
	\begin{equation*}
		\Phi\colon H^*(B) \xto{\isom} H^{* + n}(E, E_0)
	\end{equation*}
	called the \strong{Thom isomorphism}\index{Thom isomorphism}.
	The image $u = \Phi(1) \in H^n(E, E_0)$ is called the \strong{Thom class}\index{Thom class} of the bundle.
	Then the map $\Phi$ is given by the cup-product $H^*(B) \xto{\isom} H^*(E) \xto{{{-}} \smile u} H^{* + n}(E, E_0)$.
\end{theorem}
\begin{proof}
	This follows directly from the Serre spectral sequence we just considered and its multiplicative structure.
\end{proof}
Note that $u$ is uniquely determined by the fact that it restricts to a generator in $H^n(F_b, (F_b)_0)$ on each of the fibres $F_b$.
\begin{remark}
	\leavevmode
	\begin{enumerate}
		\item This construction crucially depends on $\F_2$-coefficients.
			Over $\Z$, the local system $H^q\big(F_{({{-}})}, \big(F_{({{-}})}\big)_0; \Z\big)$ is trivial if and only if the bundle is \strong{orientable}\index{orientability!of vector bundles}, i.e. one can choose orientations on each fibre $F_b$ which are locally compatible in the sense that the local trivializations all send them to the same orientation of $\R^n$.
			Since every complex vector space has a canonical real orientation, every complex vector bundle is orientable and hence has a Thom isomorphism with $\Z$-coefficients.
		\item The Thom isomorphism is sometimes phrased differently:
			If $p\colon E \to B$ is euclidean, we can form the \strong{disc bundle}\index{disc bundle} $D(E)$ and the \strong{sphere bundle}\index{sphere bundle} $S(E)$ by restricting to the vectors of length $\leq 1$ and $= 1$ fibrewise, respectively.
			Then $D(E) \incl E$ and $S(E) \incl E_0$ are both homotopy equivalences; homotopy inverses are given by the maps $E \xto{p} B \xto{s_0} D(E)$ where $s_0$ is the 0-section and $E_0 \to S(E)$, $v \mapsto \frac{v}{|v|}$, and homotopies are formed fibrewise.
			Hence, there is an isomorphism $H^*(E, E_0) \xto{\isom} H^*(D(E), S(E))$.
			The pair $(D(E), S(E))$ has the advantage that it is excisive, so we can further identify
			\begin{equation*}
				H^*(D(E), S(E)) \xto{\isom} \tilde{H}^*(D(E) / S(E))
			\end{equation*}
			The space $D(E) / S(E) \eqcolon \Th(p)$ is called the \strong{Thom space}\index{Thom space} of $p$.
	\end{enumerate}
\end{remark}
\lecture{11.12.23}
Note that we have a commutative square
\begin{equation*}
	\begin{tikzcd}
		S(E)
				\ar[r, "p|_{S(E)}"]
				\ar[d, equal]
			& B
		\\
		S(E)
				\ar[r, hook]
			& D(E)
				\ar[u, swap, "\htpyeqv"]
	\end{tikzcd}
\end{equation*}
Hence, $\Th(p)$ is homotopy equivalent to the mapping cone of the sphere bundle $S(E) \to B$.
If $B$ is compact, then $\Th(p)$ is a homeomorphism to the one-point compactification of $E$ (showing this is exercise \ref{ex:thomspacecompactification}).
% TODO "unreduced suspension" ???
If $p = \epsilon^n$ is trivial, then $\Th(p) = S^n \wedge B_{+}$ is the unreduced suspension and the Thom isomorphism is the usual suspension isomorphism.
In general, $\Th(p)$ is a twisted suspension of $B_{+}$.
The Thom isomorphism says that this twist is invisible to $H^*({{-}})$, at least if considered as a functor to graded $\F_2$-vector spaces.
\begin{definition}\index{Stiefel-Whitney class}
	Let $\xi\colon E \to B$ be a real vector bundle.
	We define $\omega_i(\xi) \coloneq \Phi^{-1}\big(\Sq^i u\big) \in H^i(B)$.
	In other words, $\omega_i(\xi)$ is characterized by the equation
	\begin{equation*}
		\omega_i(\xi) \smile u = \Sq^i u \in H^{n + i}(E, E_0)
	\end{equation*}
\end{definition}
This leads to the slogan \enquote{Stiefel-Whitney classes measure the failure of $\Phi$ to commute with the Steenrod operations.}
\begin{remark}
	By definition, the Stiefel-Whitney classes depend only on the underlying sphere bundle of the vector bundle.
\end{remark}
\begin{proof}[Proof of theorem \ref{thm:existenceofswclasses}]
	It remains to show that the $\omega_i$ satisfy the four axioms put forth:
	\begin{enumerate}
		\item We have $\omega_0(\xi) = \Phi^{-1}\big(\Sq^0 u\big) = \Phi^{-1}(u) = 1 \in H^0(B)$.
			Moreover, $u$ sits in degree $n = \dim \xi$, so $\Sq^i u = 0$ and therefore $\omega_i = 0$ for all $i > n$.
		\item For naturality, if $f\colon B' \to B$ is a map, we obtain a canonical bundle map $g\colon f^* E \to E$ which is a linear isomorphism on fibres $(f^* E)_{b'} \xto{\isom} E_{f(b')}$ for all $b' \in B'$.
			It follows that $g^* u \in H^n\big(f^* E, (f^* E)_0\big)$ is the Thom class for $f^* E$ since its restriction to the fibres can be computed through the diagram
			\begin{equation*}
				\begin{tikzcd}
					H^n\big(f^* E, (f^* E)_0\big)
							\ar[d]
						& H^n\big(E, E_0\big)
							\ar[l]
							\ar[d]
					\\
					H^n\big((f^* E)_{b'}, ((f^* E)_{b'})_0\big)
						& H^n\big(E_{f(b')}, (E_{f(b')})_0\big)
							\ar[l, swap, "\isom"]
				\end{tikzcd}
			\end{equation*}
			to be a generator.
			Hence we obtain $g^* u = u'$ and see that 
			\begin{align*}
				f^* \omega_i(\xi) \smile u' &= f^* \omega_i(\xi) \smile g^* u \\ 
											&= g^*(\omega_i(\xi) \smile u) \\ 
											&= g^*\big(\Sq^i u\big) \\ 
											&= \Sq^i(g^* u) \\
											&= \Sq^i(u')
			\end{align*}
			which is to say that $f^* \omega_i(\xi) = \omega_i(f^* \xi)$.
		\item To show the Whitney product formula, we first discuss the effect of the external cross product on Stiefel-Whitney classes.
			Given real vector bundles $p\colon E \to B$ and $p'\colon E' \to B'$ of dimension $m$ and $n$, respectively, the product $p \times p'\colon E \times E' \to B \times B'$ again forms a vector bundle of dimension $m + n$ with the fibres $(E \times E')_{(b, b')} = E_b \times E'_{b'}$ given the product vector space structure.
			Local trivializations of this bundle are then given by products of local trivializations of the component bundles.
			The previously defined \emph{internal} direct sum $p \dsum p'$ in the case where $B = B'$ is obtained as the pullback $p \dsum p' = \Delta^*(p \times p')$ along the diagonal $\Delta\colon B \to B \times B$.

			Consider the cross-product $H^m(E, E_0) \times H^n(E', E'_0) \xto{\times} H^{m + n}(E \times E', E_0 \times E' \cup E \times E'_0)$ and observe that $E_0 \times E' \cup E \times E'_0 = (E \times E')_0$ since $(V \times W) \setminus \{0\} = (V \setminus \{0\}) \times W \cup V \times (W \setminus \{0\})$ for all vector spaces $V, W$.
			Hence, we claim that $u \times u'$ is a Thom class for $p \times p'$:
			For this, it suffices to show that it restricts to a generator on each fibre.
			By naturality of the cross product, this restriction is the cross product of the restriction of $u$ to $H^m(E_0, (E_b)_0)$ and $u'$ to $H^n(E'_{b'}, (E'_{b'})_0)$.
			We know that this is a generator (for example, identify $H^m(\R^m, \R^m \setminus \{0\})$ with $H^m(D^m, S^{m - 1}) \isom \tilde{H}^m(S^m)$ and likewise for $H^n(\R^n, \R^n \setminus \{0\}) \isom \tilde{H}^n(S^n)$ and note that by the Künneth theorem $\tilde{H}^m(S^m) \tensor \tilde{H^n}(S^n) \xto{\isom} \tilde{H}^{m + n}(S^{m + n})$).
			Hence, given $a \in H^*(B)$ and $b \in H^*(B')$ we obtain $(a \times b) \smile (u \times u') = (a \smile u) \times (b \smile u')$ by the compatibility of $\times$ and $\smile$.
			This shows that the Thom isomorphism satisfies $\Phi(a \times b) = \Phi(a) \times \Phi(b)$.

			Thus,
			\begin{align*}
				\omega_i(p \times p') &= \Phi^{-1}\big(\Sq^i (u \times u')\big) \\ 
				% extremely good typesetting
									  &= \Phi^{-1}\bigg(\sum_{j = 0}^i \Sq^j u \times \Sq^{i - j} u'\bigg) \;\;\mathrlap{\text{(by the Cartan formula)}}\qquad\qquad\qquad \\
									  &= \sum_{j = 0}^i \Phi^{-1}\big(\Sq^j u \times \Sq^{i - j} u'\big) \\
									  &= \sum_{j = 0}^i \omega_j(p) \times \omega_{i - j}(p')
			\end{align*}
			As mentioned above, we obtain $p \dsum p'$ (in the case $B = B'$) as $\Delta^*(p \times p')$.
			By naturality, we get 
			\begin{align*}
				\omega_i(p \dsum p') &= \omega_i \Delta^*(p \times p') \\
									 &= \Delta^* \omega_i(p \times p') \\
									 &= \Delta^*\bigg(\sum_{j = 0}^i \omega_j(p) \times \omega_{i - j}(p')\bigg) \\
									 &= \sum_{j = 0}^i \omega_j(p) \smile \omega_{i - j}(p')
			\end{align*}
			since $\Delta^*(a \times b) = a \smile b$ by definition.
		\item We consider the Möbius bundle $\gamma_\R^{1, 2}$ over $\RP^1$ equipped with the euclidean metric from its embedding into $\RP^1 \times \R^2$.
			Then the disc bundle $D\big(\gamma_\R^{1, 2}\big)$ is a closed Möbius strip and the sphere bundle $S\big(\gamma_\R^{1, 2}\big)$ is its boundary, so 
			\begin{equation*}
				H^*\big(\gamma_\R^{1, 2}, \big(\gamma_\R^{1, 2}\big)_0\big) \isom \tilde{H}^*\big(D\big(\gamma_\R^{1, 2}\big) / S\big(\gamma_\R^{1, 2}\big)\big) \isom \tilde{H}^*\big(\RP^2\big)
			\end{equation*}
			as $D\big(\gamma_\R^{1, 2}\big) / S\big(\gamma_\R^{1, 2}\big) \isom \RP^2$ (see exercise \ref{ex:thomspacerpn}).
			The Thom class $u \in \tilde{H}^1\big(\RP^2\big)$ must be the generator, so we have $\Sq^1 u = u^2 \neq 0$ in $\tilde{H}^2\big(\RP^2\big) \isom \F_2$ and therefore $\omega_1\big(\gamma_\R^{1, 2}\big) \neq 0$.
			\qedhere
	\end{enumerate}
\end{proof}

Note that we can form further characteristic classes out of the $\omega_i$ via sum and cup product.
For example, $\omega_1^3 + \omega_1 \omega_2$ is a characteristic class of degree 3 defined for all real vector bundles.
Our next goal is to show that over paracompact spaces \emph{all} characteristic classes are of this form.
Using the identification 
\begin{equation*}
	H^*\big(\Gr_n^{\R}\big) \leftrightarrow \{\text{characteristic classes for bundles over paracompact spaces}\}
\end{equation*}
we can think of the $\omega_i$ as elements in $H^*\big(\Gr^\R_n\big)$ and abbreviate $\omega_i(\gamma_\R^n)$ to $\omega_i$.
We obtain a map
\begin{equation*}
	\F_2[\omega_1, \ldots, \omega_n] \xto{\alpha_n} H^*\big(\Gr_n^\R\big)
\end{equation*}
where the domain is the polynomial ring in the $\omega_i$ situated in the appropriate degrees.
\begin{theorem}\label{thm:cohomology_gr_R}
	The map $\alpha_n$ is an isomorphism for all $n \in \N$.
\end{theorem}
Since $\omega_1\big(\gamma_\R^1\big) \in H^1(\RP^\infty)$ is the generator, we already know that $\alpha_1$ is an isomorphism.
We now first show that $\alpha_n$ is always injective.
To this end, consider the map
\begin{equation*}
	\varphi_n\colon (\RP^\infty)^{\times n} \to \Gr_n^\R
\end{equation*}
corresponding under the bijection $\Vect_\R^n((\RP^\infty)^{\times n}) \isom \big[(\RP^\infty)^{\times n}, \Gr_n^\R\big]$ to the product bundle $\big(\gamma_\R^1\big)^{\times n}$.
We obtain a map 
\begin{equation*}
	H^*\big(\Gr^\R_n\big) \xto{\varphi_n^*} H^*((\RP^\infty)^{\times n}) \isom H^*(\RP^\infty)^{\tensor n} \isom \F_2[u_1, \ldots, u_n]
\end{equation*}
using the Künneth isomorphism, where $u_i$ is the image of $u \in H^1(\RP^\infty)$ under $\pr_i^*\colon H^1(\RP^\infty) \to H^1((\RP^\infty)^{\times n})$.
We study the composite 
\begin{align*}
	\varphi_n^* \circ \alpha_n\colon \F_2[\omega_1, \ldots, \omega_n] &\to \F_2[u_1, \ldots, u_n] \\
	\omega_i &\mapsto \omega_i\big(\big(\gamma_\R^1\big)^{\times n}\big)
\end{align*}
By the Whitney product formula, we have $\omega\big(\big(\gamma_\R^1\big)^{\times n}\big) = (1 + u_1) \cdot \ldots \cdot (1 + u_n)$.
Hence, 
\begin{equation*}
	\omega_i((\gamma_\R^1)^{\times n}) = \smashoperator{\sum_{0 < j_1 < \ldots < j_i \leq n}} u_{j_1} \cdot \ldots \cdot u_{j_i} \eqcolon e_i(u_1, \ldots, u_n)
\end{equation*}
The polynomials $e_i(u_1, \ldots, u_n)$ are called the \strong{elementary symmetric polynomials}\index{symmetric polynomial!elementary} in $n$ variables.
For example, $e_1(u_1, \ldots, u_n) = u_1 + \ldots + u_n$ and $e_n(u_1, \ldots, u_n) = u_1 u_2 \cdots u_n$.
They are called \emph{symmetric} because they are invariant under the $\Sigma_n$-action on $\F_2[u_1, \ldots, u_n]$ permuting the $u_i$'s.
\begin{proposition}
	The $e_i(u_1, \ldots, u_n)$ are algebraically independent, i.e. there are no polynomial relations among them, and $(\varphi_n)^* \circ \alpha_n$ is injective.
\end{proposition}
\lecture{15.12.23}
\begin{smallproof}
	We proceed by induction on $n$.
	The case $n = 1$ is clear.
	Assume now that we have proved the statement for the $e_i(u_1, \ldots, u_{n - 1})$ and assume that $0 \neq f \in \F_2[x_1, \ldots, x_n]$ is a polynomial with $f(e_1, \ldots, e_n) = 0$ of smallest degree when considered as an element of $\F_2[x_1, \ldots, x_{n - 1}][x_n]$.
	Write $f = f_0 + f_1 \cdot x_1 + \ldots + f_d \cdot x_n^d$ with $f_i \in \F_2[x_1, \ldots, x_{n - 1}]$ and $d = \deg f$.
	If $f_0 = 0$, then we can write $f = x_n \cdot \tilde{f}$ for some $f \in \F_2[x_1, \ldots, x_n]$ and hence $0 = f(e_1, \ldots, e_n) = e_n \cdot \tilde{f}(e_1, \ldots, e_n)$ implies that $\tilde{f}(e_1, \ldots, e_n) = 0$, contradicting minimality of $\deg f$.
	Therefore, $f_0$ must be nontrivial.

	We now apply the ring map $\F_2[u_1, \ldots, u_n] \to \F_2[u_1, \ldots, u_{n - 1}]$ which sends $u_n$ to 0.
	This takes $e(u_1, \ldots, u_n)$ to $e_i(u_1, \ldots, u_{n - 1})$ if $i < n$ and $e_n(u_1, \ldots, u_n) = u_1 \cdots u_n$ to 0.
	On the other hand, the same ring map $\F_2[x_1, \ldots, x_n] \to \F_2[x_1, \ldots, x_{n - 1}]$ considered on the $x_i$ instead of the $u_i$ takes $f$ to $f_0$, 
	Hence we obtain $0 = f_0(e_1, \ldots, e_{n - 1}) \in \F_2[u_1, \ldots, u_{n - 1}]$, contradicting the induction assumption.
\end{smallproof}
Moreover, one can in fact show that the $e_i(u_1, \ldots, u_n)$ are in fact polynomial generators of the subring $\F_2[u_1, \ldots, u_n]^{\Sigma_2}$ of symmetric polynomials.
\begin{corollary}
	The map $\alpha_n$ is injective for all $n$.
\end{corollary}
\begin{smallproof}
	We just showed that $\varphi_n^* \circ \alpha_n$ is injective, hence so is $\alpha_n$.
\end{smallproof}
To show surjectivity, we use an inductive argument built on the following:
Let $\F = \R$ or $\C$.
Again we give $\gamma_\F^n$ a euclidean metric from its embedding in $\Gr_n^\F \times \F^\infty$.
Hence we can consider the sphere bundle $S(\gamma_\F^n)$ by restricting to vectors of length 1 in each fibre.
\begin{proposition}\label{prop:spherebdlhtpygrassmannian}
	Let $\F = \R$ or $\C$.
	There is a homotopy equivalence $S(\gamma_\F^n) \htpyeqv \Gr^\F_{n - 1}$ for every $n \geq 1$.
\end{proposition}
\begin{smallproof}
	$S(\gamma_\F^n)$ consists of pairs $(V, v)$ of $n$-dimensional subspaces $V \subset \F^\infty$ and a unit vector $v \in V$.
	We obtain a map
	\begin{align*}
		\phi\colon S(\gamma_\F^n) &\to \Gr^\F_{n - 1} \\
		(V, v) &\mapsto \langle v\rangle^\perp = \{x \in V \mid x \perp V\}
	\end{align*}
	In the other direction, we define
	\begin{align*}
		\psi\colon \Gr^\F_{n - 1} &\to S(\gamma_\F^n) \\
		W &\mapsto (g(\F \dsum W), g(1, 0))
	\end{align*}
	where $g$ is the linear isomorphism $\F \dsum \F^\infty \to \F^\infty$ that shifts coordinates by 1 to the right.
	The composite $\phi \circ \psi$ sends $W \subseteq \F^\infty$ to $0 \dsum W \subseteq \F \dsum \F^\infty \xto[\isom]{g} \F^\infty$, with the straight line homotopy from $\id_{\F \dsum \F^\infty}$ to $\F^\infty \incl \F \dsum \F^\infty$ showing that this is homotopic to the identity on $\Gr^\F_{n - 1}$.
	For the other composite, $\psi \circ \phi$ sends $(V, v)$ to $(\F \dsum \langle v \rangle^\perp \subset \F \dsum \F^\infty \xto[\isom]{g} \F^\infty, (1, 0))$, and a composition of straight line homotopies to the odd/even parts again shows that this is homotopic to the identity on $S(\gamma_\F^n)$.
\end{smallproof}
Alternatively, one notes that $\phi$ is a fibre bundle with fibre over $W \in \Gr^\F_{n - 1}$ given by $S(W^\perp) = \{v \in \F^\infty \mid |v| = 1,\ w \perp v \text{ for all } w \in W\}$.
This is homeomorphic to $S^\infty$ and hence contractible.

We further note that the map
\begin{equation*}
	f\colon \Gr^\F_{n - 1} \xto[\isom]{\psi} S(\gamma_\F^n) \to \Gr^\F_n
\end{equation*}
classifies the $n$-dimensional bundle $\gamma_\F^{n - 1} \dsum \F$ since the fibre of $f^* \gamma_\F^n$ over $W \in \Gr^\F_{n - 1}$ is given by $g(\F) \dsum g(W)$:
The summand $g(\F)$ is 1-dimensional and independent of $W$ whereas $g(W)$ gives a bundle isomorphism to $\gamma^\F_{n - 1}$, again using that $g$ is homotopic to $\id_{\F^\infty}$.

In problem \ref{ex:gysinwang} we saw that for every fibre sequence $S^n \to E \to B$ with $n > 0$ and $B$ simply connected and any abelian group $A$ there is a long exact sequence of the form
\begin{equation*}
	\begin{tikzcd}[column sep = small]
		\cdots
				\ar[r]
			& H_{p - n}(B; A)
				\ar[r]
			& H_p(E; A)
				\ar[r]
			& H_p(B, A)
				\ar[r]
			& H_{p - n - 1}(B; A)
				\ar[r]
			& \cdots
	\end{tikzcd}
\end{equation*}
called the \strong{Gysin sequence}\index{Gysin sequence} of the sphere bundle.
We now discuss a cohomological version of this for more general $B$.

Let $S^n \to Y \to X$ be a fibre sequence with $n > 0$ and $X$ path-connected.
Let $R$ be a commutative ring and assume that the local system $H^n(F_{-}; R)$ on $X$ is isomorphic to the constant one.
Then the cohomological Serre spectral sequence for the fibre sequence is of the form
\begin{equation*}
	\underbrace{H^p(X; H^q(F_{-}; R))}_{\isom H^p(X; H^q(S^n; R))} \Rightarrow H^{p + q}(Y; R)
\end{equation*}
After a choice of isomorphism $H^n(S^n; R) \xto{\isom} R$, we further obtain
\begin{equation*}
	H^p(X; H^q(S^n; R)) \isom \begin{cases}
		H^p(X; R) 	& q = 0, n \\
		0 			& \text{else}
	\end{cases}
\end{equation*}
This makes use of the fact that the local system $H^0(F_{-}; R)$ is always constant since any continuous map of path-connected spaces induces the identity on $R \isom H^0({{-}}; R)$.
We obtain the following picture:
\begin{center}
	\pgfsetlayers{background,main}
	\tikzsetnextfilename{vecbun_gysin_spec_seq}
	\begin{tikzpicture}
		\matrix[
			spectral sequence/page,
			name = m, 
			column sep = {3.8em, between origins},
			row sep = 3ex, 
			column 1/.append style = {anchor = base},
			row 4/.style = {font = \scriptsize}] {
				n &[-1.6em] H^0(X) & \cdots & H^{n + 1}(X) & \cdots \\
				\vdotswithin{0} & \phantom{0} \\
				0 & H^0(X) & \cdots & H^{n + 1}(X) & \cdots \\[-2ex]
				& 0 & \cdots & n + 1 & \cdots \\
		};

		\node[inner sep = 1pt, fill = white, below = 1.5ex of m-1-2] (Generator) {$\langle y \rangle$};
		\begin{scope}[
				every edge/.append style = {
					commutative diagrams/every arrow,
					every edge quotes/.append style = {
						execute at begin node = $,
						execute at end node = $,
						commutative diagrams/every label,
						inner sep = 1pt,
						fill = white,
					},
				},
			]
			\path (m-1-2) edge[draw = none, sloped, "\textstyle=" commutative diagrams/description] (Generator);
		\end{scope}

		\coordinate (Origin) at (m-1-2.west |- m-3-2.south);

		\draw[spectral sequence/axis] (Origin) -- (Origin |- m-1-2.north) -- ++(0, 1) node[left] (q) {$q$};
		\draw[spectral sequence/axis] (Origin) -- (m-3-5.east |- m-3-4.south) -- ++(1, 0) node[below] (p) {$p$};

		\coordinate (Top Right) at (p |- q);

		\coordinate (Slightly Left of p) at ($(p) - (0.14, 0)$);
		\coordinate (Slightly Below q) at ($(q) - (0, 0.14)$);

		\begin{pgfonlayer}{background}
			\draw[spectral sequence/zero region] (Origin) rectangle (Slightly Left of p |- Slightly Below q); 
			\fill[white] (m-1-2.south west) rectangle (m-1-4.north -| Slightly Left of p);
			\fill[white] (m-3-2.south west) rectangle (m-3-4.north -| Slightly Left of p);
		\end{pgfonlayer}

		\node[spectral sequence/page label] at ($(Top Right) - (0.3, 0.3)$) {$E_2^{p, q}$};

		\path[spectral sequence/differentials] (m-1-2) edge["d_{n + 1}"] (m-3-4);
	\end{tikzpicture}
\end{center}
% TODO picture
Let $e \in H^{n + 1}(X; R)$ be the image of the generator $y \in E_2^{0, n}$ corresponding to 1 under the chosen isomorphism $E_2^{0, 1} \isom R$ under the differential $d_{n + 1}$, i.e. $e = d_{n + 1}(y) \in H^{n + 1}(X; R)$.
The class $e$ is called the \strong{Euler class}\index{Euler class!of a sphere bundle} of the spherical fibre sequence and depends on the chosen isomorphism $H^n(F_{-}; R) \isom \const(R)$, the constant $R$-valued system.
Since $H^m(X; R) \to E_2^{m, n}$, $x \mapsto x \cdot y$ is an isomorphism, it follows that $d_{n + 1}$ is determined by the Leibniz rule $d_{n + 1}(y \cdot x) = d_{n + 1}(y) x = e \smile x$.
By our description of the edge homeomorphism, it follows that
\begin{equation*}
	\ker(p^*\colon H^*(X; R) \to H^*(E; R)) = \img(e \smile {{-}}\colon H^{* - n + 1}(X; R) \to H^*(X; R))
\end{equation*}
Furthermore, the image of the map
\begin{equation*}
	\begin{tikzcd}
		H^*(Y; R)
				\ar[r, two heads]
			& E_\infty^{* - n, n}
				\ar[r, hook]
			& E_2^{* - n, n} \isom H^{* - n}(X; R)
	\end{tikzcd}
\end{equation*}
is given by the kernel of $e \smile {{-}}$.
Hence we have
\begin{corollary}
	Let $S^n \to Y \xto{p} X$ be a fibre sequence with $n \neq 0$, $X$ path-connected and $R$ a commutative ring with a choice of trivialization $H^n(E; R) \isom R$.
	Then there is an exact sequence of the form
	\begin{equation*}
		\begin{tikzcd}[column sep = 1em]
			\cdots
					\ar[r]
				& H^m(Y; R)
					\ar[r]
				& H^{m - n}(X; R)
					\ar[r, "e \smile {{-}}"]
				&[1.28em] H^{m + 1}(X; R)
					\ar[r, "p^*"]
				& H^{m + 1}(Y; R)
					\ar[r]
				& \cdots
		\end{tikzcd}
	\end{equation*}
	where $e \in H^{n + 1}(X; R)$ is the Euler class (depending on the choice of trivialization).
\end{corollary}
\begin{remark}
	\leavevmode
	\begin{enumerate}
		\item A trivialization $H^n(E; R) \isom R$ is called an \strong{$R$-orientation}\index{orientation!of fibre sequences} of the spherical fibre sequence.
		\item Every spherical fibre sequence has a unique $\F_2$-orientation since $\F_2$ has no non-trivial automorphisms.
	\end{enumerate}
\end{remark}
We further have
\begin{lemma}
	If $p\colon E \to B$ is a $(n + 1)$-dimensional euclidean real vector bundle with $B$ paracompct and associated $n$-dimensional sphere bundle $S(E) \xto{q} B$, then $e = \omega_{n + 1}(p) \in H^{n + 1}(B; \F_2)$.
\end{lemma}
\begin{proof}
	We first note that $\omega_{n + 1}(p) = f^* u \in H^{n + 1}(B; \F_2)$ where $f\colon (B, \emptyset) \to (E, E_0)$ is the zero section.
	In other words, $f^*$ is the composite $H^*(E, E_0; \F_2) \xto{i^*} H^*(E; \F_2) \xleftarrow[\isom]{p^*} H^*(B; \F_2)$.
	Indeed, we have $\Sq^{n + 1} u = u^2$ and $i^*(u) \smile u = u^2$ since
	\begin{equation*}
		\begin{tikzcd}
			H^*(E, E_0; \F_2) \times H^*(E, E_0; \F_2)
					\ar[r, "\smile"]
					\ar[d, "i^* \times \id"]
				& H^*(E, E_0; \F_2)
					\ar[d, equal]
			\\
			H^*(E; \F_2) \times H^*(E, E_0; \F_2)
					\ar[r, "\smile"]
				& H^*(E, E_0; \F_2)
		\end{tikzcd}
	\end{equation*}
	commutes by naturality of the relative cup products.
	Now let $b \in B$ and write $S(E_b)$ for the fibre $F_b$ over $b$.
	The relevant differential $d_{n + 1}\colon H^n(S(E_b); \F_2) \to H^{n + 1}(B; \F_2)$ is a transgression, hence we know it can be computed via
	\begin{equation*}
		\begin{tikzcd}[row sep = 0.3ex]
			H^n(S(E_b); \F_2)
					\ar[r, "\del"]
				& H^{n + 1}(S(E), S(E_b); \F_2) 
				& H^{n + 1}(B; \F_2)
					\ar[l, hook', swap, "q^*"]
			\\
			y
					\ar[r, mapsto]
				& \del y
				& d_{n + 1} y
					\ar[l, mapsto]
		\end{tikzcd}
	\end{equation*}
	We now note that the diagram
	\begin{equation*}
		\begin{tikzcd}
			H^n(S(E_b); \F_2)
					\ar[r, "\del"]
					\ar[d, swap, "\del", "\isom"']
				& H^{n + 1}(S(E), S(E_b); \F_2)
			\\
			H^{n + 1}(D(E_b), S(E_b); \F_2)
			\\
			H^{n + 1}(D(E), S(E_b); \F_2)
					\ar[u]
					\ar[uur]
				& H^{n + 1}(D(E), S(E); \F_2)
					\ar[l]
					\ar[uu, swap, "j^*"]
		\end{tikzcd}
	\end{equation*}
	commutes, which implies that $\del y = j^* u$ where $j\colon (S(E), S(E_b)) \to (D(E), S(E))$ is the inclusion.
	Furthermore, we have a commutative diagram
	\begin{equation*}
		\begin{tikzcd}
			H^{n + 1}(S(E), S(E_b); \F_2)
				& H^{n + 1}(B, *; \F_2)
					\ar[l, hook', "p^*"]
					\ar[d, hook, "p^*"]
					\ar[r, "\isom"]
				& H^{n + 1}(B; \F_2)
					\ar[d, "p^*", "\isom"']
			\\
			H^{n + 1}(D(E), S(E); \F_2)
					\ar[u, "j^*"]
					\ar[r]
					\ar[urr, "f^*" near end, to path = {
						|- ($(\tikzcdmatrixname-2-3.south east) + (.22, -.45)$) \tikztonodes
						|- (\tikztotarget) 
					}]
				& H^{n + 1}(D(E), S(E_b); \F_2)
					\ar[ul]
					\ar[r]
				& H^{n + 1}(D(E); \F_2)
		\end{tikzcd}
	\end{equation*}
	which implies that $p^*(f^* u) = j^* u = \del y$ and hence $f^* u = d_{n + 1} y = e$.
\end{proof}
\lecture{18.12.23}
We now show that $\alpha_n\colon \F_2[\omega_1, \ldots, \omega_n] \to H^*\big(\Gr_n^\R; \F_2\big)$ is an isomorphism for all $n$ by induction on $n$.
The case $n = 1$ is clear.
Assume thus that the statement holds up to some $n \geq 1$ and consider the case for $n + 1$:
By proposition \ref{prop:spherebdlhtpygrassmannian} we know that $S\big(\gamma_\R^{n + 1}\big) \isom \Gr_n^\R$ and that there is a long exact sequence
\begin{equation*}
	\begin{tikzcd}[column sep = small]
		\cdots 
				\ar[r]
			& H^*\big(\Gr^\R_{n + 1}; \F_2\big) 
				\ar[r, "\omega_{n + 1} \cdot"]
			&[1.4em] H^{* + n + 1}\big(\Gr^\R_{n + 1}; \F_2\big)
				\ar[r]
			& H^{* + n + 1}\big(\Gr^\R_n; \F_2\big)
				\ar[r]
			& \cdots
	\end{tikzcd}
\end{equation*}
By induction, we know that $H^*\big(\Gr^\R_n; \F_2\big)$ is generated by $\omega_1, \ldots, \omega_n$.
We know that each $\omega_i$ lifts to a class of $H^*\big(\Gr^\R_{n + 1}; \F_2\big)$ of the same name (this uses the stability property $\omega_i(\xi \dsum \epsilon) = \omega_i(\xi)$).
Since $H^*\big(\Gr^\R_{n + 1}; \F_2\big) \to H^*\big(\Gr^\R_n; \F_2\big)$ is a ring map, it must hence be surjective.
This implies that the Gysin sequence splits up into short exact sequences.
We then get a comparison map
% TODO fugly
\begin{equation*}
	\begin{tikzcd}[column sep = smallish]
		0 
				\ar[r]
			& \F_2[\omega_1, \ldots, \omega_{n + 1}]_*
				\ar[r, "\omega_{n + 1} \cdot"]
				\ar[d, "\alpha_{n + 1}"]
			&[.3em] \F_2[\omega_1, \ldots, \omega_{n + 1}]_{* + n + 1}
				%\ar[r, "\omega_{n + 1} \mapsto 0"]
				\ar[d, "\alpha_{n + 1}"]
				\ar[dd, rounded corners = 10pt, swap, "\omega_{n + 1} \mapsto 0", to path = {[pos = 1]
					-- ([xshift = 2em] \tikztostart.east)
					|- ($(\tikzcdmatrixname-2-3)!0.67!(\tikzcdmatrixname-3-3)$) \tikztonodes
					-| ([xshift = -1.5em] \tikztotarget.west)
					-- (\tikztotarget)
				}]
		\\
		0 
				\ar[r]
			& H^*\big(\Gr^\R_{n + 1}; \F_2\big)
				\ar[r, "\omega_{n + 1} \cdot"]
			& H^{* + n + 1}\big(\Gr^\R_{n + 1}; \F_2\big)
				\ar[dd, rounded corners = 10pt, to path = {
					-- ([xshift = 1.5em] \tikztostart.east)
					|- ($(\tikzcdmatrixname-2-3)!0.33!(\tikzcdmatrixname-3-3)$) \tikztonodes
					-| ([xshift = -3.5em] \tikztotarget.west)
					-- (\tikztotarget)
				}]
		\\[4ex]
			& & \F_2[\omega_1, \ldots, \omega_n]_{* + n + 1}
				\ar[r]
				\ar[d, "\alpha_n", "\isom"']
			& 0
		\\
			& & H^{* + n + 1}\big(\Gr^\R_n; \F_2\big)
				\ar[r]
			& 0
	\end{tikzcd}
\end{equation*}
Since all graded rings in this diagram are concentrated in non-negative degrees, it follows by induction on the degree of $H^*\big(\Gr^\R_{n + 1}; \F_2\big)$ that $\alpha_{n + 1}$ must be an isomorphism (as the diagram shows that if $\alpha_{n + 1}$ is an isomorphism in degree $k$, then by the 5-lemma it is also in degree $k + n + 1$).
This finally concludes the proof of theorem \ref{thm:cohomology_gr_R}.
\begin{corollary}
	The map 
	\begin{equation*}
		\varphi^*_n\colon H^*\big(\Gr^\R_n; \F_2\big) \to H^*((\RP^\infty)^{\times n}; \F_2)
	\end{equation*}
	is injective.
\end{corollary}
\begin{proof}
	We already saw that $\varphi^*_n \circ \alpha_n$ is injective, and we now know that $\alpha_n$ is an isomorphism.
\end{proof}
\begin{proposition}
	If two characteristic classes $\beta_1, \beta_2\colon \Vect_\R^n({{-}}) \to H^m({{-}}; \F_2)$ over paracompact spaces agree on all bundles that decompose into sums of line bundles, then $\beta_1 = \beta_2$.
\end{proposition}
\begin{proof}
	The bundle $\varphi^*_n \gamma_\R^n$ over $(\RP^\infty)^{\times n}$ is by definition a sum of line bundles, namely $\varphi_n^* \gamma_R^n = \pi_1^* \gamma_\R^1 \dsum \cdots \dsum \pi_n^* \gamma_\R^1$ where $\pi_i\colon (\RP^\infty)^{\times n} \to \RP^\infty$ is the $i$th projection.
	The assumption then guarantees that
	\begin{equation*}
		\varphi_n^* \beta_1(\gamma_\R^n) = \beta_1(\varphi_n^* \gamma_\R^n) = \beta_2(\varphi_n^* \gamma_\R^n) = \varphi_n^* \beta_2(\gamma_\R^n)
	\end{equation*}
	since $\varphi_n^*$ is injective, it follows that $\beta_1(\gamma_\R^n) = \beta_2(\gamma_\R^n)$ and by universality that $\beta_1 = \beta_2$.
\end{proof}
\begin{corollary}
	The Stiefel-Whitney classes are uniquely determined by axioms \ref{ax:stiefwhittriviality}--\ref{ax:stiefwhitnormalization} of theorem \ref{thm:existenceofswclasses} over paracompact spaces.
\end{corollary}
\begin{proof}
	Assume $\omega_i$ and $\omega'_i$ both satisfy the axioms and let $\omega$ and $\omega'$ be the associated total classes.
	We have seen that $\omega(\gamma_\R^1) = 1 + u = \omega'(\gamma_\R^1) \in H^\pi(\RP^\infty; \F_2) \isom \F_2[[u]]$.
	Hence, by universality and naturality, $\omega$ and $\omega'$ must agree for all line bundles over paracompact spaces:
	If $\xi = \xi_1 \dsum \cdots \dsum \xi_n$ and all $\xi_i$ are line bundles, the axiom \ref{ax:whitneyproduct} implies that
	\begin{equation*}
		\omega(\xi) = \prod_{i = 1}^n \omega(\xi_i) = \prod_{i = 1}^n \omega'(\xi_i) = \omega'(\xi)
	\end{equation*}
	By the preceding proposition, this implies $\omega = \omega'$.
\end{proof}

\subsection{The Complex Case}
We now want to compute characteristic classes over paracompact spaces of the $\Vect_\C^n({{-}}) \to H^m({{-}}; \Z)$, or in other words the cohomology $H^*\big(\Gr^\C_n; \Z\big)$.
We want to use the same inductive process through the identification $S(\gamma_\C^n) \isom \Gr^\C_{n - 1}$ and the Gysin sequence.
For this, we need to understand the local system $H^*(S(E_{-}); \Z)$ where $p\colon E \to B$ is complex vector bundle.

Fix a generator $\beta \in H^2(D(\C), S(\C); \Z)$ throughout this section.
Then the external relative cross-power $\beta_{\C^n} \coloneq \beta^{\times n}$ forms a generator of $H^{2n}(D(\C^n), S(\C^n); \Z)$.
If $V$ is any hermitian vector space, we choose an isometry $\varphi\colon V \xto{\isom} \C^n$ and obtain a generator $\varphi^* \beta_{\C^n} \in H^{2n}(D(V), S(V); \Z)$.
Given another choice $\psi\colon V \xto{\isom} \C^n$ of isometry, the composite of $\psi \circ \varphi^{-1}\colon \C^n \to \C^n$ is an element of $\Uni(n)$ and hence acts by a degree 1 map on $S(\C^n)$.
It follows that $\psi^* \beta_{\C^n} = (\psi \circ \varphi^{-1} \circ \varphi)^* \beta_{\C^n} = \varphi^* (\psi \circ \varphi^{-1})^* \beta_{\C^n} = \varphi^* \beta_{\C^n}$.
Hence we obtain a canonical generator $\beta_V \in H^{2n}(D(V), S(V); \Z)$ independent of the choice of (isometric) identification $V \xto{\isom} \C^n$.

Now let $p\colon E \to B$ be a hermitian $n$-dimensional vector bundle with sphere bundle $q\colon S(E) \to B$.
For $b \in B$, we obtain an isomorphism
\begin{equation*}
	\begin{tikzcd}[row sep = 0.3ex]
		\delta_b\colon H^{2n - 1}(S(E_b); \Z)
				\ar[r, "\del", "\isom"']
			& H^{2n}(D(E_b), S(E_b); \Z)
				\ar[r, swap, "\isom"]
			& \Z
		\\
			& \beta_{E_b}
			& 1
				\ar[l, mapsto]
	\end{tikzcd}
\end{equation*}
\begin{lemma}
	$\delta_b$ defines an isomorphism from the local system on $H^{2n - 1}(S(E_{-}); \Z)$ to the constant one on $\Z$, i.e. a $\Z$-orientation of $q\colon S(E) \to B$.
\end{lemma}
\begin{proof}
	Given a path $\omega\colon [0, 1] \to B$ from $b$ to $b'$, we obtain maps
	\begin{equation*}
		\begin{tikzcd}
			H^{2n - 1}(S(E_b); \Z)
				& H^{2n - 1}(S(\omega^* E); \Z)
					\ar[l, swap, "\isom"]
					\ar[r, "\isom"]
				& H^{2n - 1}(S(E_b))
		\end{tikzcd}
	\end{equation*}
	% TODO ?
	Now $\omega^* E$ is a vector bundle over $[0, 1]$, hence isomorphic to a trivial bundle, and for trivial bundles the statement is clear.
\end{proof}
We can now show:
\begin{theorem}\label{thm:cohomology_gr_C}
	$H^*\big(\Gr^\C_n; \Z\big)$ is a polynomial ring on classes $c_1, \ldots, c_n$ with $|c_i| = 2i$ uniquely determined by the following two properties:
	\begin{enumerate}
		\item $c_n$ equals the Euler class for $S(\gamma_\C^n)$ for the $\Z$-orientation just constructed.
			% TODO which map?
		\item For $i < n$, the map $H^*\big(\Gr^\C_n; \Z\big) \to H^*\big(\Gr^\C_{n - 1}; \Z\big)$ sends $c_i$ to $c_i$.
	\end{enumerate}
	The associated characteristic classes are called the \strong{Chern classes}\index{Chern class} for complex vector bundles.
\end{theorem}
\begin{proof}
	We proceed by induction.
	The case $n = 0$ is trivial.
	Assuming the statement holds for some $n - 1$, we are forced to set $c_n \coloneq e \in H^{2n}\big(\Gr^\C_n; \Z\big)$ for $e$ the Euler class of $S(\gamma_\C^n)$ by the first property.
	The Gysin sequence gives a long exact sequence
	\begin{equation*}
		\begin{tikzcd}[column sep = smallish]
			\cdots
					\ar[r]
				& H^*\big(\Gr^\C_n; \Z\big)
					\ar[r, "c_n \cdot"]
				& H^{* + 2n}\big(\Gr^\C_{2n}; \Z\big)
					\ar[r]
				& H^{* + 2n}\big(\Gr^\C_{n - 1}; \Z\big)
					\ar[r]
				& \cdots
		\end{tikzcd}
	\end{equation*}
	It follows that $H^j\big(\Gr^\C_n; \Z\big) \to H^j\big(\Gr^\C_{n - 1}; \Z\big)$ is an isomorphism for $j < 2n - 1$ since in this case $H^{j - 2n}\big(\Gr^\C_n; \Z\big) = H^{j - 2n + 1}\big(\Gr^\C_n; \Z\big) = 0$ as $j - 2n + 1 < 0$.
	Thus, the elements $c_1, \ldots, c_{n - 1} \in H^*\big(\Gr^\C_{n - 1}; \Z\big)$ lift to unique elements $c_1, \ldots, c_{n - 1} \in H^*\big(\Gr^\C_n; \Z\big)$.
	We have hence defined all the $c_i$.
	It remains to show $H^*\big(\Gr^\C_n; \Z\big) \isom \Z[c_1, \ldots, c_n]$, which is the same argument as in the real case:
	Since we already know that $H^*\big(\Gr^\C_{n - 1}; \Z\big) \isom \Z[c_1, \ldots, c_{n - 1}]$ and we have lifted the $c_i$, the map $H^*\big(\Gr^\C_n; \Z\big) \to H^*\big(\Gr^\C_{n - 1}; \Z\big)$ must be surjective and the Gysin sequence collapses into short exact sequences.
	By induction on the degree, one shows that these short exact sequences are together isomorphic to $0 \to \Z[c_1, \ldots, c_n] \xto{c_n \cdot} \Z[c_1, \ldots, c_n] \xto{c_n \mapsto 0} \Z[c_1, \ldots, c_{n - 1}] \to 0$.
\end{proof}
\begin{remark}\index{Chern class!axioms for}
	Like the Stiefel-Whitney classes, the Chern classes $c_i(\xi)$ are uniquely characterized by the following four axioms, if we set $c_0(\xi) = 1$:
	\begin{enumerate}
		\item $c_0(\xi) = 1$ and $c_i(\xi) = 0$ if $i > \dim_\C(\xi)$.
		\item \strong{Naturality}: $c_i(f^* \xi) = f^* c_i(\xi)$.
		\item \strong{Product formula}: $c_i(\xi \dsum \eta) = \sum_{j = 0}^i c_j(\xi) \smile c_{i - j}(\eta)$.
		\item $c_1(\gamma_\C^{1, 2}) \in H^2(\CP^1; \Z)$ agrees with the Euler class for the sphere bundle $S(\gamma_\C^{1, 2})$ (which is the Hopf bundle $S^1 \to S^3 \xto{\eta} \CP^1$).
	\end{enumerate}
\end{remark}
% TODO cite
Seeing that the Chern classes we constructed satisfy these axioms is clear for axioms 1, 2, and 4, but more for is needed for 3, see Milnor-Stasheff section 14.4.
Uniqueness is then the same argument as in the real case, using the map 
\begin{equation*}
	(\CP^\infty)^{\times n} \to \Gr^\C_n
\end{equation*}
classifying $\gamma_\C^1 \times \ldots \times \gamma_\C^1$.
